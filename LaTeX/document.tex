\documentclass{svmono}
\begin{document}
In the third and second centuries BCE, the Sanskrit scholar Piṅgala recorded an algorithm for generating poetic metre in which the Fibonacci sequence is implicit. Subsequent writers in the tradition over more than a millenium continued to note properties of the resulting syllabic groupings or \textit{mātrā-vṛtta}, most explicitly around 1150 when the scholar Hemacandra observed `the sum of the last and the one before the last is the number ... of the next mātrā-vṛtta'. 
Starting at around same time, the golden ratio $\tau$ can be found in the ancient Greek concept of `extreme and mean ratio' recorded in Euclid's \textit{Elements}. 
A connection between the golden ratio and Fibonacci numbers was first made in Simon Jacob's 1564 \textit{Ein New und Wolgegründt Rechenbuch} and was  re-discovered by French mathematicians studying the Euclidean algorithm in the nineteenth century when the sequence was known as the Lamé sequence. It was the Édouard Lucas who renamed the sequence as the Fibonacci sequence in the 1870s. For this reason the Lucas numbers should be pronounced in the same way as the French surname.
\end{document}