

%%%%%%%%%%%%%%%%% typefaces and colours


\usepackage{xcolor}
\definecolor{CylinderColour}{rgb}{0.86, 0.86, 0.64}
\definecolor{parastichy1}{rgb}{0.71, 0.3214, 0.27}
\definecolor{parastichy2}{rgb}{0.46, 0.2947, 0.591}
\definecolor{parastichy3}{rgb}{0.45, 0.54, 0.24}
\definecolor{parastichy4}{rgb}{0.95, 0.94,0.94}
\definecolor{parastichy5}{rgb}{0.21, 0.1875, 0.15}
\newcommand{\jHeadingColour}{\color{parastichy1}}




%%%%%%%%%%%%% maths
\renewcommand{\implies}{\Rightarrow} % remove horrid long arrow

\newcommand{\pvec}[1]{{\mathbf p}_{#1}}
\newcommand{\phatvec}[1]{\hat{\mathbf p}_{#1}}
\newcommand{\rpvec}[1]{\mathbf{r}_{#1}}
\newcommand{\Pvec}[1]{{\mathbf P}_{#1}}
\newcommand{\mvec}{\pvec{m}}
\newcommand{\nvec}{\pvec{n}}
\newcommand{\fvec}{\mathbf{f}}
\newcommand{\svec}{\mathbf{s}}
\newcommand{\tvec}{\mathbf{t}}

\newcommand{\jvec}[1]{\mathbf{#1}}

\newcommand{\jhalf}{\textstyle{\frac{1}{2}}}
\newcommand{\jfrac}[2]{\textstyle{\frac{#1}{#2}}}

\usepackage{tikz}
\newcommand*\circled[1]{%
\protect\tikz[baseline=(char.base)]{
		\node[shape=circle,draw,inner sep=1pt] (char) {#1};}%
}

\newcommand{\jpoint}[1]{\circled{{\textsf{#1}}}}
%\newcommand{\jpoint}[1]{\whitestone{{#1}}}

\newcommand{\tends}{\rightarrow}


\newcommand{\gp}[1]{\textsf{(#1)}}
\newcommand{\gphat}[2]{\textsf{%
(\ensuremath{\textsf{#1}},\ensuremath{\hat{\textsf{#2}}})%
}}%
\newcommand{\gphatnothat}[3]{\textsf{%
(\ensuremath{\textsf{#1}},\ensuremath{\hat{\textsf{#2}}},\ensuremath{{\textsf{#3}}})%
}}%
\newcommand{\gpbug}[1]{(#1)}
 

\newcommand{\pp}[1]{\ppstyle{(#1)}}
\newcommand{\pphat}[2]{\textsf{
		(\ensuremath{\textsf{#1}},\ensuremath{\hat{\textsf{#2}}})
}}%
\newcommand{\ppstyle}[1]{\textsf{#1}}

\newcommand{\branch}[1]{\textsf{#1}}


\newcommand{\Mod}[1]{\ (\mathrm{mod}\ #1)}
\newcommand{\Sign}{\mathrm{sign}}

\newcommand{\jC}{\mathbb{C}}
\newcommand{\jR}{\mathbb{R}}
\newcommand{\jS}{\mathbb{S}}
\newcommand{\jN}{\mathbb{N}}
\newcommand{\jZ}{\mathbb{Z}}

\newcommand{\jq}{{\jHeadingColour q}}
\newcommand{\jqi}{{\jHeadingColour q_i}}
\newcommand{\jqn}[1]{{\jHeadingColour #1}}
\newcommand{\jqix}[1]{{\jHeadingColour q_{#1}}}

\newcommand{\jFarey}[4]{\left[\displaystyle\frac{#1}{#2},\frac{#3}{#4}\right]}




\newcommand{\mmafig}[3]{\jdofig{#1}{#2}{#3}{./MathematicaFigures}{.pdf}}
\newcommand{\jpgfig}[3]{\jdofig{#1}{#2}{#3}{./Figures}{.jpg}}
\newcommand{\pdffig}[3]{\jdofig{#1}{#2}{#3}{./Figures}{.pdf}}

\newlength{\jfigwidth}
\setlength{\jfigwidth}{\textwidth}
\newcommand{\jdofig}[5]{
	\begin{figure}
		\caption{#2}\label{fig:#1}
		\includegraphics[width=#3\jfigwidth]{#4/#1#5}
		\end{figure}
}
\newcommand{\sidefig}[3]{
	\begin{figure}
		\sidecaption
		\caption{#2}
		\label{fig:#1}
		%\centering
		\includegraphics[width=#3\jfigwidth]{./Figures/#1.jpg}
	\end{figure}
}


\newcommand{\jEndChapter}{}
\newcommand{\jNote}[1]{\footnote{#1}}



