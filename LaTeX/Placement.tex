
 
\chapter{Placement models}
\label{ch:placement}
\abstract{Up to now we have studied only static lattices, which fix the location of every node simultaneously, and then we smoothly transformed the lattice as a whole. This approach cannot represent the one-by-one developmental choices for node commitment, so now we will consider more general node-placement models. This Chapter, and the next, review a range of models that have usefully been deployed to understand Fibonacci phyllotaxis in this way. 
}

 A modern book-length survey of placement models, with a rather different perspective to this one, is~\autocite{barabePhyllotacticPatternsMultidisciplinary2020}. Recent reviews mainly devoted to the biological issues include~\cite{traasPhyllotaxis2013} and~\cite{godinPhyllotaxisGeometricCanalization2020}. One survey of models in plant morphogenesis more broadly is  ~\autocite{prusinkiewiczComputationalModelsPlant2012}.
 
 Although we no longer require all node positions to form a lattice, we retain the cylindrical coordinates $(x,z)$ of section~\ref{sec:latticedef}, and suppose that the position of all the points $\jpoint{i} = (x_i,z_i)$ is fixed up to the $(k-1)$-th, with $z_i$ non-decreasing. The node-placement model of Chapter~\ref{sec:bio} is now expressible as a function whose output is $\jpoint{k}$ and whose input is the $\jpoint{i}$s, $i=0,\ldots, k-1$,
although in practice the output will only depend on recent or nearby node-positions. The time at which the new node is placed might be an output of the model (as in Snows' model) but if we follow Hofmeister's model and choose to hold constant the inter-node time, or \textit{plastochrone},  then we have a discrete-time model. 


\section{The Douady-Couder ferromagnetic model}
\label{sec:DouadyCouderModel}

A seminal physical model for phyllotaxis was Douady and Couder's 1992 ferromagnetic system, in which charged drops of oil are dropped onto a cylindrical plate in a magnetic field that causes a centrifugal motion. A caricature of this model is shown in Figure~\ref{fig:Txb0901DCApparatus}.
\mmafig{Txb0901DCApparatus}{Simplified view of Douady and Couder's apparatus for generating phyllotactic patterns. A centrifugal magnetic gradient transports ferrofluid magnetic dipoles periodically dropped from the central pipette. Redrawn from~\cite{douadyPhyllotaxisDynamicalSelf1996a}.}{1}
If there is a very long time between successive drops, the drops will move off at random directions; if the process is speeded up slightly then mutual repulsion of the drops 
will ensure they move in successively opposite directions  as in the spots 0, 1 and 2 in part (a) of Figure~\ref{fig:F0902Douady1996Fig3crop}. However if the process is sped up still faster, then the same mutual repulsion squeezes out the new spot to the side.  It is this symmetry breaking that has placed the spot numbered 2 away from the line between the two earlier spots in part (b) of Figure~\ref{fig:F0902Douady1996Fig3crop} and initiated a spiral form.

\jpgfig{F0902Douady1996Fig3crop}{
		Different patterns of ferrofluid drops in Douady and Couder's experiment at different values of the parameter $G$ which control the rate of centrifugal movement. (a) When the time between drops is large compared to the time to be removed from the plate, each new drop is repelled only by the previous one and the node position  alternates by 180$^\circ$ each time: in the language of this book a distichous pattern $(1,1)$ is obtained with a divergence of $d=\jhalf$. (b) As the removal time decreases, a symmetry breaking bifurcation  to a  $(1,2)$ pattern appears with a divergence $d\approx 5/12$. (c) As the control parameter is further decreased higher order Fibonacci modes, here $(5,8)$, appear with divergence $d\approx{1}/{\tau^2}$.  \copyright Academic Press, 1996; from~\cite{douadyPhyllotaxisDynamicalSelf1996a}.}{0.25}
		%{From ~\cite{douadyPhyllotaxisDynamicalSelf1996a}}


\todo{reword}
\jpgfig{F0903Douady}{		
	Simulation results for a model of the Douady-Couder system,
	showing how only the Fibonacci branch is reachable with smooth changes of the control parameter $G$ down from 1.  The graph shows  $G$, corresponding to relative magnitude of time to clear the plate to inter-drop period, against emergent divergence angle $\phi= 2\pi d$.
	For each $G$, and for each of six divergence angles between $60^\circ$ and $180^\circ$, a spiral pattern was initiated with that angle and then run for 200 further iterations, and all spiral (i.e. periodicity 1) angles and higher period solutions recorded. 
	Recalculated from the model in~\cite{douadyPhyllotaxisDynamicalSelf1996a}.
}{1}

A model of this spiral generation-phenomenon  can be built by assuming that the $k$\nobreakdash-th drop, arriving periodically at the centre of the disk at time $t_k= k T$, has its angle determined as soon as it has left a small inner ring of radius $z_c$, and then moves off with displacement $z(t)$ so has polar co-ordinates $\jpoint{k}=(z_c+z(t-t_k), \theta_k)$ with  $z(0)=0$, or the cylindrical co-ordinates above $(x=2\pi \theta_k,z(t))$. Douady and Couder modelled the radial displacement typically as exponential growth $z(t)=z_c\exp (Gt/T)$. 
 The choice of angle $\theta_n$ for the $n$-th and newest drop is made by minimising $\sum_{i<n} E(|(\jpoint{i}, (z_c,\theta_n) )|)$, where $|\cdot,\cdot|$ is the distance  between two points on the surface of the cylinder and $E$ is an energy function of the form say $E(r)=1/r^3$.  For these and similar functional forms, the system is controlled by the dimensionless parameter $G$. 
  
  Figure~\ref{fig:F0903Douady} shows some results of simulation of this model, with a clear path through several Fibonacci transitions.  When versions of these  results were first published in the 1990s, they formed the first demonstration of a mechanism for Fibonacci phyllotaxis realised in a model which could be plausibly connected to known biology. It was a brilliant proof-of-concept for the possibilities of modelling in biology. 
  
  In the quarter-century since, though, the model itself has only occasionally   been  re-used in plant developmental biology~\cite{couderInitialTransitionsOrder1998,yonekuraMathematicalModelStudies2019}. One reason for this is the 
  energy function $E(\cdot)$. While this is a mathematically natural formalism, it cannot easily be connected to any measurable phyllotactic process. In addition, while the model constraint that only one drop can be dropped at a time is crucial to its success, it is a limitation as a phyllotactic model because it forbids patterns in which two nodes are formed near simultaneously at the same height on the stem, although there is ample evidence that this happens biologically. 
  
 


\section{Energy based models}
\label{sec:energy}
The Douady-Couder model of section~\ref{sec:DouadyCouderModel} imposed a time delay between new nodes and a centrifugal movement, which together ensured that new nodes could not be placed too closely to old ones. It then defined an energy function for the new pattern, so that minimising the energy step-by-step then placed nodes as close as possible given the first constraint.  By contrast the stacked-coin model uses a fixed disk to keep the nodes separate, whilst using `gravity', that is the requirement that the new node is in contact with the old ones, to keep them close. Both of these models yield  a typical inter-node distance of say $D$, and can be thought of as defining a  node-node potential energy which is large before $D$, then drops off sharply to a minimum near $D$ before more smoothly re-increasing. 

Given such a potential energy representing a mixture of close-range repulsion and long-range attraction, it is possible to define an energy for cylindrical node patterns, and then to analyse the patterns which are local minima of this energy. Such an energy is of the form 
$E=\sum U(\left|\jpoint{n}-\jpoint{m}\right|)$ as a sum of distances over node-pairs.  {A key difference between our application and most physical ones is that there is no mechanism for \textit{global} minimisation of the energy: a pattern may be built up node-by-node, locally minimising this interaction energy, but in general the final pattern is not a global minimum of all possible patterns on the same cylinder with the same mean $D$.}
Nevertheless it was these formal similarities with theoretical physics which led Levitov and others to the renormalisation technique~\autocite{levitovFibonacciNumbersBotany1991}. If we restrict to patterns which are lattices, then the energy of a lattice can be analysed by renormalising the lattice, which simply scales all the node-pair distances by the same scalar. This analysis thus rediscovers the  lattice bifurcation structure of the van Iterson diagram.

 Early node placement models such as those of Adler~\autocite{adlerConsequencesContactPressure1977} were criticised by Lee and Levitov~\autocite{leeUniversalityPhyllotaxisMechanical1998}, I think unfairly, as not corresponding directly to any biological system. By contrast,
 Lee and Levitov claimed a deep universality for energy-based models as a justification for pruning the van Iterson tree to Fibonacci structure. However the energy function is defined in terms of `distance', which has an intuitively reasonable interpretation in many field-based physical systems but is very hard to interpret here.
 
Nevertheless, this is a convenient formalism to express how the ratio of a  fixed pattern  length scale changes relative to developmental geometry. In a pioneering study in 1998, Yves Couder used such a model to show that it could generate some of the non-Fibonacci solutions described in Chapter~\ref{ch:empirical} and seen in a sample, albeit a relatively small one of sunflower seedheads~\cite{couderInitialTransitionsOrder1998}. 
 While these energy models have been relatively neglected in recent research relative to the stacked-disk models of the next chapter, their flexibility 
 in expressing the idea of relatively soft disks and limited local rearrangements means they may well continue as a helpful tool. 

\section{Partial differential equation models}
\subsection{An early reaction-diffusion approach}
 In 1951 Alan Turing wrote to a prominent British zoologist of the time that 
\begin{quotation}
	\dots my mathematical theory of embryology\dots is yielding to 
	treatment, and it will so far as I can see, give satisfactory 
	explanations of\\
	(i) gastrulation\\
	(ii) polygonally symmetrical structures, e.g. starfish, flowers\\
	(iii) leaf arrangements, in particular the way the Fibonacci \\
	series ($0,1,1,2,3,5,8,13,\dots$) comes to be involved\\
	(iv) colour patterns on some animals, e.g. stripes, spots and \\
	dappling\\
	(v) pattern on nearly spherical structures such as some Radiolara \dots\jNote{Letter from AM Turing to JZ Young 8 Feb 1951, King's College Cambridge AMT K.1.78.}
\end{quotation}
By `my theory of embryology' Turing meant primarily what we now call the Turing Instability, a  pattern formation mechanism for length-scale emergence in a reaction-diffusion PDE system. 
Turing's mechanism, which he later published in his celebrated 1952 paper~\autocite{turingChemicalBasisMorphogenesis1952} has been a mathematical blessing in its generality and a biological curse because of the same generality. With a few exceptions, biologists over the last 70 years have viewed the Turing instability as elegant but unhelpful or wrong for understanding the major questions in developmental biology. 	 Subsequent mathematical biology using the Turing Instability has been particularly rich in (iv), is probably the wrong approach for (i) and (ii), and is largely forgotten for  (v). But we are now in a position to see what Turing meant when he though it could help with (iii). 
The previous models of this Chapter abstracted known, or guessed, details of the molecular biology of node formation, into various representations of pattern formation on a cylinder with a typical length scale of $D$. It was the availability of such a model, in the form of the Turing Instability,  that allowed Turing to develop \textit{The Outline of the Development of the Daisy}. This was an unpublished manuscript in which he used a PDE model to explore transitions through lattice space and to try to find conditions for  the Hypothesis of Geometrical Phyllotaxis to be fulfilled.


\subsection{Auxin flow models}

As molecular details have accumulated, Turing's essentially arbitrary activator-inhibitor  model for primordia formation has had to be abandoned in its details.  One modelling problem with any reaction-diffusion theory has been that auxin transport is not primarily diffusive, and is more importantly transported through the plant cell wall via PIN1-type proteins, in directions highly dependent on which faces of the cell wall PIN1 is localised. 
The Turing instability itself  is unlikely to be a helpful concept for primordium formation. Its defining mathematical ability to generate a macroscopic length scale through a spatially extended linear wave instability seems not to play a biological role in these phyllotactic problems. Instead the length scale is generated, on our current biological understanding, by the physical scale of the developing primordium, which is largely independent of neighbouring primordia, and certainly distant ones, and controlled by a mixture of hierarchical genetic signalling,  the hormonal and mechanical dynamics that maintain the primordium, and the need to fit against existing patterns. 

PDE models, or their spatially discrete analogues which represent individual cells, continue to be a core tool in understanding the role of the spatio-temporal patterning of auxin in phyllotaxis, following~\cite{reinhardtRegulationPhyllotaxisPolar2003,smithPlausibleModelPhyllotaxis2006,jonssonAuxindrivenPolarizedTransport2006}, and have been particularly useful in articulating different hypotheses about the role of polarised auxin flux  within cells.
 At present we do not have reliable enough quantitative measurements of auxin to validate the models at a cellular scale, and 
 even recent work~\cite{pennybackerPhyllotaxisProgressStory2015} validates the model through emergent node positions, just as we might with the stacked-coin model. However emerging molecular data at cellular and subcellular scales is likely to change this in the foreseeable future.
 
Some other researchers continue to claim simple continuum reaction diffusion models as plausible source of phyllotactic pattern \textit{de novo}. Carteni et al~\autocite{carteniModellingDevelopmentArrangement2014} published a model suggesting that the patterning of vascular bundles could arise in this way, rather than as a sequence of patterns of increasing complexity. It is far from clear that such simple reaction-diffusion models can overcome the frequent problem of the strong dependence of emergent pattern on arena geometry. 


\section{Mechanical tension models}
Mechanical stress on the rigid plant cell well is known to reorient the microtubule orientation within the cell, and this has been suggested as a way in which PIN1 orientation, and thus auxin flux, can be affected by mechanical stress~\autocite{heislerAlignmentPIN1Polarity2010}. Galvan~\autocite{galvan-ampudiaPhyllotaxisPatternsOrganogenesis2016} is a recent review that includes this alternative tradition of modelling node placement as resulting from mechanical buckling, associated with Green~\cite{greenPhyllotacticPatternsBiophysical1996};  recently models which include both biochemical and mechanical elements have been developed by a variety of different groups~\cite{newellFibonacciPatternsCommon2013,galvan-ampudiaTemporalIntegrationAuxin2020,bull-herenuMechanicalForcesFloral2022}.  

Indeed it is a significant advantage of a PDE-type formalism that it can be combined with a similar continuous stress-strain term using  well-established formalisms. However when Pennybacker and Newell took advantage of this in their model, they still found they could derive all the key pattern behaviours they sought for Fibonacci structure by setting the strain dynamics to a constant~\autocite{pennybackerPhyllotaxisProgressStory2015}. But while the interaction between mechanical stress and molecular dynamics for phyllotaxis remains controversial, this class of models is likely to remain important.


\section{A divergence: protractor models}
All of the models above are either consequences of embracing the Standard Picture or at least not inconsistent with it. 
An alternative  explanation for Fibonacci phyllotaxis which has been greatly loved by mathematicians is that patterns are created by
repeated rotation by the Golden angle and that this has evolved because of an fitness arising from the resulting `close' packing. The earliest appearance of this idea was Wright's 1859 suggestion that a uniform distribution of leaves would maximise sunlight capture~\autocite{wrightMostThoroughUniform1859}, but despite being dismissed by high profile writers such as D'Arcy Thompson~\cite{thompsonGrowthForm1917} it still persists; one goal of this book is to argue that these arguments really should be retired. 

The most challenging modern voice to the Standard Picture is Okabe, who has used a series of peer-reviewed papers to argue that phyllotaxis shows `exquisite control' of the divergence angle~\autocite{okabeRiddlePhyllotaxisExquisite2016}. Okabe's papers are better sourced and more attentive to data than many proponents of the Standard Picture, but their reasoning is often hard to follow. Their prime claim seems to be that the Fibonacci angle is so finely on display in mature plants that it must have adaptive significance and have been arrived at by natural selection. 

The circumstances under which an asymptotic approach to a uniform distribution over divergence angles creates a significant fitness advantage have never been demonstrated and it seems implausible they ever could be, given the sophisticated alternative shade avoidance mechanism available to plants. Close packing has also been claimed to prevent ingress of pathogens, but this is more easily explained as a consequence of organs filling the Voronoi cells of their primordium patterns. As for maximising either uniform- or close-packing of the seeds of the sunflower, this could have some fitness, but it has never been identified. The sunflower has been subject to several thousands of generations, not of natural selection, but directed evolution by humans who have used it as a food source in the Americas. There must have been direct pressure to develop species with a large capitulum, but it is fanciful that any other morphogenetic factor played a role. 
But most fundamentally, the known biology offers no evidence for any molecular protractor in the plant delivering repeated divergence angles close to the golden angle and independently of previous primordia. Most current theorists would agree that precision in repeated instances of the divergence angle is, instead, an emergent property of pattern formation. 

Okabe does makes one observation that is useful to highlight here. The theorem that Jean christened the `Fundamental' Theorem of Phyllotaxis says that a $(1,2)$ system can take any divergence between 128.5$^\circ$ and 180$^\circ$, while experimental evidence is that divergence angles are in a much narrower range around the Fibonacci angle than this. Okabe's claim is perhaps that this suggests the divergence is a more fundamental property than the parastichy\index{parastichy}{} count, but my perspective is that this observation merely underlines how non-Fundamental the Theorem actually is for Phyllotaxis: it classifies lattices, but does not describe how they have evolved from other patterns, not even nearby lattices. 

\section{Summary}
We have reviewed a range of models for node placement. Many of these models --- with the notable exception of protractor models --- are likely to continue to have a role in helping to understand the biological mechanisms of node formation. However we have not yet reviewed one of the most promising contemporary models for node-formation, the stacked-disk model. We turn to this in the next chapter.

