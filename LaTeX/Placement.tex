
 
\chapter{Placement models}
\label{ch:placement}
Up to now we have studied only static lattices, which fix the location of every node simultaneously, and then we smoothly transformed the lattice as a whole. This approach cannot represent the one-by-one developmental choices for node commitment, so now we will consider more general node-placement models. This Chapter reviews a range of models that have usefully been deployed to understand Fibonacci phyllotaxis in this way. A modern book-length survey of such models with a rather different perspective is ~\autocite{barabePhyllotacticPatternsMultidisciplinary2020}. Recent reviews mainly devoted to the biological issues include~\cite{traasPhyllotaxis2013} and~\cite{godinPhyllotaxisGeometricCanalization2020}. One survey of models in plant morphogenesis more broadly is  ~\autocite{prusinkiewiczComputationalModelsPlant2012}.

Although we no longer require all node positions to form a lattice, we retain the cylindrical coordinates $(x,z)$ of section~\ref{sec:latticedef}, and suppose that the position of all the points $\jpoint{i} = (x_i,z_i)$ is fixed up to the $(k-1)$-th, with $z_i$ non-decreasing. The node-placement model of Chapter~\ref{sec:bio} is now expressible as a function whose output is $\jpoint{k}$ and whose input is the $\jpoint{i}$s, $i=0,\ldots, k-1$,
although in practice the output will only depend on recent or nearby node-positions. The time at which the new node is placed might be an output of the model (as in Snows' model) but if we follow Hofmeister's model and choose to hold constant the inter-node time, or \textit{plastochrone},  then we have a discrete-time model. 


\section{The Douady-Couder electrostatic model}
\label{sec:DouadyCouderModel}

A seminal physical model for phyllotaxis was Douady and Couder's 1992 electrostatic system, in which charged drops of oil are dropped onto a cylindrical plate in a magnetic field that causes a centrifugal motion. A caricature of this model is shown in Figure~\ref{fig:Ch8DCApparatus}.
\vnmafig{Ch8DCApparatus}{Simplified view of Douady and Couder's apparatus for generating phyllotactic patterns. A centrifugal electromagnetic gradient transports ferrofluid dipoles periodically dropped from the central pipette. Redrawn from~\cite{douadyPhyllotaxisDynamicalSelf1996a}.}{1}
If there is a very long time between successive drops, the drops will move off at random directions; if the process is speeded up slightly then mutual repulsion of the drops 
will ensure they move in successively opposite directions  as in the spots 0, 1 and 2 in part (a) of Figure~\ref{fig:Douady1996Fig3crop}. However if the process is sped up still faster, then the same mutual repulsion squeezes out the new spot to the side.  It is this symmetry breaking that has placed the spot numbered 2 away from the line between the two earlier spots in part (b) of Figure~\ref{fig:Douady1996Fig3crop} and initiated a spiral form.
\clearpage
\clearedjpgfig{Douady1996Fig3crop}{
		Different patterns of ferrofluid drops in Douady and Couder's experiment at different values of the parameter $G$ which control the rate of centrifugal movement. (a) When the time between drops is large compared to the time to be removed from the plate, each new drop is repelled only by the previous one and the node position  alternates by 180$^\circ$ each time: in the language of this book a distichous pattern $(1,1)$ is obtained with a divergence of $d=\jhalf$. (b) As the removal time decreases, a symmetry breaking bifurcation  to a  $(1,2)$ pattern appears with a divergence $d\approx 5/12$. (c) As the control parameter is further decreased higher order Fibonacci modes, here $(5,8)$, appear with divergence $d\approx{1}/{\tau^2}$.  \copyright Academic Press, 1996; from~\cite{douadyPhyllotaxisDynamicalSelf1996a}.}{0.25}{From ~\cite{douadyPhyllotaxisDynamicalSelf1996a}}

\clearpage

\vnmafig{Ch8DouadyFig4}{		
	Simulation results for a model of the Douady-Couder system,
	showing how only the Fibonacci branch is reachable with smooth changes of the control parameter $G$ down from 1.  The graph shows  $G$, corresponding to relative magnitude of time to clear the plate to inter-drop period, against emergent divergence angle $\phi= 2\pi d$.
	For each $G$, and for each of six divergence angles between $60^\circ$ and $180^\circ$, a spiral pattern was initiated with that angle and then run for 200 further iterations, and all spiral (i.e. periodicity 1) angles and higher period solutions recorded. 
	Recalculated from the model in~\cite{douadyPhyllotaxisDynamicalSelf1996a}.
}{.7}

A model of this spiral generation-phenomenon  can be built by assuming that the $k$\nobreakdash-th drop, arriving periodically at the centre of the disk at time $t_k= k T$, has its angle determined as soon as it has left a small inner ring of radius $z_c$, and then moves off with displacement $z(t)$ so has polar co-ordinates $\jpoint{k}=(z_c+z(t-t_k), \theta_k)$ with  $z(0)=0$, or the cylindrical co-ordinates above $(x=2\pi \theta_k,z(t))$. Douady and Couder modelled the radial displacement typically as exponential growth $z(t)=z_c\exp (Gt/T)$. 
 The choice of angle $\theta_n$ for the $n$-th and newest drop is made by minimising $\sum_{i<n} E(|(\jpoint{i}, (z_c,\theta_n) )|)$, where $|\cdot,\cdot|$ is the distance  between two points on the surface of the cylinder and $E$ is an energy function of the form say $E(r)=1/r^3$.  For these and similar functional forms, the system is controlled by the dimensionless parameter $G$. 
  
  Figure~\ref{fig:Ch8DouadyFig4} shows some results of simulation of this model, with a clear path through several Fibonacci transitions.  When versions of these  results were first published in the 1990s, they formed the first demonstration of a mechanism for Fibonacci phyllotaxis realised in a model which could be plausibly connected to known biology. It was a brilliant proof-of-concept for the possibilities of modelling in biology. 
  
  In the quarter-century since, though, the model itself has not been  re-used in plant developmental biology. One reason for this is the 
  energy function $E(\cdot)$. While this is a mathematically natural formalism, it cannot easily be connected to any measurable phyllotactic process. In addition, while the model constraint that only one drop can be dropped at a time is crucial to its success, it is a limitation as a phyllotactic model because it forbids patterns in which two nodes are formed near simultaneously at the same height on the stem, although there is ample evidence that this happens biologically. 
  
 

% However over the intervening years, the model itself has not (and presumably was never intended to be) become adopted within wider developmental biology. 



\clearpage

\section{The stacked-coin model}

One effect of the success of the Doaudy-Couder approach was to reanimate research into an older family of models based on coin-stacking.%

Coin-stacking is my invented name for these models but they have been widely used under various names. In the 1870s, Schwendener published figures of similar models which may well have influenced the work of van Iterson. In the early 1990s van der Linden demonstrated the first simulations of such models showing large Fibonacci numbers~\cite{vanderlindenCreatingPhyllotaxisDislodgement1990}, and particularly since the 2000s there have been a number of  useful papers I draw on in this chapter~\autocite{atelaRhombicTilingsPrimordia2017,hottonPossibleActualPhyllotaxis2006,atelaGeometricDynamicEssence2011}. 

In these models,  non-overlapping `coins' of diameter $D$ are successively placed at the lowest points possible on the cylinder,  touching two or three adjacent coins for their support. They then define node placements by their centre. The size of the $k$-th coin, $D(k)$, might be fixed or might vary with $k$, typically through the height of the stack so far.  In general we will be interested in $D(k)$ as a decreasing function of $z$, while the cylinder on which the coins are placed continues to have circumference 1. Figure~\ref{fig:SchwendenerCollectedp202} shows an example of such a model from the 1870s~\autocite{schwendenerMechanischeTheorieBlattstellungen1878}.%
\jpgfig{SchwendenerCollectedp202}{Schwendener's stacked-coin model showing transitions from a $\pp{3,5}$ to a $\pp{5,8}$ to a $\pp{8,13}$ parastichy~\autocite{schwendenerGesammelteBotanischeMittheilungen1898}. }{.3}

\clearpage 
These disks represent the inhibition zones in the SAM that are established by patterning of auxins and other morphogens, and the `gravity constraint' of adding the new disk at the lowest possible place corresponds to Snow's empirical rule that a new node will form as soon as it has enough space. The changing disk size corresponds to the change in proportion between the sizes of the inhibition zone and that of the SAM cylinder. It seems likely that most of the this change in relative geometry in actual plant growth is due to expansion of the SAM rather than modification of the inhibition zone size. So in some ways a more accurate visual representation than stacking variable-sized coins on a cylinder is to stack fixed-size coins on a cone. Figure~\ref{fig:2pstack} gives an example of such a model. The two model classes are not quite identical because the map between cone and cylinder will distort circles, but to the extent that the results of model simulations is not strongly dependent on the shape of the inhibitor zone we can expect similar results between the two.
  \jpgfig{2pstack}{The `stacked-coin' model, implemented with British tuppeny pieces, inspired by Atela~\autocite{atelaGeometricDynamicEssence2011}.}{0.5}

\clearpage

\vnmafig{Ch8InhibitionBoundary}{Finding the next location in the stacked-coin model by finding the lowest point of the boundary of the inhibition zone, itself defined by the topmost coins. The width of the inhibition zone may be variable and defines the radius of the next coin.}{1}
But when we do not have a precise lattice we need a more general way to compute parastichy numbers for our patterns.

Stacked-coin models are intuitively simple to solve computationally, as illustrated in Figure~\ref{fig:Ch8InhibitionBoundary}. We can 
 think of the existing coins creating an inhibition field, shown in grey in the Figure: no new coin can be placed below the boundary of the region formed by disks centred on all the existing coins, but with twice their radius. The new coin is placed at the minimum of this boundary.  Attractively, this placement depends only on the most recent `front' of coins around the cylinder, and is the minimum of a finite number of intersections of disks.
 
 A version of this model in which the fronts were curves called `pseudoconchoids'  was  arrived at by Schoute at the first quarter of the 20th century~\cite{schouteUberPseudokonchoiden1913}. According to  Richards' account~\cite{richardsGeometryPhyllotaxisIts1948}  of this German-language work, a pseudoconchoid allowed phyllotactic transitions by giving the front shape 'sufficient flexibility'. 
  
 The series of stacking positions can be thought of as the trajectory of a map on the space of (recent) positions, and locally stable trajectories correspond to observed outcomes of the model. The map's output can be written as the vector from the $k$-th to the $k+1$-th-node; if this map has a fixed point then we will have a lattice in the sense of Chapter~\ref{ch:cylinder}. 
%
\clearpage
\subsection{Parastichy numbers in stacked-coin models}
\vnmafig{Ch8ChainExample}{Computing the parastichy counts for non-lattice patterns by counting up and down links in chains around the cylinder. If the patterns is close to a lattice, these counts coincide with the parastichy numbers of the lattice.}{.5}
As is evident from Figure~\ref{fig:Ch8NewTransitionFibonacci}, stacked-coin models frequently generate patterns which are not parts of lattices, so we can no longer use the  definitions of Chapter~\ref{ch:cylinder}
to find a global pair of parastichy numbers for the pattern, because  parastichies are not exactly straight lines.  However, they are not so far off, and there is still a way to compute parastichy pairs locally~\autocite{goleFibonacciQuasisymmetricPhyllotaxis2016}.
The process is shown in Figure~\ref{fig:Ch8ChainExample}. Starting from one coin, we look for a shortest chain of touching coins that encircle the cylinder and returns to the original. That chain will have a number of up-steps to higher coins and down-steps to lower coins and the total number of up-steps and down-steps is exactly what we define as the parastichy count. 
If the pattern is exactly a lattice each such chain will give the same parastichy count. If not, then in regions of transition it is possible for there to be multiple counts associated with one coin, which is in itself a useful diagnostic of lack of regularity.

%
\clearpage
\subsection{Fibonacci transitions}
%
\vnmafig{Ch8NewTransitionFibonacci}{A stacked-coin model run, showing parastichy count transitions from a distichous pattern through Fibonacci numbers when the coin radius changes slowly enough. Each coin is added in turn with radius $D(z)$ where $z$ is the highest coin centre previously added, and $D(z)$ is decreased in a piecewise linear way from 1 to 1/6.}{1}
%
If we take one of the opposed touching-circle lattices with disk diameter $D$ of the previous chapter, truncate all the coins above one  point, and then use that as an initial condition for a stacked-coin model with fixed disk diameter $D$ we will regenerate the lattice. But a non-opposed lattices cannot in general be generated in this way because 
 whenever a new coin is added, the contact lines through its two supports must be in opposite directions on the cylinder. So we should not expect to see non-opposed lattices generated by stacked-coin models.
 
Putting this avoidance of opposed lattices together with Chapter~\ref{ch:classifying}'s Hypothesis of Geometrical Phyllotaxis would seem to go a long way towards explaining Fibonacci phyllotaxis in these models, and unsurprisingly it is possible to persuade them to exhibit the desired behaviour, as in Figure~\ref{fig:Ch8NewTransitionFibonacci}.
% 


It is tempting to stop at this point: we have built a model of node-formation which is informed by and fairly consistent with the known molecular biology, and we have shown that as biologically relevant parameter is varied, the model outputs pass through a series of increasingly complex patterns, each transition preserving the Fibonacci property. More than that, this property is generic and does not rely on model fitting. Apart from ensuring that the change of disk radius is slow enough, we have had to specify no particular parameters to achieve this sequence of Fibonacci transitions

However, quite apart from the question of biological validation, the mathematical satisfaction should be tempered by the fact that it took several attempts to generate Figure~\ref{fig:Ch8NewTransitionFibonacci}: vary the disk radius too slowly, and the model takes a biologically infeasible number of iterations to make the necessary transitions; vary it too quickly and the Fibonacci structure rapidly breaks down. As we shall shortly see, There is more going on in the dynamics of these systems than transitions between lattices, and this more mathematically complex dynamics appears also to have at least some biological relevance.  So showing the   relevance of the van Iterson classification requires overcoming  a (to me) surprising mathematical obstacle: when is a lattice  adequate to describe stacked coin dynamics?

\clearpage

\subsection{Stacking fixed coins in fixed geometry}
\vnmafig{Ch8FixedRadius}{A stacked coin model with disk size fixed at that of a square $\pp{2,3}$ lattice, started from different  initial conditions with in which a single disk of the square lattice has been distorted in size. }{1}

 \textit{Does} a stacked-coin model reliably lead to lattices? Intriguingly, the answer is no. 
 Figure~\ref{fig:Ch8FixedRadius} shows the results of five simulations of a fixed-size stacked-coin model, of which the centre starts from an exact $\pp{2,3}$ lattice, and the neighbours start from a perturbation of that lattice.
On the left we can see our square $\pp{2,3}$ lattice with straight line parastichies. For the other initial conditions with the same disk radius the lines joining the disk centres are now longer exactly straight, yet would still be described by eye as having 2-parastichies and 3-parastichies, although not always in the same direction.

 So a stacked-coin model can lead to solutions which are close to lattices, and if we start with a disk radius corresponding to a $\pp{m,n}$ lattice and an initial condition close to one, we might  hope to remain with a $\pp{m,n}$ parastichy count in our generalised sense. We can also expect that, just as the new third parastichy number at a lattice bifurcation is either $\pp{n+m}$ or $\pp{n-m}$, the same will be true for patterns near to those lattices, and indeed Golé et al~\cite{goleFibonacciQuasisymmetricPhyllotaxis2016} give an elegant argument based on chain transitions for this. But how close to a lattice do these patterns need to be for  the  elegant bifurcation theory of Chapter~\ref{ch:classifying} still to hold? 


\clearpage
\subsection{Cylinder tilings and rhombic-tilings}
\label{sec:tilings}
\vnmafig{Ch8AreaTimeSeries}{Area of each polygon in the lattices of Figure~\ref{fig:Ch8FixedRadius}, read left to right, plotted from bottom to top.}{1}
These seem to be periodic graphs, corresponding to periodic orbits of the stacked-coin map. 

One way to quantify the way in which the runs of Figure~\ref{fig:Ch8FixedRadius} depart from lattice structure is to plot a function associated with each node, such as the area of the polygon below it,  as a function of the dropped-coin sequence. This is done in Figure~\ref{fig:Ch8AreaTimeSeries}.

Just as a lattice corresponds to a fixed point of the next-node vector stacked-coin map, with a constant $(d,h)$ between nodes, a cylinder tiling corresponds to a low-order periodic solution of the map, with a sequence $((d_1,h_1),\ldots (d_k,h_k) )$. The upper parts of the tilings in Figure~\ref{fig:Ch8FixedRadius} are close approximations to cylindrical tilings.
A consequence of the disk stacking model is that each node in Figure~\ref{fig:Ch8FixedRadius} is generically at the apex of a single polygon. (Exceptionally, if the pattern is a hexagonal lattice, there will be two such polygons.)
More than that, in this Figure at least, all of the polygons of the tiling being approached appear to be rhombi: four-sided polygons with opposite parallel sides.%
A cylindrical tiling in which the polygon attached to each point is one of a finite number of different rhombi is called a rhombic tiling.%


	Rhombic tilings were introduced in this context by~\cite{atelaRhombicTilingsPrimordia2017}.
	Tilings of the plane by low order polygons are well studied, especially after the surprising discovery of Penrose tilings: there seems no mathematical study of cylinder-periodic ones outside of phyllotaxis.
\clearpage
\vnmafig{Ch8Tilings}{Rhombic tilings arising from a stacked coin model.
	Tiles are coloured to emphasise a periodicity of 6 for each tiling after the first. }{1}

Figure~\ref{fig:Ch8Tilings} shows an example in which the stacked-coin map is visually close to having periodicity of 6. The map trajectory defines a series of polygons attached to the coin centre which is its lowest point. In this rhombic-tiling pattern each coin has exactly one rhombus attached to it.

 


%	\item Proof for simple case of 3 fixed-radius disks 

\clearpage
 Douady and Gol\'e gave an interesting example of a cylinder tiling and a lattice separately fitted to the same observation, which is shown in Figure~\ref{fig:Douadygole2016bFig4}.
\clearedpdffig{Douadygole2016bFig4}{Fit of a lattice and a cylinder tiling to the same unrolled birch catkin image. The cylinder tiling is a 14\% better fit to the node positions than the lattice. From~\autocite{douadyFibonacciQuasisymmetricPhyllotaxis2016}.
\copyright CC-BY Douady and Gol\'e 2016.}{1.0} {Figure 1 of ~\autocite{atelaRhombicTilingsPrimordia2017}} 
 Since a tiling made up of $n$ different rhombi has at least $2n$ degrees of freedom, it can naturally give an better fit to an observed pattern than a lattice, which has $n=1$, and it is an open question whether empirical observations can at present reject lattice patterns in the framework of statistical hypothesis testing.%
 
 It is striking that before these quasi-lattice patterns were seen in model outputs, there were few attempts to parameterise empirical data in any way other than as realisations of a lattice. One notable exception was Atela et al~\cite{atelaDynamicalSystemPlant2002} who suggested that a period 8 orbit they found in a placement map corresponded to a similar periodicity in divergences observed on a magnolia carpel~\cite{tuckerPhyllotaxisVascularOrganization1961}. 
 Beyond this example, shifting the unit of analysis to a cylindrical or rhombic tiling has generated an interesting new series of observational questions~\autocite{douadyFibonacciQuasisymmetricPhyllotaxis2016}.
 

 Rhombic-tilings are also important mathematically because, as Figure~\ref{fig:Ch8FixedRadius} illustrates, small perturbations to initial lattice patterns typically evolve into rhombic-tilings under fixed-size coin stacking.~\autocite{atelaRhombicTilingsPrimordia2017}.
   Golé and Douady have proved that every initial chain that has a parastichy count of $(1,2)$ will indeed evolve into a rhombic tiling in finite time or become exponentially close to one, and conjectured 
   this is true in general.~\autocite{goleConvergenceDiskStacking2020}.
 
% \subsection{Butterfly effects?}
  It has been suggested that stacked-coin models provide an explanation of a `butterfly effect' in which the phyllotaxis of, e.g., the magnolia stem sometimes changes `for no obvious reason'\autocite{zagorska-marekSignificanceGandLdislocations2016}. Numerical simulations can indeed sometimes show surprisingly long intervals of apparent equilibrium followed by a burst of changes, and even if these trajectories
  are only transient may be of value in exploring phyllotactic transitions in species like \textit{Magnolia}.
     
\section{Columnar models}
\label{sec:columns}

Exploration of the dynamics of stacked-coin models has revealed another intriguing pattern formation process. If instead of very slowly changing the coin radius we very quickly reduce it and then keep it small, we are effectively starting a fixed-radius run with random initial conditions and relatively small disks compared to the cylinder circumference. Figure~\ref{fig:Ch8TransitionLattices} gives an example.
 It turns out, for reasons explored in~\autocite{goleFibonacciQuasisymmetricPhyllotaxis2016}, that
 runs often converge to a series of near horizontal sets of disk stacked on top of each other. This stacking is fairly regular so that disks in either every row or every other row are vertically above each other. In the older language of phyllotaxis, these patterns might be called whorled orthostatic, and  preferable alternative. Douady and collaborators~\cite{goleConvergenceDiskStacking2020}, who have done unparalleled work on them,  call these QSS patterns, for quasi-symmetric solutions, but here I call them \textit{columnar} patterns.

This also provides a potential pattern mechanism for columnar patterns like those of the sweetcorn of Figure~\ref{fig:sweetcorn20191008} or of cacti. These models might well replace Turing Instability-based ones as a plausible pattern-formation mechanisms for such patterns. 

\vnmafig{Ch8TransitionLattices}{Runs of a stacked-coin model for increasingly steep radius changes. In each case, as in Figure~\ref{fig:Ch8NewTransitionFibonacci} the disk radius is reduced from one in which a lattice can, according to the van Iterson diagram, have a \gp{2,3} parastichy through a \gp{3,5} parastichy to a \gp{5,8} parastichy and then fixed. When the transition is slow enough for the \gp{3,5} lattice to establish well before before moving into the zone where a \gp{5,8} lattice is possible, that transition is visible  is followed by an equally ordered transition to what appears to be a rhombic tiling close to a \gp{5,8} lattice. 
	 When the disk radius is reduced more rapidly, the ordered transitions are lost. In the central run, the parastichy count at the top of the cylinder are $\gp{7,8}$ while in the right hand run with a still faster reduction of radius the parastichy count at the top is $(7,7)$. 
	}{.8}
	\clearpage
\vnmafig{Ch8TransitionParastichy}{Parastichy numbers as a function of height up the cylinder for each of the three runs in Figure~\ref{fig:Ch8TransitionLattices}. 
}{1}




It is clear numerically that the elegance of pattern analysis as transitions through near-to-touching-circle lattices is lost if the geometric change is too fast, and this can be interpreted as being because the close-packing property is lost: the constraints of a lattice pattern cannot adapt quickly enough to stay close to being well-packed,
and lattices as the units of analysis become unhelpful. 
But the extra degrees of freedom of rhombic tilings might well allow them to adapt more efficiently, in terms of close-packing, to geometric change than lattices do. So it might be that Fibonacci transitions through rhombic-tiling space can be maintained at higher rates of geometric change than if lattice structure were enforced. Arguments have already been made that rhombic tilings have their own more general version of the Hypothesis of Geometrical Phyllotaxis, and so taking the rhombic tiling as the unit of analysis rather than the lattice might provide 
a more biologically robust explanation of Fibonacci phyllotaxis than one based solely on lattices. 
\clearpage
\subsection{Summary of stacked-coin models}
Stacked-coin models provide an attractive generalisation of lattice models for exploring Fibonacci phyllotaxis, although their mathematical properties are much well less worked out, even for fixed disk size.  For example,  it is unclear if every attractor of such models is a cylinder tiling, or under what circumstances
 such models can yield disk packings which are denser than those of the corresponding lattice. Nevertheless variable-size  stacked-coin models at present provide a good candidate for being able to make precise the idea of the `smoothly changing lattice' in a biologically important way and they are the subject of active research.
 
\section{Energy based models}
\label{sec:energy}
The Douady-Couder model of section~\ref{sec:DouadyCouderModel} imposed a time delay between new nodes and a centrifugal movement, which together ensured that new nodes could not be placed too closely to old ones. It then defined an energy function for the new pattern, so that minimising the energy step-by-step then placed nodes as close as possible given the first constraint.  By contrast the stacked-coin model uses a fixed disk to keep the nodes separate, whilst using 'gravity', that is the requirement that the new node is in contact with the old ones, to keep them close. Both of these models yield  a typical inter-node distance of say $D$, and can be thought of as defining a  node-node potential energy which is large before $D$, then drops off sharply to a minimum near $D$ before more smoothly re-increasing. 

Given such a potential energy representing a mixture of close-range repulsion and long-range attraction, it is possible to define an energy for cylindrical node patterns, and then to analyse the patterns which are local minima of this energy. Such an energy is of the form 
$E=\sum U(\left|\jpoint{n}-\jpoint{m}\right|)$ as a sum of distances over node-pairs.  {A key difference between our application and most physical ones is that there is no mechanism for \textit{global} minimisation of the energy: a pattern may be built up node-by-node, locally minimising this interaction energy, but in general the final pattern is not a global minimum of all possible patterns on the same cylinder with the same mean $D$.}
Nevertheless it was these formal similarities with theoretical physics which led Levitov and others to the renormalisation technique~\autocite{levitovFibonacciNumbersBotany1991}. If we restrict to patterns which are lattices, then the energy of a lattice can be analysed by renormalising the lattice, which simply scales all the node-pair distances by the same scalar. This analysis thus rediscovers the  lattice bifurcation structure of the van Iterson diagram.

 Lee and Levitov~\autocite{leeUniversalityPhyllotaxisMechanical1998} criticised, I think unfairly, even early node placement models such as those of Adler~\autocite{adlerConsequencesContactPressure1977} as not corresponding directly to any biological system. By contrast,
 Lee and Levitov claimed a deep universality for energy-based models as a justification for pruning the van Iterson tree to Fibonacci structure. However the energy function is defined in terms of 'distance', which has an intuitively reasonable interpretation in many physical systems but is very hard to interpret here. It is an open question, though, whether the flexibility of the energy-based formalism to include relatively soft disks and thus allow local rearrangements might remove some of the difficulties that we have seen for hard disk models in reaching local lattice solutions under slow changes. 
\newpage
\section{Partial differential equation models}
\subsection{An early reaction-diffusion approach}
 In 1951 Alan Turing wrote to a prominent British zoologist of the time that 
\begin{quotation}
	\dots my mathematical theory of embryology\dots is yielding to 
	treatment, and it will so far as I can see, give satisfactory 
	explanations of\\
	(i) gastrulation\\
	(ii) polygonally symmetrical structures, e.g. starfish, flowers\\
	(iii) leaf arrangements, in particular the way the Fibonacci \\
	series ($0,1,1,2,3,5,8,13,\dots$) comes to be involved\\
	(iv) colour patterns on some animals, e.g. stripes, spots and \\
	dappling\\
	(v) pattern on nearly spherical structures such as some Radiolara \dots\jNote{Letter from AM Turing to JZ Young 8 Feb 1951, King's College Cambridge AMT K.1.78.}
\end{quotation}
By `my theory of embryology' Turing meant primarily what we now call the Turing Instability, a  pattern formation mechanism for length-scale emergence in a reaction-diffusion PDE system. 
Turing's mechanism, which he later published in his celebrated 1952 paper~\autocite{turingChemicalBasisMorphogenesis1952} has been a mathematical blessing in its generality and a biological curse because of the same generality. With a few exceptions, biologists over the last 70 years have viewed the Turing instability as elegant but unhelpful or wrong for understanding the major questions in developmental biology. 	 Subsequent mathematical biology using the Turing Instability has been particularly rich in (iv), is probably the wrong approach for (i) and (ii), and is largely forgotten for  (v). But we are now in a position to see what Turing meant when he though it could help with (iii). 
The previous models of this Chapter abstracted known, or guessed, details of the molecular biology of node formation, into various representations of pattern formation on a cylinder with a typical length scale of $D$. It was the availability of such a model, in the form of the Turing Instability,  that allowed Turing to develop \textit{The Outline of the Development of the Daisy}. This was an unpublished manuscript in which he used a PDE model to explore transitions through lattice space and to try to find conditions for  the Hypothesis of Geometrical Phyllotaxis to be fulfilled.


\subsection{Auxin flow models}

As molecular details have accumulated, Turing's essentially arbitrary activator-inhibitor  model for primordia formation has been abandoned in its details.  One modelling problem with any reaction-diffusion theory has been that auxin transport is not primarily diffusive, and is more importantly transported through the plant cell wall via PIN1-type proteins, in directions highly dependent on which faces of the cell wall PIN1 is localised. 
The Turing instability itself  is unlikely to be a helpful concept for primordium formation. Its defining mathematical ability to generate a macroscopic length scale through a spatially extended linear wave instability seems not to play a biological role in these phyllotactic problems. Instead the length scale is generated, on our current biological understanding, by the physical scale of the developing primordium, which is largely independent of neighbouring primordia, and certainly distant ones, and controlled by a mixture of hierarchical genetic signalling,  the hormonal and mechanical dynamics that maintain the primordium, and the need to fit against existing patterns. 

PDE models, or their spatially discrete analogues which represent individual cells, continue to be a core tool in understanding the role of the spatio-temporal patterning of auxin in phyllotaxis, following~\cite{reinhardtRegulationPhyllotaxisPolar2003,smithPlausibleModelPhyllotaxis2006,jonssonAuxindrivenPolarizedTransport2006}, and have been particularly useful in articulating different hypotheses about the role of polarised auxin flux  within cells.
 At present we do not have reliable enough quantitative measurements of auxin to validate the models at a cellular scale, and 
 even recent work~\cite{pennybackerPhyllotaxisProgressStory2015} validates the model through emergent node positions, just as we might with the stacked-coin model. However emerging molecular data at cellular and subcellular scales is likely to change this in the foreseeable future.
 
Some other researchers continue to claim simple continuum reaction diffusion models as plausible source of phyllotactic pattern \textit{de novo}. Carteni et al~\autocite{carteniModellingDevelopmentArrangement2014} published a model suggesting that the patterning of vascular bundles could arise in this way, rather than as a sequence of patterns of increasing complexity. It is far from clear that such simple reaction-diffusion models can overcome the frequent problem of the strong dependence of emergent pattern on arena geometry. 


\section{Mechanical tension models}
Mechanical stress on the rigid plant cell well is known to reorient the microtubule orientation within the cell, and this has been suggested as a way in which PIN1 orientation, and thus auxin flux, can be affected by mechanical stress~\autocite{heislerAlignmentPIN1Polarity2010}. Galvan~\autocite{galvan-ampudiaPhyllotaxisPatternsOrganogenesis2016} is a recent review that includes this alternative tradition of modelling node placement as resulting from mechanical buckling, associated with Green~\cite{greenPhyllotacticPatternsBiophysical1996};  recently models which include both biochemical and mechanical elements have been developed by groups around Newell~\cite{newellFibonacciPatternsCommon2013} and Godin~\cite{galvan-ampudiaTemporalIntegrationAuxin2020}.  

Indeed it is a significant advantage of a PDE-type formalism that it can be combined with a similar continuous stress-strain term using  well-established formalisms. However when Pennybacker and Newell took advantage of this in their model, they still found they could derive all the key pattern behaviours they sought for Fibonacci structure by setting the strain dynamics to a constant~\autocite{pennybackerPhyllotaxisProgressStory2015}. But while the interaction between mechanical stress and molecular dynamics for phyllotaxis remains controversial, this class of models is likely to remain important.


\section{A divergence: protractor models}
All of the models above are either consequences of embracing the Standard Picture or at least not inconsistent with it. 
An alternative  explanation for Fibonacci phyllotaxis which has been greatly loved by mathematicians is that patterns are created by
repeated rotation by the Golden angle and that this has evolved because of an fitness arising from the resulting close packing. The earliest appearance of this idea was Wright's 1859 suggestion that a uniform distribution of leaves would maximise sunlight capture~\autocite{wrightMostThoroughUniform1859}, but despite being dismissed by high profile writers such as D'Arcy Thompson~\cite{thompsonGrowthForm1917} it still persists.

The most challenging modern voice to the Standard Picture is Okabe, who has used a series of peer-reviewed papers to argue that phyllotaxis shows `exquisite control' of the divergence angle~\autocite{okabeRiddlePhyllotaxisExquisite2016}. Okabe's papers are better sourced and more attentive to data than many proponents of the Standard Picture, but their reasoning is often hard to follow. His prime claim seems to be that the Fibonacci angle is so finely on display in mature plants that it must have adaptive significance and have been arrived at by natural selection.  Okabe reads to me as though he conceives of plants possessing a molecular protractor which they calibrate to the Fibonacci angle for maximum fitness. 

The circumstances under which an asymptotic approach to a uniform distribution over divergence angles creates a significant fitness advantage have never been demonstrated and it seems implausible they ever could be, given the sophisticated alternative shade avoidance mechanism available to plants. Close packing has also been claimed to prevent ingress of pathogens, but this is more easily explained as a consequence of organs filling the Voronoi cells of their primordium patterns. As for close-packing the seeds of the sunflower, this could have some fitness, but it has never been identified. The sunflower has been subject to several thousands of generations, not of natural selection, but directed evolution by humans who have used it as a food source in the Americas. There must have been direct pressure to develop species with a large capitulum, but it is fanciful that any other morphogenetic factor played a role. 
But most fundamentally, the known biology offers no evidence for any molecular protractor in the plant delivering repeated divergence angles close to the golden angle and independently of previous primordia. Most current theorists would agree that precision in repeated instances of the divergence angle is, instead, an emergent property of pattern formation. 

Okabe does makes one observation that is useful to highlight here. The so-called Fundamental Theorem of Phyllotaxis says that a $(1,2)$ system can take any divergence between 128.5$^\circ$ and 180$^\circ$, while experimental evidence is that divergence angles are in a much narrower range around the Fibonacci angle than this. Okabe's claim is perhaps that this suggests the divergence is a more fundamental property than the parastichy count, but my perspective is that this observation merely underlines how non-Fundamental the Theorem actually is for Phyllotaxis: it classifies lattices, but does not describe how they have evolved from other patterns, not even nearby lattices.

