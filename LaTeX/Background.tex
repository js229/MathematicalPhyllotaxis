
\chapter{Fibonacci numbers, the golden ratio, and co-prime integers}
\SpringerAbstract{The Fibonacci sequence $F_n$ has as its first two members  $F_0=0$, $F_1=1$ and every subsequent member is the sum of the previous two:  $F_{n+2}=F_{n+1}+F_{n}$. Although there is a substantial literature on Fibonacci and related sequences we really only need this simple sum property and some elementary related facts collected here.}

\label{CH:0}
\label{ch:0}
\section{Fibonacci sequences}

The Fibonacci sequence $F_n$ has as its first two members  $F_0=0$, $F_1=1$ and every subsequent member is the sum of the previous two:  $F_{n+2}=F_{n+1}+F_{n}$. 
A related sequence is the Lucas\footnote{Named after Édouard Lucas, a French mathematician~\cite{ballotLucasSequencesTheory2023,decaillotEdouardLucas184218911999}, and so best pronounced \textit{Luke--aah} by English speakers.} sequence  ($1,3,4,7,11,\ldots$) with the same rule but different initial conditions. Both Fibonacci and Lucas numbers are special cases of the generalised Fibonacci numbers with starting pair $F^k_1=1$, $F^k_2=k$. We will also encounter the sequence created by doubling each Fibonacci term, for which terminology varies but I'll call a double-Fibonacci sequence. 
%
\begin{table}[ht]
	\caption{Various sequences with Fibonacci structure}
	\label{tab:sequences}
	\begin{center}
		\begin{tabular}{ll}
			\hline
			Fibonacci  &  $1,1,2,3,5,8,13,21,34,55,89,144,\ldots$ 
			\\
			Double Fibonacci & $2,2,4,6,10,16,26,42,68,110,\ldots$
			\\
			Lucas ($F^3$)&      $ 1,3,4,7,11,18,29,47,76,123,199\ldots$
			\\
			$F^4$  & $1,4,5,9,14,23,37,60,97,157\ldots$
			\\
			$F^5$ & $1,5,6,11,17,28,45,73,118,191\ldots$
			\\
			\ldots &
			\\
			$F^8$ & $1,8,9,17,26,43,69,112,181\ldots$
			\\
			\hline
		\end{tabular}
	
	\end{center}
\end{table}
%

We might note that from Table~\ref{tab:sequences} that the Fibonacci, double Fibonacci and Lucas sequences together include all  of the first eleven integers except 9, so there is little remarkable about the observation that a particular system exhibits a structure including a low member of one of the sequences~\cite{cookeFibonacciNumbersReveal2006}. Finding \textit{pairs} of adjacent members of the same series, or examples of numbers greater than ten, might be more significant.

\subsection{The golden ratio}
Starting the sequence with $F_0=0$, $F_1=1$, the general Fibonacci term is 
\begin{eqnarray}
F_n &=& \frac{\tau^n - (1-\tau)^n}{\sqrt{5}}
\end{eqnarray}
where $\tau$ is the golden ratio 
satisfying
\begin{eqnarray}
\tau^2 &=& \tau+1 \label{eq:tauQuadratic}
\\
\tau &=& \frac{1+\sqrt{5}}{2} \approx 1.618
\\
&=& \lim_{n\rightarrow\infty} \frac{F_{n+1}}{F_n} 
\end{eqnarray}
The sequences in Table~\ref{tab:sequences}, and indeed any sequence obeying the Fibonacci rule, all have $\tau$  as the limit of the ratio of terms. These results can be found by trying solutions of the form $F_n=k\tau^n$ and seeing that $\tau$ must be one of the two solutions of~\eqref{eq:tauQuadratic}. This book only needs the facts in this chapter, but there is vast literature on Fibonacci and related sequences of which Vajda~\cite{vajdaFibonacciLucasNumbers2008}, say, is an  accessible example. 

\subsection{The golden angle}
The golden angle is
\[
\Phi = \frac{2\pi}{\tau^2}  \approx 137^\circ
\]
so
\[
\frac{\Phi}{2 \pi} \approx \frac{F_{n}}{F_{n+2}}
\]
As we'll see, the angular rotation between successive leaf structures is often very close to this angle for Fibonacci (but not, for example, Lucas) phyllotaxis.




\section{Co-prime integers}

\label{sec:coprime}
Two integers $(m,n)$ are co-prime iff%
\jNote{The word `iff' is not a typo. It means if and only if, and is also a shibboleth.
	If it is bewildering to you that `iff' means logical equivalence and not just implication then you have not been inducted into the kind of mathematical exposition, roughly first-year undergraduate, we use until around section~\ref{sec:ClassifyingSummary}, so you will want to skip to there. On the other hand you can read all of the rest of the book without knowing what a shibboleth is.}
their greatest common divisor is equal to 1, so that for example 4 and 11 are co-prime. There are some fiddly edge cases:  $(0,1)$, $(1,1)$ and $(1,n)$ are co-prime for integer $n>1$, but $(0,0)$ and $(0,n)$ are not. These cases are consistent with the observation that  every integer is a divisor of $0$, but the only divisors of $1$ are $1$ and $-1$. S In the next chapter we will review the classic machinery for computing co-primaliy:  the B{\'e}zout relationship and the Euclidean algorithm. 
% 
