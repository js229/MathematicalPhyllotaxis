

\chapter{Transformed Lattices}
\label{ch:transformations}


\abstract{
This chapter collects the mathematical tools needed to transform lattices without preserving isometry.  The lines drawn on the sunflower, seen locally, join the nearest seed positions and they are by definition the shortest distances between those two points. Yet globally they are not straight lines but curved spirals. Conversely, if we take the lattices of previous chapters and project them onto two-dimensional surfaces which are neither planar nor cylindrical, then there will be metric distortions: we have to stretch the parastichy\index{parastichy lines!transformed} lines. As we saw by thinking about increasingly squashed lattices,  with fixed divergence $d$ but decreasing rise $h$, the definition and perception of the principal parastichy numbers depend crucially on the metric properties. The same lattice projected onto two different surfaces will typically have the same sets of generating pairs but the specific pair with the property of being the shortest can change depending on the projection. This is to some extent a mathematical artefact: biologically there is no initial constant-curvature surface of sunflower meristem cells upon which organ commitment actually happens before being rolled into the final shape. In practice, each such 'lattice point' is a developmental commitment made at a particular times and geometry of the surrounding tissue micro-environment. 
}


\section{Smoothly changing lattices}
The bifurcation theory approach of the previous chapter invites us to think about the paradoxical idea of the `smoothly changing lattice'. We saw, for example, that lattices with a golden angle divergence had parastichy numbers which were increasing pairs of adjacent Fibonacci numbers as the rise decreased. Figure~\ref{fig:Txb0602Transition} presents the same result is a different way, applying a non-uniform vertical stretch to a golden lattice.
While we can no longer define parastichy numbers using the lattice formalism, they are still easily identifiable visually;  Chapter~\ref{ch:placement} will give a  way to formalise this but for now this intuitive visual definition will suffice. 
\mmafig{Txb0602Transition}{ Lattice-like patterns change principal parastichy numbers under smooth geometric transformations. A regular golden lattice has its $z$-coordinates smoothly transformed so that the rises between successive points become smaller. From bottom to top, the parastichy pairs $(3,5)$, $(5,8)$ and $(8,13)$ are shown. Compare with Figure~\ref{fig:F0103ChurchEuphorbia}.
}{1.0}%
If the change of rise from point to point is too sharp, 
then we will not necessarily see Fibonacci numbers, even in a lattice stretched from one using the golden angle, but in Figure~\ref{fig:Txb0602Transition}, the transformation has been carefully chosen to show a change from  $(3,5)$ through $(5,8)$ to $(8,13)$, and even here there are some intermediate heights at which assigning a parastichy pair at all is not straightforward. 

Although phyllotactic patterns are undoubtedly deformed after creation, it is important to understand that, eg, large Fibonacci parastichy numbers are not formed in this way empirically. Nevertheless this approach is useful for thinking about the scale by which geometry must change for large Fibonacci numbers to be possible. Figure~\ref{fig:Txb0601ScalingBar}  shows how the parastichy numbers for a golden lattice depend on the geometry of the touching-circles making up the lattice: we see that a change of developmental scale of 100-250 fold is necessary to move between a $(0,1)$ and a $(144,233)$ lattice.

\mmafig{Txb0601ScalingBar}{
	The relationship between the stem circumference relative to the developmental length-scale and principal parastichy counts. If the touching-circle of a  golden lattice is fixed to have radius 1, then Chapter~\ref{ch:classifying} shows that the  circumference of a cylinder on which a hexagonal lattice has parastichy counts $m$ and $n$ is $\sqrt{m^2+mn+n^2}$, as illustrated here for Fibonacci pairs.
	}{1}%




\section{Lattices on disks}
A related stretching of lattices comes from putting them onto disks. 
Visualisations of phyllotaxis on a disk rely on looking down from above on either a microscopic horizontal slice across the SAM, or on a macroscopic view  onto the head of a sunflower, but no matter the scale a crucial point is that the oldest organs placed are on the \textit{outside} of the disk. 
So we would like to construct a map between the lattice on a rectangular cylinder of Figure~\ref{fig:Txb0403Coordinates}, with rectangular co-ordinates $(x,z)$ and one on a disk with co-ordinates $r,\theta$. If we work on a  range $0,z_{\text{max}}$ of the cylinder, and then we need a decreasing function of $z$, $r(z):[ 0,z_{\text{max}}]\rightarrow[ 1, r(z_{\text{max}})]$ to move to in lattice polar co-ordinates $(r(k h),2\pi d)$ on the unit disk. In particular $r'=dr/dz<0$. 

\mmafig{Txb0603Transforms}{Transformation of a cylindrical golden lattice onto a disk through (left) an exponential transform and (right) a quadratic one. On the left the exponential transform leaves the principal parastichies remain unchanged: the 55 and 89 spirals are the ones that go through adjacent cells both on the outside and the inside of the rim. On the right the quadratic transform gives transitions from $(55,89)$ down through $(21,34)$.
	%
	The exponential scaling is characterised by the parastichy\index{parastichy lines!as spirals}{} lines becoming log-spirals,
	but the quadratic scaling by the area of each point remaining approximately constant.  On the bottom row the Voronoi cells around each point are plotted.
	%
	On the left, the $z$-coordinate of the cylindrical lattice is transformed with $r=\exp(-\alpha z)$, on the right with $ r^2+ \gamma z^2=1$, with the constants $\alpha$, $\beta$, and $\gamma$ chosen so that the annulus between the first and second-outermost points has the same area and height:width ratio as the cylindrical lattice. For the exponential transform but not the quadratic one the aspect ratio remains the same throughout the disk.
}{1}
	
The difference in radial distance $r(kh)-r((k+1)h)$ is sometimes known as the plastochrone. For $h$ small relative to 1 this is approximately $-hr'(kh)$.  We know from previous chapters that even if the divergence angle of a lattice is fixed, the principal parastichy numbers for the lattice depend on the aspect ratio of the strip between the $k$th and $k+1$th lattice point on the cylinder. On the cylinder this aspect ratio remains constant and equal to $1::h$. But the corresponding annulus on the disk has circumference close to $2\pi r(kh)$ and thickness close to $-hr'(hk)$ and so the aspect ratio is $1::-hr'/2\pi r$. The area of the annulus, which is the same as the area per node, is $-2\pi h  r r'$. 
We can find useful functional forms for $r(z)$ by considering how these two quantities scale with $r$ when a cylindrical lattice is mapped onto the disk.
\subsection{Exponential scaling}
 If we control the aspect ratio by setting it as  $1:: h/\alpha(r)$ then
\begin{align}
\alpha(r) = -\frac{1}{2\pi}\frac{r'(z)}{r(z)}.
\end{align}
In particular, the aspect ratio of the strip is kept constant by the exponential scaling
\begin{align}
{r(z)} &= \exp(-2\pi\alpha z)
\end{align}
The exponential scaling has been a historically common choice for the radial scaling $r(z)$. It maps parastichy lines to logarithmic spirals, and because it preserves aspect ratios also preserves the principal parastichy numbers. 

The ratio of the radii between the $k$th and the $kth$ node is 
\begin{align}
R &= \frac{r((k+1) h)}{ r(kh)} 
\\
&= \exp (-2 \pi \alpha h) \label{eq:plastoratio}
\end{align}
and this is sometimes called the plastochrone ratio; assuming that this is fixed is equivalent to assuming the scaling \textit{is} exponential. 



\subsection{Quadratic scaling}
 If we supposed the area per seed was $2 \pi h \beta (r)$ then we get from above that $-2\pi h  r r'=2\pi h \beta (r)$ and if we further set $\beta$ as constant get
\begin{align}
	r(z) = \sqrt{ 1- 2\beta z } 
\end{align}
 



\subsection{Biological choices}
 Figure~\ref{fig:Txb0603Transforms} gives an example of the same cylindrical lattice transformed onto a disk with two different functions.
The historical attachment to exponential scaling has often, I suspect, been made for mathematical convenience rather than based on biological data, if it has consciously been made at all. Exponential scaling   preserves parastichy counts, while quadratic scaling preserves seed size and does display parastichy transitions. Since many sunflower heads do show elements of both, it might be tempting to use a functional form for $r(z)$ intermediate between exponential and quadratic scaling, together with a golden ratio lattice, to yield a model sunflower. 

\mmafig{Txb0604Sunflower}{A model sunflower head constructed from a golden lattice together with disk scaling function that allows seed size to become smaller at the centre of the disk, with  principal parastichies indicated}{1}
Figure~\ref{fig:Txb0604Sunflower} is the latest of many examples of doing this. One might even be tempted to try and infer the disk scaling of an observed pattern by fitting such a model. 
But in understanding the biological process fitting such a form would have little explanatory value. Biologically there is no Euclidean cylindrical lattice to map from in the first place, and even if there was quantifying a map from the geometry and dynamics of the developing plant to that of the observed outcome is not a prospect with much chance of success. Instead we will start to need to look at the dynamics of organ placement on the embryonic seed head.  

\begin{jExercise}\label{ex:richards}
 Consider a golden lattice, vertically stretched so that the cylinder segments into regions $C_k$ where pattern is close to a lattice with parastichy counts of adjacent Fibonacci pairs $(F_k,F_{k+1})$.
	Show that the average rise in segment $C_k$ scales like $\tau^{-2k}$.
	Find a relationship between $k$ and the plastochrone ratio $R$ of a disk pattern mapped from a near-Fibonacci lattice with an exponential scaling.
\end{jExercise}
\begin{jAnswer}{ex:richards}{
	The first result itself follows
	from reading the rise of a lattice from~\eqref{eq:gReImA}, combined with observing that Fibonacci numbers scale like $\tau^k$
	and so $h$ scales like $n^{-2}$. 
	
The exercise is to warn the reader against trying to make too much sense of the following strange expression which appeared in
	\autocite{richardsPhyllotaxisItsQuantitative1951}:
	\begin{align}
		k\approx \mbox{P.I} &= 0.33 - 2.39 \log_{10}  \log_{10} R.
		\label{eq:richards}
	\end{align}
	 Richards wanted to fit real disk patterns as idealisations of those on the left of Figure~\ref{fig:Txb0603Transforms}. He  
	 seems to have assumed an orthogonal golden lattice transformed with logarithmic spirals, and his unfortunate idea was to use the $R$ of equation~\eqref{eq:plastoratio} to estimate $h$, which gives one $\log$ in equation~\ref{eq:richards}, and then use the knowledge of how $h$ scaled with $k$ to estimate $k$, giving another $\log$ in that equation. The specific numerical values of the coefficients in equation~\eqref{eq:plastoratio} depend on exactly what lattice is being imagined as the precursor. The argument in~	\autocite{richardsPhyllotaxisItsQuantitative1951} is not easy to follow mathematically, creating an opportunity for several subsequent papers to attempt a more rigorous formulation~\autocite{thornleyPhyllotaxisIIDescription1975,jeanBib647,jeanBib278}.

Quite apart from the problems inherent in a $\log \log$, term, estimates of $R$ will be in practice very sensitive  to the estimate of position of the unknown centre of the disk. Even if the plastochrone ratio $R$ can be reliably estimated from the observed pattern then it seems much, much,  simpler to just count the $k$. But more fundamentally the Richards plastochrone index just assumes too much to discriminate between different developmental models.  
}
\end{jAnswer}


\section{Other arenas}
\mmafig{Txb0605Pineapple}{A golden lattice painted onto a bulging cylinder with the \gp{5,8} parastichies highlighted. The lattice is transformed so that the $z$-height in the planar lattice corresponds to arc-length $s$ in the vertical direction on the bulging cylinder. See text. }{1.0}
The map between a cylinder and a disk is a special case of the more general problem of lattice-like structures on surfaces of revolution. One way to map our regular cylindrical lattices to those on a bulging cylinder, that is a surface of revolution described by a profile function $r(z)$ giving the radius of the shape at each height, is to use the vertical arc-length $s(z)=\int^z \sqrt{1+r'}$.  If we interpret this as rise in the original lattice, then each point can be mapped 
onto this bulging shape. The final image in Figure~\ref{fig:Txb0605Pineapple} shows the shape from above and illustrates 
one way in which a cylinder-to-disk mapping can be derived from a shape function. But again it is important to bear in mind that in practice the patterns emerge on the developing shape, not through the transformation of an idealized cylindrical lattice. 

\section{Notes}
van Iterson himself did not restrict his lattice analyses to cylinders: he also explored, in effect, mapping lattices patterns onto cones and disks~\cite{vanitersonjrMathematischeUndMikroscopischAnatomische1907}. 
 Similar analyses were carried out by Richards~\cite{richardsPhyllotaxisItsQuantitative1951}, Ridley~\cite{ridleyIdealPhyllotaxisGeneral2001}, and Yeatts~\cite{yeattsGrowthcontrolledModelShape2004}. From the viewpoint of this book, though, all such models are limited in that they specify the divergence in advance  rather than viewing it as an outcome of  localised molecular node-placement dynamics. It is these models we turn to in the third Part. 