
\chapter{Stacked disk models}
\label{ch:stackeddisk}

Disk-stacking models for stem development have recently emerged as an excellent compromise between mathematical simplicity and biological relevance. 
Because disk-stacking models allow lattice solutions,  the van Iterson paradigm remains a powerful organising principle for their dynamics, and it is by now well established that disk-stacking models can indeed demonstrate Fibonacci structure~\cite{goleFibonacciQuasisymmetricPhyllotaxis2016}. Moreover, as the van Iterson paradigm suggests, there are parameter regions in which strict Fibonacci patterns are lost but patterns remain closely ordered with either Lucas numbers or double-Fibonacci pair counts occurring~\cite{goleFibonacciQuasisymmetricPhyllotaxis2016,yonekuraMathematicalModelStudies2019}.
As we will see, disk-stacking models are capable of exhibiting further empirical phenomena in ways no other published models can~\cite{swintonDiskstackingModelsAre2024}. However there remain many open questions about the dynamics of this important class of models.

\jpgfig{SchwendenerCollectedp202}{Schwendener's stacked-coin model showing transitions from a $\pp{3,5}$ to a $\pp{5,8}$ to a $\pp{8,13}$ parastichy~\autocite{schwendenerGesammelteBotanischeMittheilungen1898}. }{.2}

\clearpage#
\section{Motivation}
\jpgfig{2pstack}{A stacked-disk model, implemented with British tuppeny pieces, inspired by Atela~\autocite{atelaGeometricDynamicEssence2011}.}{0.5}
The disks of the model represent the inhibition zones in the SAM that are established by patterning of auxins and other morphogens, and the `gravity constraint' of adding the new disk at the lowest possible place corresponds to Snow's empirical rule that a new node will form as soon as it has enough space. The changing disk size corresponds to the change in proportion between the sizes of the inhibition zone and that of the SAM cylinder. It seems likely that most of the this change in relative geometry in actual plant growth is due to expansion of the SAM rather than modification of the inhibition zone size. So in some ways a more accurate visual representation than stacking variable-sized coins on a cylinder is to stack fixed-size coins on a cone. Figure~\ref{fig:2pstack} gives an example of such a model. The two model classes are not quite identical because the map between cone and cylinder will distort circles, but to the extent that the results of model simulations is not strongly dependent on the shape of the inhibitor zone we can expect similar results between the two.
But when we do not have a precise lattice we need a more general way to compute parastichy numbers for our patterns.


\clearpage
\section{Model structure}

\vnmafig{Ch8InhibitionBoundary}{Finding the next location in the stacked-disk model by finding the lowest point of the boundary of the inhibition zone, itself defined by the topmost coins. The width of the inhibition zone may be variable and defines the radius of the next coin.}{1}
Schwendener-type models are conceptually simple and fairly straightforward to implement, as illustrated in Figure~\ref{fig:Ch8InhibitionBoundary}. 
 Non-overlapping disks of diameter $D$ are successively placed at the lowest points possible on the cylinder,  touching two or three adjacent disks for their support. The centre of each disk then defines a node placement. The size of the $k$-th disk, $D(k)$, might be fixed or might vary with $k$, typically through the height of the stack so far.  In general we will be interested in $D(k)$ as a decreasing function of the height $z$ of the most recently placed disk, although when we look at the generation of sunflower capitulum patterns $D$ will be increasing. 
 
The series of stacking positions can be thought of as the trajectory of a map on the space of (recent) positions, and locally stable trajectories correspond to observed outcomes of the model. The map's output can be written as the vector from the $k$-th to the $k+1$-th-node; if this map has a fixed point then we will have a lattice in the sense of Chapter~\ref{ch:cylinder}. 
%
 We can 
think of the existing disks creating an inhibition field, shown in grey in the Figure: no new disk can be placed below the boundary of the region formed by disks centred on all the existing disks, but with twice their radius. The new disk is placed at the minimum of this boundary.  Attractively, this placement depends only on the most recent `chain' or `front' of disks around the cylinder, and is the minimum of a finite number of intersections of disks.

\clearpage

\subsection{Parastichy numbers in stacked-disk models}
\pdffig{txbDGParastichy}{Computing the parastichy counts for non-lattice patterns by counting up and down links in chains around the cylinder. If the patterns is close to a lattice, these counts coincide with the parastichy numbers of the lattice.}{.5}
While stacked-disk models \textit{can} generate lattices, usually they do not. 
So we can no longer rely on the  definitions of Chapter~\ref{ch:cylinder}
to find a global pair of parastichy numbers for the pattern, because  parastichies are not exactly straight lines.  However, they are not so far off, and there is still a way to compute parastichy pairs locally~\autocite{goleFibonacciQuasisymmetricPhyllotaxis2016}.
The process is shown in Figure~\ref{fig:txbDGParastichy}. Starting from one coin, we look for a shortest chain of touching coins that encircle the cylinder and returns to the original. That chain will have a number of up-steps to higher coins and down-steps to lower coins and the total number of up-steps and down-steps is exactly what we define as the parastichy count pair. 
%
This method coincides well with human assessment of spiral counts in relatively well ordered patterns but can still be assigned in strongly disordered patterns when the human eye is unable to identify structure. 
	 %
%	 
	 If the pattern is exactly a lattice, or near to one,
	 then other chains  through a given disk will have the same parastichy count pair, but near pattern transitions, as illustrated in Figure~\ref{fig:txbDGParastichy}, the presence of triangles in the pattern graph give rise to  multiple possible  counts. It is in this way that the discrete-valued parastichy counts can transition by jumps of 1 as the disk-radius changes slowly. 
	 Each of these jumps corresponds to a `$\gamma$' dislocation (taking its name from the shape of the letter~\cite{zagorska-marekPhyllotacticPatternsTransitions1985}) of the parastichy lines; conversely, $\lambda$ dislocations correspond to a decrease in the count.


\clearpage
\subsection{Stacked-disk models naturally generate Fibonacci transitions}
%
\vnmafig{Ch8NewTransitionFibonacci}{A stacked-coin model run, showing parastichy count transitions from a distichous pattern through Fibonacci numbers when the coin radius changes slowly enough. Each coin is added in turn with radius $r(z)$ where $z$ is the highest coin centre previously added, and $r(z)$ is linearly decreased from 1/2 to 1/12.}{1}
%
If we take one of the opposed touching-circle lattices with disk diameter $D$ of the previous chapter, truncate all the coins above one  point, and then use that as an initial condition for a stacked-coin model with fixed disk diameter $D$ we will regenerate the lattice. But a non-opposed lattices cannot in general be generated in this way because 
whenever a new coin is added, the contact lines through its two supports must be in opposite directions on the cylinder. 


Because disk-stacking models cannot generically generate non-opposed lattices, they ought to display a preference for Fibonacci structure, at least as long as the rate of change of disk radius with stem height,  is not too large.
Indeed solutions to stacked disk models showing large Fibonacci pair parastichy counts were exhibited by Bursill and colleagues in the 1980s~\cite{bursillSpiralLatticeConcepts1987,xudongPackingEqualDiscs1989} and more recently and systematically investigated by Gol\'e and colleagues~\cite{goleFibonacciQuasisymmetricPhyllotaxis2016}.

 We can see that Figure~\ref{fig:scpDeterministicTransition} reproduces these results by showing a (5,8) to (8,13) transition.
As disk radius continues to decrease, the process can be repeated, as illustrated in  Figure~\ref{fig:txbParastichyCountsTo89} which shows a series of transitions between adjacent Fibonacci pairs.  
%
\clearpage
\pdffig{txbParastichyCountsTo89}{Transitions between parastichy counts as a function of the slowly changing inverse radius of the disks. $y$ axis: Local chain parastichy counts for the topmost chain through each successive disk in a single
	deterministic run with a  $r'=0.03$ started from a single disk of radius 1/2. $x$ axis: inverse radius of the disk at the highest point of that chain.
	The thin grey vertical lines display the theoretical prediction that the transition from a parastichy number of $F{k-2}$ to $F_{k}$ should occur at $r{-1}=\sqrt{2}F_{k+2}$. 
}{1}
%


An understanding of the van Iterson tree adds further insight to these smooth, deterministic, transitions. In van Iterson parameter space, lattices change their parastichy numbers exactly when they are hexagonal lattices, then widen in internal angle into square lattices before narrowing again into hexagonal lattices at the next lower transition.  For slow rates of disk radius change, the disk pattern at the transition is very close to a hexagonal lattice, and the transition is able to occur smoothly. In particular, the positions of lattice points above and below the transition are strongly correlated: the change of parastichy number occurs by a smooth changeover between which of these points are closest to each other. 


As expected from the van Iterson tree, this disk-stacking model can account for the dominance of Fibonacci counts in the empirical data of Figure~\ref{fig:txbAhaPair}.  What it does not by itself account for is the other peaks in that data: we explore that later in this Chapter.

\clearpage
\section{Outstanding questions} 


It is tempting to stop at this point: we have built a model of node-formation which is informed by and fairly consistent with the known molecular biology, and we have shown that as biologically relevant parameter is varied, the model outputs pass through a series of increasingly complex patterns, each transition preserving the Fibonacci property. More than that, this property is generic and does not rely on model fitting. Apart from ensuring that the change of disk radius is slow enough, we have had to specify no particular parameters to achieve this sequence of Fibonacci transitions

However, quite apart from the question of biological validation, the mathematical satisfaction should be tempered by the fact that it took several attempts to generate Figure~\ref{fig:Ch8NewTransitionFibonacci}: vary the disk radius too slowly, and the model takes a biologically infeasible number of iterations to make the necessary transitions; vary it too quickly and the Fibonacci structure rapidly breaks down. 
Moreover there is more going on in the dynamics of these systems than transitions between lattices, and this more mathematically complex dynamics appears also to have at least some biological relevance.  So showing the   relevance of the van Iterson classification requires overcoming  a (to me) surprising mathematical obstacle: when is a lattice  adequate to describe stacked coin dynamics?

\clearpage
\section{Coherent structures in stacked-disk models}
\vnmafig{Ch8FixedRadius}{A stacked coin model with disk size fixed at that of a square $\pp{2,3}$ lattice, started from different  initial conditions with in which a single disk of the square lattice has been distorted in size. }{1}

\textit{Does} a stacked-coin model reliably lead to lattices? Intriguingly, the answer is no. 
Figure~\ref{fig:Ch8FixedRadius} shows the results of five simulations of a fixed-size stacked-coin model, of which the centre starts from an exact $\pp{2,3}$ lattice, and the neighbours start from a perturbation of that lattice.
On the left we can see our square $\pp{2,3}$ lattice with straight line parastichies. For the other initial conditions with the same disk radius the lines joining the disk centres are now longer exactly straight, yet would still be described by eye as having 2-parastichies and 3-parastichies, although not always in the same direction.

So a stacked-coin model can lead to solutions which are close to lattices, and if we start with a disk radius corresponding to a $\pp{m,n}$ lattice and an initial condition close to one, we might  hope to remain with a $\pp{m,n}$ parastichy count in our generalised sense. We can also expect that, just as the new third parastichy number at a lattice bifurcation is either $\pp{n+m}$ or $\pp{n-m}$, the same will be true for patterns near to those lattices, and indeed Golé et al~\cite{goleFibonacciQuasisymmetricPhyllotaxis2016} give an elegant argument based on chain transitions for this. But how close to a lattice do these patterns need to be for  the  elegant bifurcation theory of Chapter~\ref{ch:classifying} still to hold? 


\clearpage
\subsection{Cylinder tilings and rhombic tilings}
\label{sec:tilings}
\vnmafig{Ch8AreaTimeSeries}{Area of each polygon in the lattices of Figure~\ref{fig:Ch8FixedRadius}, read left to right, plotted from bottom to top.}{1}

One way to quantify the way in which the runs of Figure~\ref{fig:Ch8FixedRadius} depart from lattice structure is to plot a function associated with each node, such as the area of the polygon below it,  as a function of the dropped-coin sequence. This is done in Figure~\ref{fig:Ch8AreaTimeSeries}. These seem to be periodic graphs, corresponding to periodic orbits of the stacked-coin map. 


Just as a lattice corresponds to a fixed point of the next-node vector stacked-coin map, with a constant $(d,h)$ between nodes, a cylinder tiling corresponds to a low-order periodic solution of the map, with a sequence $((d_1,h_1),\ldots (d_k,h_k) )$. The upper parts of the tilings in Figure~\ref{fig:Ch8FixedRadius} are close approximations to cylindrical tilings.
A consequence of the disk stacking model is that each node in Figure~\ref{fig:Ch8FixedRadius} is generically at the apex of a single polygon. (Exceptionally, if the pattern is a hexagonal lattice, there will be two such polygons.)
More than that, in this Figure at least, all of the polygons of the tiling being approached appear to be rhombi: four-sided polygons with opposite parallel sides.%
A cylindrical tiling in which the polygon attached to each point is one of a finite number of different rhombi was called a rhombic tiling by~\cite{atelaRhombicTilingsPrimordia2017} who introduced them in this context.

Figure~\ref{fig:Ch8Tilings} shows an example in which the stacked-coin map is visually close to having periodicity of 6. The map trajectory defines a series of polygons attached to the coin centre which is its lowest point. In this rhombic-tiling pattern each coin has exactly one rhombus attached to it.

\clearpage
\vnmafig{Ch8Tilings}{Rhombic tilings arising from a stacked coin model.
	Tiles are coloured to emphasise a periodicity of 6 for each tiling after the first. }{1}





%	\item Proof for simple case of 3 fixed-radius disks 

\clearpage
Douady and Gol\'e gave an interesting example of a cylinder tiling and a lattice separately fitted to the same observation, which is shown in Figure~\ref{fig:Douadygole2016bFig4}.%
\clearedpdffig{Douadygole2016bFig4}{Fit of a lattice and a cylinder tiling to the same unrolled birch catkin image. The cylinder tiling is a 14\% better fit to the node positions than the lattice. From~\autocite{douadyFibonacciQuasisymmetricPhyllotaxis2016}.
	\copyright CC-BY Douady and Gol\'e 2016.}{1.0} {Figure 1 of ~\autocite{atelaRhombicTilingsPrimordia2017}} 
Since a tiling made up of $n$ different rhombi has at least $2n$ degrees of freedom, it can naturally give an better fit to an observed pattern than a lattice, which has $n=1$, and it is an open question whether empirical observations can at present reject lattice patterns in the framework of statistical hypothesis testing.%

It is striking that before these quasi-lattice patterns were seen in model outputs, there were few attempts to parameterise empirical data in any way other than as realisations of a lattice. One notable exception was Atela et al~\cite{atelaDynamicalSystemPlant2002} who suggested that a period 8 orbit they found in a placement map corresponded to a similar periodicity in divergences observed on a magnolia carpel~\cite{tuckerPhyllotaxisVascularOrganization1961}. 
Beyond this example, shifting the unit of analysis to a cylindrical or rhombic tiling has generated an interesting new series of observational questions~\autocite{douadyFibonacciQuasisymmetricPhyllotaxis2016}.


Rhombic-tilings are also important mathematically because, as Figure~\ref{fig:Ch8FixedRadius} illustrates, small perturbations to initial lattice patterns typically evolve into rhombic-tilings under fixed-size coin stacking.~\autocite{atelaRhombicTilingsPrimordia2017}.
Golé and Douady have proved that every initial chain that has a parastichy count of $(1,2)$ will indeed evolve into a rhombic tiling in finite time or become exponentially close to one, and conjectured 
this is true in general.~\autocite{goleConvergenceDiskStacking2020}.

% \subsection{Butterfly effects?}
It has been suggested that stacked-coin models provide an explanation of a `butterfly effect' in which the phyllotaxis of, e.g., the magnolia stem sometimes changes `for no obvious reason'\autocite{zagorska-marekSignificanceGandLdislocations2016}. Numerical simulations can indeed sometimes show surprisingly long intervals of apparent equilibrium followed by a burst of changes, and even if these trajectories
are only transient may be of value in exploring phyllotactic transitions in species like \textit{Magnolia}.

\clearpage
\section{Columnar models}
\label{sec:columns}

Exploration of the dynamics of stacked-coin models has revealed another intriguing pattern formation process. If instead of very slowly changing the coin radius we very quickly reduce it and then keep it small, we are effectively starting a fixed-radius run with random initial conditions and relatively small disks compared to the cylinder circumference. Figure~\ref{fig:Ch8TransitionLattices} gives an example.
It turns out, for reasons explored in~\autocite{goleFibonacciQuasisymmetricPhyllotaxis2016}, that
runs often converge to a series of near horizontal sets of disk stacked on top of each other. This stacking is fairly regular so that disks in either every row or every other row are vertically above each other. In the older language of phyllotaxis, these patterns might be called whorled orthostatic.  Douady and collaborators~\cite{goleConvergenceDiskStacking2020}, who have done unparalleled work on them,  call these QSS patterns, for quasi-symmetric solutions, but here I call them \textit{columnar} patterns.

This also provides a potential pattern mechanism for columnar patterns like those of the sweetcorn of Figure~\ref{fig:sweetcorn20191008} or of cacti. These models might well replace Turing Instability-based ones as a plausible pattern-formation mechanisms for such patterns. 

\vnmafig{Ch8TransitionLattices}{Runs of a stacked-coin model for increasingly steep radius changes. In each case, as in Figure~\ref{fig:Ch8NewTransitionFibonacci} the disk radius is reduced from one in which a lattice can, according to the van Iterson diagram, have a \gp{2,3} parastichy through a \gp{3,5} parastichy to a \gp{5,8} parastichy and then fixed. When the transition is slow enough for the \gp{3,5} lattice to establish well before before moving into the zone where a \gp{5,8} lattice is possible, that transition is visible  is followed by an equally ordered transition to what appears to be a rhombic tiling close to a \gp{5,8} lattice. 
	When the disk radius is reduced more rapidly, the ordered transitions are lost. In the central run, the parastichy count at the top of the cylinder are $\gp{7,8}$ while in the right hand run with a still faster reduction of radius the parastichy count at the top is $(7,7)$. 
}{.8}
\clearpage




It is clear numerically that the elegance of pattern analysis as transitions through near-to-touching-circle lattices is lost if the geometric change is too fast, and this can be interpreted as being because the close-packing property is lost: the constraints of a lattice pattern cannot adapt quickly enough to stay close to being well-packed,
and lattices as the units of analysis become unhelpful. 
But the extra degrees of freedom of rhombic tilings might well allow them to adapt more efficiently, in terms of close-packing, to geometric change than lattices do. So it might be that Fibonacci transitions through rhombic-tiling space can be maintained at higher rates of geometric change than if lattice structure were enforced. Arguments have already been made that rhombic tilings have their own more general version of the Hypothesis of Geometrical Phyllotaxis, and so taking the rhombic tiling as the unit of analysis rather than the lattice might provide 
a more biologically robust explanation of Fibonacci phyllotaxis than one based solely on lattices. 
\clearpage
\section{Finite-time dynamics and sunflower data}
Stacked-coin models provide an attractive generalisation of lattice models for exploring Fibonacci phyllotaxis, although their mathematical properties are much well less worked out, even for fixed disk size.  For example,  it is unclear if every attractor of such models is a cylinder tiling, or under what circumstances
such models can yield disk packings which are denser than those of the corresponding lattice. Nevertheless 

\section{expaper}

  
 
\subsection{Stacking fixed coins in fixed geometry}


\section{Observational data}
The reference dataset, and in particular Figure~\ref{fig:txbAhaPair},  is taken from the MOSI Turing's Sunflower's project's 2012 collection of data on several hundred sunflower heads~\cite{swintonNovelFibonacciNonFibonacci2016}.    This histogram of parastichy counts, and other data in the Supplementary Information of that paper can be summarised as follows.
\begin{enumerate}
	\item There was a strong but not complete preponderance of Fibonacci counts.
	\item	Excluding Fibonacci counts, the next commonest parastichy count was one less than a Fibonacci number (specifically, 33, 54, or 88).  A Fibonacci number less one (like 33) was statistically significantly more likely to occur than a Fibonacci number plus one (like 35).
	\item The next most common was a Lucas number (29, 47, or 76), and then a double-Fibonacci number (42, 68).
	\item It was common to see sunflower heads in which parastichy spirals could be clearly counted in one direction but not in another
	\item In a small number of relatively small sunflower heads, pairs of nearly matching but non-Fibonacci parastichy counts like (11,11)  were seen.
\end{enumerate}



\section{Results}

 


\subsection{Slow change rates support Fibonacci transitions}


\clearpage
\subsection{Parastichy count distributions after multiple transitions can match empirical data}
Figure~\ref{fig:txbAhaPair} illustrates the main result of this paper. It gives an example of the parastichy numbers observed with a single fixed $r'$ over the course of multiple stochastic replicates of a disk stacking model.    There are some qualitative differences at this particular example: the Fibonacci numbers are less overwhelmingly dominant than in the empirical dataset, while the Lucas numbers are more frequent, rather than less frequent, than the $F\pm1$ counts. Nevertheless these capture the key observations drawn from  Figure~\ref{fig:txbAhaPair}: the dominance of Fibonacci numbers, the  presence of Lucas and double-Fibonacci numbers, and,  not previously demonstrated in models, the presence of Fibonacci numbers plus or minus 1. 
The parameters $r'=0.03$ and noise $\sigma=0.05$ used to generate Figure~\ref{fig:txbAhaPair} were not specified \textit{a priori}. They were chosen after inspection of the wider grid of Figure~\ref{fig:scpTo55} but there was no further tuning of the model;  indeed the model has no further free parameters to tune.

\pdffig{txbAhaPair}{Left, empirical parastichy counts redrawn from Figure~\ref{fig:txbAhaPair} to emphasise non-Fibonacci observations. Right,  parastichy counts in a disk stacking model with $r'=0.03$ and noise $\sigma=0.05$. In each simulation the initial condition was a single disk of radius $r(0)=1/2$, with a disk size function $r(z)=1/2- r'z$, run until $r(z)$ changed by a factor of  60, corresponding in a lattice model to a change from a (0,1) lattice to a (34,55) one. 
	Simulated parastichy numbers were pooled over the course of each run and then further pooled over 10 replicates of the randomisation. }{1}

\subsection{Sensitivity of parastichy distributions to model parameters}

The two-dimensional parameter space arising from varying slope $r'$ and noise $\sigma$ was scanned as shown in  Figure~\ref{fig:scpTo55}. As expected, at slow disk change rates and low noise, a strong dominance of Fibonacci numbers 13, 21, 34 and 55 is observed in the upper left corner of the table. For rapidly changing and noisy simulations run over the same 60-fold reduction in disk radius, shown in the bottom right corner of the table, the 34 and 55 peaks completely disappear and there is a cluster of parastichy counts between 34 and 55, which appears to peak at the Lucas number 47, although as discussed below this is more a coincidence arising from the fact that 47 is close to the 44.5, the mean of 34 and 55, than a result of Fibonacci structure. 

Comparison of the first row (with low disk change speed $r'=0.01$) and the first column (with no noise: $\sigma=0$) suggests that the transition between these two outcomes can be achieved either by speeding up the disk radius change or by increasing noise, but these two mechanisms interact in a complex way: for example the $r=0.07$ row shows increasing noise first increases the prevalence of Fibonacci counts before then decreasing it. To understand this behaviour we need to look at the nature of the pattern transitions in more detail.

\pdffig{scpTo55}{Parastichy histograms depending on speed of disk size contraction $r'$ and magnitude of noise $\sigma$.  Runs and parastichy numbers colour-codings as in Figure~\ref{fig:txbAhaPair}. The histogram shown in Figure~\ref{fig:txbAhaPair} is highlighted. The vertical axis of each histogram is a relative frequency count; bars higher than 0.07 are truncated vertically.}{1}


\subsection{Deterministic transitions around criticality generate near-Fibonacci counts, but also introduce dynamic noise}

This section explores how a pattern starting from an exact $(8,13)$ lattice evolves: Figure~\ref{fig:scpDet813} illustrates some example runs.
 \pdffig{scpDet813}{Deterministic dynamics starting from an (8,13) lattice at different transition rates $r'=$0.05, 0.1, 0.5, 1.0 respectively. In each case the disk size is fixed outside of the transition zone marked with a gray rectangle, and linearly changed by a ratio of 1.6 within that transition zone as defined in the Method section with slope $r'$.  Histograms are of $\log_{10} (1+x) $ where $x$ is the binned parastichy count at each disk evaluated using the chain method of Douady et al, and coloured as in Figure~\ref{fig:txbAhaPair}. Below each histogram is the corresponding set of parastichy lines; changes in parastichy counts are associated with $\gamma$ and $\lambda$ dislocations in these patterns.}{.9}
%
 van Iterson lattice theory predicts that the gradient of what is initially the higher parastichy count lines, here the left-winding blue 13-parastichies, will gradually rotate through the transition zone. The gradient of the lower, 8-parastichy lines, which are the red, right-winding ones, also rotates slowly below the transition point, but then there is a sharp transition to the 13-parastichies. Moreover  van Iterson theory predicts that at that point the 8- and 13-lines are $120^{\circ}$ apart, and the new 21-lines bisect this angle and make an angle of $60^\circ$ with each of them, reflecting the hexagonal lattice underlying the bifurcation point. Before the transition the two angles each set of parastichy lines make with the vertical are asymmetric, with the blue lines steeper, while after they are also asymmetric but now it is the red lines which are steeper.
 
  In the transition region itself a series of dislocations in the red lines correspond to a rapid change of the red count from 8 up to 21, which can be detected in the 
 log histogram for the parastichy counts, but is always localised to the transition region. Turning to the 13-parastichies,  a further dynamic complexity emerges.
 For the  $r'=0.5$ panel of  Figure~\ref{fig:scpDet813} there is an occasional occurrence of 14 in the left winding count. Although the relative frequency of this $F+1$ count is low, it does intermittently occur, and occurs through occasional $\gamma$ dislocations away from  transition zone; by contrast the 21-parastichies remain conserved once past the transition region. 
 
At the yet higher value of $r'=1$, two further effects are visible in the left-winding parastichy lines. First of all,  the typical left parastichy count increases from 13 to 16; in addition dislocations in both the left- and right- winding parastichy lines continue over the  course of the entire simulation and produce a blurring of the peaks so that both 15 and 17 are seen in the left counts and 19, 20, and 21 in the right counts. All of these more complex change in parastichy counts occur outside of the transition zone and during the fixed-disk-size regime of the simulation.  In summary a form of deterministic noise is introduced into the dynamics of the model by a the jump to new and non-lattice conditions  in the narrow transition zone; near to the critical $r'$ value below which Fibonacci transitions occur this can promote $F+1$ counts.


\subsection{Rapid transitions promote near-Fibonacci counts while noise promotes columnar structures}
This numerical observation of deterministic chaos arising only at rapid transitions offers one intriguing potential explanation for non-Fibonacci structure. In order to explore whether this has a detectably different signature to the presence of biological noise we further explore the effect of non-zero noise $\sigma$ in the model.   
Figure~\ref{fig:scpSch813} is an analogue of Figure~\ref{fig:scpDet813},
 with the indicated level of noise.
We can see a number of the same phenomena as in the deterministic runs: a narrow transition zone still acts to perturb the initial condition of the post transition fixed-disk-size dynamics, with the corresponding shift of the parastichy counts. So for example, the $r'=1$, $\sigma=0.02$ case shows a predominance of the 16 and 19 counts analogously to the deterministic model. The same underlying deterministic dynamics also creates paired $\gamma$ and $\lambda$ dislocations that generate small amounts of blurring of the peaks.
 
\pdffig{scpSch813}{Stochasticity flattens out parastichy count histograms, but also shifts them into the centre of the Fibonacci pair range. Single replicates of Figure~\ref{fig:scpDet813},  but with the addition of noise $\sigma=0.02$ or $\sigma=0.1$.} {.8}

The impact of noise is clearly seen in the increased number of dislocations in the parastichy lines. As specified in the methods section,  noise is implemented in such a way that disk positions are changed but disks are considered to remain in contact for drawing parastichy lines, so the contact lines are \textit{not} broken by this perturbation. Instead, additional dislocations can occur alongside those caused by the deterministic dynamics because the disk above the randomly-moved disk is in turn moved `out of position'. 

However there is an additional phenomenon which is not easily seen in the deterministic runs, but can be noticed in, for example the $r'=1$, $\sigma=0.1$ case. Looking at the gradient of the right-winding red parastichy lines after the transition zone, they share the gradient of the 21-parastichies seen in the less noisy case $\sigma=0.02$ and indeed correspond to parastichy counts of 21 in that region. However, starting some considerable time after the transition zone, there is a pattern shift and a series of unpaired $\lambda$ dislocations see a flattening of the gradient to about 45$^\circ$. The left-winding blue parastichies  also  become close to 45$^\circ$, although since they leave the transition zone already closer to 45$^\circ$ anyway the shift is less clear. 

What is happening in this new transition is a loss of asymmetry:  Fibonacci type structure requires that the one of the families of parastichy lines is much flatter than the other. In terms of disk placement this means that say, the disk above and to the left (say) of any given disk must be in general lower than the disk above and to the right of that disk. However with random initial conditions, or run stochasticity, there is no enforcement of this property, and there appears no theoretical reason why the dynamics of the disk-stacking model should tend to return the asymmetry. As a consequence, a generic run of the model from arbitrary initial conditions should be expected to yield parastichy lines with nearly equal angles, and correspondingly nearly equal parastichy counts. If, as in these simulations, the disk radius at the end of the simulation is calculated to allow a (13,21) pair, then this argument shows that two nearly equal parastichy counts adding to 34, in other words (17,18) are also likely patterns. Similarly, if the geometry is such that a (21,34) pair is possible then a generic pattern will be close to a (27,28) pair. These columnar solutions to the disk stacking model were identified by Gol\'e and colleagues~\cite{goleFibonacciQuasisymmetricPhyllotaxis2016} as QSS or `quasi-symmetric scenarios'  and an example is shown in Figure~\ref{fig:scpColumn}.
\pdffig{scpColumn}{Column-type patterns (and roughly equal parastichy pairs) emerge when noise is large enough to move away from Fibonacci structure. The resulting pattern contains patches in which the parastichy lines are each at $45^\circ$ and the disks are arranged in alternating vertical columns, which correspond to roughly equal parastichy numbers, here 16, 17 or 18. As Figure~\ref{fig:scpSch813} with $r'=0.1$ and $\sigma=0.1$, }{1}

\subsection{Double-Fibonacci and Lucas numbers emerge at large $r$ near the critical $r'$ and are promoted by noise}
The possibility of double-Fibonacci and Lucas numbers in disk-stacking models has already been demonstrated to occur near critical $r'$ values (e.g. Figure 15 of~\cite{goleFibonacciQuasisymmetricPhyllotaxis2016} or Figure 5 of ~\cite{yonekuraMathematicalModelStudies2019}.) However these results by themselves do not indicate at which point in pattern evolution these shifts from exact to more general Fibonacci structure occur. 

van Iterson theory shows a strong geometrical constraint on the parastichy numbers near lattice transitions. If the parastichy numbers before the bifurcation are $m$ and $n$ with $m<n$ then then the lattice geometry enforces that the higher parastichy number after bifurcation must be $m+n$; depending on the choice of branch in Figure~\ref{fig:scpVanIterson} the lower parastichy number is either $m$ or $n$, and only on the opposed branch is it $n$ so that the new pair is $n,n+m$. This branch choice is enough to enforce successive Fibonacci parastichy pairs from a starting pair of $(1,2)$, but it has long  been noticed that it also enforces successive Lucas pairs from a starting pair of $(1,3)$ and successive double-Fibonacci pairs from a starting pair of $(2,4)$, and that as we saw in the MOSI dataset such counts are frequent amongst non-Fibonacci observations. van Iterson lattice theory, though, cannot readily explain why and when these jumps which  ignore the normal branch choice occur.  If  it is possible to jump across the van Iterson tree once why not many times? In other words, suppose a Lucas pair $(11,18)$ is observed. Has that indeed evolved from an initial $(1,3)$ pattern? Or is it more likely for the jump to occur later so that, e.g., a $(8,13)$ pair directly transitions into an $(11,18)$. 
Disk stacking models support the hypothesis that only early transitions are likely: if either noise or too-fast transitions are strong enough effects to shift away from the left-right asymmetry of the  Fibonacci structure, the most likely resulting patterns are the columnar structures of the previous section, and these are close to Lucas or double-Fibonacci patterns only at the beginning of pattern formation. 


\subsection{Disk stacking models generate falling phyllotaxis and observed asymmetry on the capitulum}

The observational approach adopted in the 2016 analysis of the MOSI images was to visually identify spirals as far out on the capitulum as possible, and so for the primary analysis, Figure~\ref{fig:txbAhaPair} it has been enough to consider only rising phyllotaxis, corresponding to decreasing $f$. 
However when analysing the MOSI images, we did call attention to a previously unreported departure from Fibonacci structure, which we there called a lack of rotational symmetry~\cite{swintonNovelFibonacciNonFibonacci2016}. Perhaps surprisingly, the stacked-disk model for falling phyllotaxis, with increasing $f$ is also capable of illustrating this type of pattern, as illustrated in Figure~\ref{fig:scpFalling}, which 
shows such a simulation, together with a mapping of the simulation back to the capitulum surface as described in the Methods section. 
For rising phyllotaxis, the transition zones where parastichy number change are very narrow and horizontal, even in the highly-disordered example of Figure~\ref{fig:scpColumn}. By contrast, the transition zone in this falling phyllotaxis example is not horizontal, but follows the gradient of the parastichy lines and so  there is a large region of the cylinder on which the parastichy counts are difficult and perhaps meaningless to assign, Looking at the lower right panels we can see how the uncountability manifests itself: there is a family of 21-ish red contact lines making one angle in the cylinder, and another family of 8 red-contact lines making another, but the two families coexist over an extended region of the cylinder. This is the pattern that was identified in sample 667 of the MOSI image, whose hand-assigned parastichy lines are reproduced in Figure~\ref{fig:667nophoto}.
 
\pdffig{scpFalling}{Stacked-disk models on the cylinder can generate radially asymmetric patterns by rising phyllotaxis to the capitulum rim followed by falling phyllotaxis on the capitulum. This simulation shows horizontal variability in the location on the cylinder of dislocations of the red right-ward leaning parastichy lines. When the cylinder is mapped to a disk, this corresponds to a visible rotational asymmetry on the simulated capitulum.    There is no satisfying way of assigning `spiral counts' on the outer rim for the red contact lines: compare Figure~\ref{fig:667nophoto}.}{1}

\jpgfig{667nophoto}{Observed contact line patterns in sample 667 of the MOSI dataset, with no satisfying way to assign a spiral count to the family of red/yellow/green spirals reflecting a rotational asymmetry as modelled in Figure~\ref{fig:scpFalling}.}{.5}


variable-size  stacked-coin models at present provide a good candidate for being able to make precise the idea of the `smoothly changing lattice' in a biologically important way and they are the subject of active research.

\section{Further reading}Disk-stacking models were first introduced by Schwendener in the 1870s as an exploration of organ placement in plants~\cite{schwendenerMechanischeTheorieBlattstellungen1878}.  Schwendener sketched the patterns seen when disks of decreasing size were stacked, one after the other, around a cylinder (Figure~\ref{fig:SchwendenerCollectedp202}).
Although Schwendener's work was not influential on mainstream plant morphology, it seems to have influenced the better known 1907 PhD thesis of van Iterson. van Iterson  who explored the possible patterns arising from arrangements of fixed-size disks as a function of disk size and first drew a version of the van Iterson diagram~\cite{vanitersonjrMathematischeUndMikroscopischAnatomische1907}. 
A version of this model in which the fronts were curves called `pseudoconchoids'  was  arrived at by Schoute at the first quarter of the 20th century~\cite{schouteUberPseudokonchoiden1913}. According to  Richards' account~\cite{richardsGeometryPhyllotaxisIts1948}  of this German-language work, a pseudoconchoid allowed phyllotactic transitions by giving the front shape `sufficient flexibility'. But 
the attention of mathematicians in the twentieth-century was almost entirely restricted to elaborating the van Iterson lattice theory described in the first half of this book.  It was not until the  1990s that van der Linden demonstrated the first simulations of a Schwendener-type model showing large Fibonacci numbers~\cite{vanderlindenCreatingPhyllotaxisDislodgement1990}.

The 21st century has seen a  revival of interest in the dynamics of these stacked-disk models~\cite{
	atelaDynamicalSystemPlant2002,
	atelaRhombicTilingsPrimordia2017,
	hottonPossibleActualPhyllotaxis2006,
	atelaGeometricDynamicEssence2011,adlerConsequencesContactPressure1977,godinPhyllotaxisGeometricCanalization2020};  I have been particularly strongly influenced by~\cite{goleFibonacciQuasisymmetricPhyllotaxis2016} and~\cite{goleConvergenceDiskStacking2020}. 	
