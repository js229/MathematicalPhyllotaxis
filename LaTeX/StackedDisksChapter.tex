
\chapter{Stacked disk models}
\label{ch:stackeddisk}


In response to the limitation of lattice models, disk-stacking models have been recently revived~\cite{atelaDynamicalSystemPlant2002}. Although they date from the nineteenth century they have only recently become a significant paradigm for pattern formation~\cite{godinPhyllotaxisGeometricCanalization2020,goleFibonacciQuasisymmetricPhyllotaxis2016}. Because disk-stacking models allow lattice solutions,  the van Iterson paradigm remains a powerful organising principle for their dynamics, and it is by now well established that disk-stacking models can indeed demonstrate Fibonacci structure~\cite{goleFibonacciQuasisymmetricPhyllotaxis2016}. Moreover, as the van Iterson paradigm suggests, there are parameter regions in which strict Fibonacci patterns are lost but patterns remain closely ordered with either Lucas numbers or double-Fibonacci pair counts occurring~\cite{goleFibonacciQuasisymmetricPhyllotaxis2016,yonekuraMathematicalModelStudies2019}. An interesting possibility, not obvious within the van Iterson paradigm and only recently reported in the mathematical literature, is of paired but approximately equal parastichy counts~\cite{goleFibonacciQuasisymmetricPhyllotaxis2016}; these arise from a loss of the strong left-right asymmetry implied by Fibonacci structure. 



 Disk-stacking models were first introduced by Schwendener in the 1870s as an exploration of organ placement in plants~\cite{schwendenerMechanischeTheorieBlattstellungen1878}.  Schwendener sketched the patterns seen when disks of decreasing size were stacked, one after the other, around a cylinder (Figure~\ref{fig:schwendener1878}).%

Although Schwendener's work was not influential on mainstream plant morphology, it did influence the better known 1907 PhD thesis of van Iterson who explored the possible patterns arising from arrangements of fixed-size disks as a function of disk size and first drew a version of the van Iterson diagram~\cite{vanitersonjrMathematischeUndMikroscopischAnatomische1907,	schouteUberPseudokonchoiden1913}.
 The 21st century has seen a modest revival of interest in the dynamics of these stacked-disk models~\cite{
	atelaRhombicTilingsPrimordia2017,
	hottonPossibleActualPhyllotaxis2006,
	atelaGeometricDynamicEssence2011};  this paper was particularly strongly influenced by~\cite{goleFibonacciQuasisymmetricPhyllotaxis2016} and~\cite{goleConvergenceDiskStacking2020}. 	

Schwendener-type models are conceptually simple and fairly straightforward to implement (Figure~\ref{fig:scpDeterministicTransition}).
 We take a vertical cylinder with circumference fixed at 1. Apart from the initial conditions, the deterministic model is entirely defined by the function $r_i$ which gives the radius of the $i$th disk. For the models in this paper $r_i$ is a function only of $z_{i-1}$, the height of the centre of the most recently placed disk.
 For the tabulated results on parastichy counts, we take  $r(z)$ to be a piece-wise linear function of $z$ with $r(z)=r_L$ for $z<z_L$, $r(z)=r_U$ for $z>z_U$ and $r$ linearly interpolated between $r_L$ and  $r_U$ in the transition region from `Lower' to `Upper'. We use the slope of this linear interpolation, $r'= -(r_U-r_L)/(z_U-z_L)$, corresponding to the speed at which the disk size changes and chosen to be positive when the disk size is decreasing, as a control parameter in the simulations below.
 %

%
\pdffig{scpDeterministicTransition}{
	Stacked disk models with varying disk radii enable transitions in parastichy counts. (a): Intermediate stage of a disk stacking model started near a square  (5,8) lattice. Disks are joined by lines to those they are in direct contact with, and the lines coloured red or blue depending on whether they slope upwards to the right or left. The position of the next disk to be placed (solid red) is determined only by the disks within the grey rectangle, and with the corresponding chain of connections shown with a thicker line. (b): completed model run, showing parastichy lines joining disk centres and a transition  from a (5,8) to an (8,13) lattice-like pattern. (c) right parastichy lines only, showing that a parastichy count of 8 is conserved throughout the run.  (d)  left parastichy lines only, showing how a series of $\gamma$ dislocations correspond to increases in the parastichy count from 8 up to 13. These transitions correspond to non-linearity of the topmost chain, as highlighted in (a). Diagram (a) also illustrates the Douady-Gol\'e method of parastichy assignment:  when the chain in (a) is traversed around the cylinder there are always 8 distinct sets of blue segments in the path, corresponding to the unchanging 8-parastichy of the red spirals in (c). By contrast the number of distinct sets of red segments can vary depending on the traversal taken, which allows this count to increase from 5 to 13 over the course of the run. In the numerical results the parastichy count for each disk is derived from the highest traversal through but not above that disk.
}{1}%  
 
  In order to introduce noise with magnitude $\sigma$ we first compute the position of the $i$-th disk using the deterministic model, but then multiply its radius by a scaling drawn from a random distribution which is uniform on $[1-\sigma,1+\sigma]$. It is this randomly rescaled disk which is then used to compute the position of subsequent disks. Disks are treated as `touching' for the purposes of computing parastichies if the un-rescaled disks touch. 
  
 
\subsection{Cylinder to capitulum mappings}

The output from disk-stacking models can be mapped from cylinders to any surface of revolution, as in Figure~\ref{fig:scpConeTransformation} which shows how  results of disk-stacking models can be compared with empirical data on spirals at the outer sunflower seedhead rim. 
%
\pdffig{scpConeTransformation}{Mapping a disk-stacked pattern to a seedhead pattern. Top: a rising and then falling phyllotaxis modelled using a radius function whose inverse first linearly increases then linearly decreases; functional types corresponding to the colours and labels are arbitrarily added later to aid visualisation. After nodes labelled as uncommitted, bracts, and ray florets, and above an arbitrarily chosen point $z_S$, nodes are deemed to correspond to seed positions in the mature seed head.  Still higher node positions are deemed to correspond to a disk floret which does not proceed to set seed, as is common in the centre of large seedheads, up to the end of the run at $z=z_U$.   Right parastichy lines are shown in the region which will correspond to mature seeds in the adult seedhead. Below: the resulting placement pattern and parastichy lines mapped onto a seedhead disk. Points at coordinates $(x,z)$ on the cylinder are mapped to radial coordinates $\rho=(z_U-z)/(z_U-z_S)$, $\theta=2\pi x$. Note that occasionally some disks which are in contact in the cylinder pattern, for example in the ray florets), correspond to Voronoi cells which are not in  contact, partly because of the non-isometry of the mapping from cylinder to seedhead disk.}{1}

Crucially for the purposes of comparison to the empirical data, the principal parastichy counts are conserved by this mapping as long as it is not too distorting, which means that we do not have to address here the difficult biomathematical question of inferring the mapping from observational data. We do though note the dimensional argument that the disk-radius function $r(z)$ on the cylinder should scale like  the inverse of the circumference of the stem tip at the developmental stage when organ commitment takes place, together with the very crude assumption that this is in turn roughly proportional to $R(z)$, the mature stem radius, to find that roughly $r(z)\approx c/R(z)$. Observation on mature sunflowers suggests  a change in $R$ and hence $r$ of between 10 and 100 between stem and capitulum rim. 
We can compare this the results of Figure~\ref{fig:scpParastichyCountsTo89} below to see that, depending on the size of the initial disk radius, a scaling of  around 85 to 90 is needed to transition to a parastichy pair of (55,89). We might note that $89=F_{11}$, and more generally that van Iterson theory~\cite{swintonMathematicalPhyllotaxis2023} provides convincing arguments that if a transition to a Fibonacci pair $(F_n,F_{n+1})$ occurs, it is at a scaling of the disk radius of about $F_{n+1}$ from that of a $(0,1)$ lattice. 

We concentrate here on functional forms for $r(z)$ which decrease linearly to the point where the rim of the sunflower develops, corresponding to the data analysis of~\cite{swintonNovelFibonacciNonFibonacci2016} which counted the parastichy numbers at the outside rim of the seedhead. The majority of our numerical results are taken at the end of this linear  period, as indicated in Figure~\ref{fig:scpConeTransformation} by the points suggestively coloured for explanatory purposes as though they were bracts, ray-floret and seed positions, although our results do not rely on this classification. 



\subsection{Parastichy numbers in stacked-disk models change at $\gamma$ and $\lambda$ dislocations}
 Parastichy numbers for the resulting patterns are assigned as follows, using Douady and Gol\'e's method~\cite{goleFibonacciQuasisymmetricPhyllotaxis2016}, and as shown in Figure~\ref{fig:scpDeterministicTransition}. This method coincides well with human assessment of spiral counts in relatively well ordered patterns but can still be assigned in strongly disordered patterns when the human eye is unable to identify structure. 
%
The mathematical definition of parastichy numbers to model spiral counts is well established for lattices~\cite{swintonMathematicalPhyllotaxis2023}, and gives a pair of integers which is fixed across the whole lattice. Stacked-disk models usually generate patterns which are not parts of lattices, but we look instead for a chain of touching disks that encircle the cylinder and return to the original.   A chain has a number of up-steps to higher disks and down-steps to lower disks and the pair of the total number of up-steps and down-steps is what we define as the parastichy count pair. The `top chain' for a disk is the unique chain that consists only of disks no higher than the starting disk and always choosing the higher of the possible next contact disks to the right. (In disordered regions there can occasionally be no contact disks to the right in which case the lowest of those to the left is chosen.)  Generally, we take the parastichy pair for a disk to be that of its top chain. 

If the pattern is exactly a lattice, or near to one,
then other chains  through a given disk will have the same parastichy count pair, but near pattern transitions, as illustrated in Figure~\ref{fig:scpDeterministicTransition}, the presence of triangles in the pattern graph give rise to  multiple possible  counts. It is in this way that the discrete-valued parastichy counts can transition by jumps of 1 as the disk-radius changes slowly. 
Each of these jumps corresponds to a `$\gamma$' dislocation (taking its name from the shape of the letter~\cite{zagorska-marekPhyllotacticPatternsTransitions1985}) of the parastichy lines,  as visible in Figure~\ref{fig:scpDeterministicTransition}(d). 
Conversely, $\lambda$ dislocations correspond to a decrease in the count.
\section{Observational data}
The reference dataset, and in particular Figure~\ref{fig:scpMOSIHistogram},  is taken from the MOSI Turing's Sunflower's project's 2012 collection of data on several hundred sunflower heads~\cite{swintonNovelFibonacciNonFibonacci2016}.    This histogram of parastichy counts, and other data in the Supplementary Information of that paper can be summarised as follows.
\begin{enumerate}
	\item There was a strong but not complete preponderance of Fibonacci counts.
	\item	Excluding Fibonacci counts, the next commonest parastichy count was one less than a Fibonacci number (specifically, 33, 54, or 88).  A Fibonacci number less one (like 33) was statistically significantly more likely to occur than a Fibonacci number plus one (like 35).
	\item The next most common was a Lucas number (29, 47, or 76), and then a double-Fibonacci number (42, 68).
	\item It was common to see sunflower heads in which parastichy spirals could be clearly counted in one direction but not in another
	\item In a small number of relatively small sunflower heads, pairs of nearly matching but non-Fibonacci parastichy counts like (11,11)  were seen.
\end{enumerate}

\pdffig{scpMOSIHistogram}{Counts of how often each parastichy number  was observed in the MOSI dataset. Redrawn from~\cite{swintonNovelFibonacciNonFibonacci2016}. An expanded view is shown in Figure~\ref{fig:scpAhaPair}.}{1}


\section{Results}

 


\subsection{Slow change rates support Fibonacci transitions}
	
Solutions to stacked disk models showing large Fibonacci pair parastichy counts were exhibited by Bursill and colleagues in the 1980s~\cite{jeanBib41,jeanBib721}.  Previous explanations for consistent choice of the Fibonacci branch in the van Iterson tree have either assumed that non-opposed lattices can never occur~\cite{mitchisonPhyllotaxisFibonacciSeries1977} or that they can be rejected as less-well-packed than the corresponding opposed lattice near the hexagonal bifurcation~\cite{douadyCh21Phyllotactic1998}. Because disk-stacking models cannot generically generate non-opposed lattices, they ought to display a preference for Fibonacci structure, at least as long as $r'$, the rate of change of disk radius with stem height,  is not too large. This was demonstrated in  numerical calculations of Gol\'e and colleagues~\cite{goleFibonacciQuasisymmetricPhyllotaxis2016}. We can see that Figure~\ref{fig:scpDeterministicTransition} reproduces these results by showing a (5,8) to (8,13) transition.
As $r$ continues to decrease, the process can be repeated, as illustrated in  Figure~\ref{fig:scpParastichyCountsTo89} which shows a series of transitions between adjacent Fibonacci pairs.  
%
\clearpage
\pdffig{scpParastichyCountsTo89}{Transitions between parastichy counts as a function of the slowly changing inverse radius of the disks. $y$ axis: Local chain parastichy counts for the topmost chain through each successive disk in a single
	deterministic run with a  $r'=0.03$ started from a single disk of radius 1/2. $x$ axis: inverse radius of the disk at the highest point of that chain.
	The thin grey vertical lines display the theoretical prediction that the transition from a parastichy number of $F{k-2}$ to $F_{k}$ should occur at $r{-1}=\sqrt{2}F_{k+2}$. 
	}{1}
%


An understanding of the van Iterson tree adds further insight to these smooth, deterministic, transitions. In van Iterson parameter space, lattices change their parastichy numbers exactly when they are hexagonal lattices, then widen in internal angle into square lattices before narrowing again into hexagonal lattices at the next lower transition.  The tiles of Figure~\ref{fig:scpDeterministicTransition} reflect this transition. Because $r'$ is small, the disk pattern at the transition is very close to a hexagonal lattice, and the transition is able to occur smoothly. In particular, the positions of lattice points above and below the transition are strongly correlated: the change of parastichy number occurs by a smooth changeover between which of these points are closest to each other. 



As expected from the van Iterson tree, this disk-stacking model can account for the dominance of Fibonacci counts in the empirical data of Figure~\ref{fig:scpMOSIHistogram}.  What it does not by itself account for is the other peaks in that data, which is the main concern of this paper. 


\clearpage
\subsection{Parastichy count distributions after multiple transitions can match empirical data}
Figure~\ref{fig:scpAhaPair} illustrates the main result of this paper. It gives an example of the parastichy numbers observed with a single fixed $r'$ over the course of multiple stochastic replicates of a disk stacking model.    There are some qualitative differences at this particular example: the Fibonacci numbers are less overwhelmingly dominant than in the empirical dataset, while the Lucas numbers are more frequent, rather than less frequent, than the $F\pm1$ counts. Nevertheless these capture the key observations drawn from  Figure~\ref{fig:scpMOSIHistogram}: the dominance of Fibonacci numbers, the  presence of Lucas and double-Fibonacci numbers, and,  not previously demonstrated in models, the presence of Fibonacci numbers plus or minus 1. 
The parameters $r'=0.03$ and noise $\sigma=0.05$ used to generate Figure~\ref{fig:scpAhaPair} were not specified \textit{a priori}. They were chosen after inspection of the wider grid of Figure~\ref{fig:scpTo55} but there was no further tuning of the model;  indeed the model has no further free parameters to tune.

\pdffig{scpAhaPair}{Left, empirical parastichy counts redrawn from Figure~\ref{fig:scpMOSIHistogram} to emphasise non-Fibonacci observations. Right,  parastichy counts in a disk stacking model with $r'=0.03$ and noise $\sigma=0.05$. In each simulation the initial condition was a single disk of radius $r(0)=1/2$, with a disk size function $r(z)=1/2- r'z$, run until $r(z)$ changed by a factor of  60, corresponding in a lattice model to a change from a (0,1) lattice to a (34,55) one. 
	Simulated parastichy numbers were pooled over the course of each run and then further pooled over 10 replicates of the randomisation. }{1}

\subsection{Sensitivity of parastichy distributions to model parameters}

The two-dimensional parameter space arising from varying slope $r'$ and noise $\sigma$ was scanned as shown in  Figure~\ref{fig:scpTo55}. As expected, at slow disk change rates and low noise, a strong dominance of Fibonacci numbers 13, 21, 34 and 55 is observed in the upper left corner of the table. For rapidly changing and noisy simulations run over the same 60-fold reduction in disk radius, shown in the bottom right corner of the table, the 34 and 55 peaks completely disappear and there is a cluster of parastichy counts between 34 and 55, which appears to peak at the Lucas number 47, although as discussed below this is more a coincidence arising from the fact that 47 is close to the 44.5, the mean of 34 and 55, than a result of Fibonacci structure. 

Comparison of the first row (with low disk change speed $r'=0.01$) and the first column (with no noise: $\sigma=0$) suggests that the transition between these two outcomes can be achieved either by speeding up the disk radius change or by increasing noise, but these two mechanisms interact in a complex way: for example the $r=0.07$ row shows increasing noise first increases the prevalence of Fibonacci counts before then decreasing it. To understand this behaviour we need to look at the nature of the pattern transitions in more detail.

\pdffig{scpTo55}{Parastichy histograms depending on speed of disk size contraction $r'$ and magnitude of noise $\sigma$.  Runs and parastichy numbers colour-codings as in Figure~\ref{fig:scpAhaPair}. The histogram shown in Figure~\ref{fig:scpAhaPair} is highlighted. The vertical axis of each histogram is a relative frequency count; bars higher than 0.07 are truncated vertically.}{1}


\subsection{Deterministic transitions around criticality generate near-Fibonacci counts, but also introduce dynamic noise}

This section explores how a pattern starting from an exact $(8,13)$ lattice evolves: Figure~\ref{fig:scpDet813} illustrates some example runs.
 \pdffig{scpDet813}{Deterministic dynamics starting from an (8,13) lattice at different transition rates $r'=$0.05, 0.1, 0.5, 1.0 respectively. In each case the disk size is fixed outside of the transition zone marked with a gray rectangle, and linearly changed by a ratio of 1.6 within that transition zone as defined in the Method section with slope $r'$.  Histograms are of $\log_{10} (1+x) $ where $x$ is the binned parastichy count at each disk evaluated using the chain method of Douady et al, and coloured as in Figure~\ref{fig:scpAhaPair}. Below each histogram is the corresponding set of parastichy lines; changes in parastichy counts are associated with $\gamma$ and $\lambda$ dislocations in these patterns.}{.9}
%
 van Iterson lattice theory predicts that the gradient of what is initially the higher parastichy count lines, here the left-winding blue 13-parastichies, will gradually rotate through the transition zone. The gradient of the lower, 8-parastichy lines, which are the red, right-winding ones, also rotates slowly below the transition point, but then there is a sharp transition to the 13-parastichies. Moreover  van Iterson theory predicts that at that point the 8- and 13-lines are $120^{\circ}$ apart, and the new 21-lines bisect this angle and make an angle of $60^\circ$ with each of them, reflecting the hexagonal lattice underlying the bifurcation point. Before the transition the two angles each set of parastichy lines make with the vertical are asymmetric, with the blue lines steeper, while after they are also asymmetric but now it is the red lines which are steeper.
 
  In the transition region itself a series of dislocations in the red lines correspond to a rapid change of the red count from 8 up to 21, which can be detected in the 
 log histogram for the parastichy counts, but is always localised to the transition region. Turning to the 13-parastichies,  a further dynamic complexity emerges.
 For the  $r'=0.5$ panel of  Figure~\ref{fig:scpDet813} there is an occasional occurrence of 14 in the left winding count. Although the relative frequency of this $F+1$ count is low, it does intermittently occur, and occurs through occasional $\gamma$ dislocations away from  transition zone; by contrast the 21-parastichies remain conserved once past the transition region. 
 
At the yet higher value of $r'=1$, two further effects are visible in the left-winding parastichy lines. First of all,  the typical left parastichy count increases from 13 to 16; in addition dislocations in both the left- and right- winding parastichy lines continue over the  course of the entire simulation and produce a blurring of the peaks so that both 15 and 17 are seen in the left counts and 19, 20, and 21 in the right counts. All of these more complex change in parastichy counts occur outside of the transition zone and during the fixed-disk-size regime of the simulation.  In summary a form of deterministic noise is introduced into the dynamics of the model by a the jump to new and non-lattice conditions  in the narrow transition zone; near to the critical $r'$ value below which Fibonacci transitions occur this can promote $F+1$ counts.


\subsection{Rapid transitions promote near-Fibonacci counts while noise promotes columnar structures}
This numerical observation of deterministic chaos arising only at rapid transitions offers one intriguing potential explanation for non-Fibonacci structure. In order to explore whether this has a detectably different signature to the presence of biological noise we further explore the effect of non-zero noise $\sigma$ in the model.   
Figure~\ref{fig:scpSch813} is an analogue of Figure~\ref{fig:scpDet813},
 with the indicated level of noise.
We can see a number of the same phenomena as in the deterministic runs: a narrow transition zone still acts to perturb the initial condition of the post transition fixed-disk-size dynamics, with the corresponding shift of the parastichy counts. So for example, the $r'=1$, $\sigma=0.02$ case shows a predominance of the 16 and 19 counts analogously to the deterministic model. The same underlying deterministic dynamics also creates paired $\gamma$ and $\lambda$ dislocations that generate small amounts of blurring of the peaks.
 
\pdffig{scpSch813}{Stochasticity flattens out parastichy count histograms, but also shifts them into the centre of the Fibonacci pair range. Single replicates of Figure~\ref{fig:scpDet813},  but with the addition of noise $\sigma=0.02$ or $\sigma=0.1$.} {.8}

The impact of noise is clearly seen in the increased number of dislocations in the parastichy lines. As specified in the methods section,  noise is implemented in such a way that disk positions are changed but disks are considered to remain in contact for drawing parastichy lines, so the contact lines are \textit{not} broken by this perturbation. Instead, additional dislocations can occur alongside those caused by the deterministic dynamics because the disk above the randomly-moved disk is in turn moved `out of position'. 

However there is an additional phenomenon which is not easily seen in the deterministic runs, but can be noticed in, for example the $r'=1$, $\sigma=0.1$ case. Looking at the gradient of the right-winding red parastichy lines after the transition zone, they share the gradient of the 21-parastichies seen in the less noisy case $\sigma=0.02$ and indeed correspond to parastichy counts of 21 in that region. However, starting some considerable time after the transition zone, there is a pattern shift and a series of unpaired $\lambda$ dislocations see a flattening of the gradient to about 45$^\circ$. The left-winding blue parastichies  also  become close to 45$^\circ$, although since they leave the transition zone already closer to 45$^\circ$ anyway the shift is less clear. 

What is happening in this new transition is a loss of asymmetry:  Fibonacci type structure requires that the one of the families of parastichy lines is much flatter than the other. In terms of disk placement this means that say, the disk above and to the left (say) of any given disk must be in general lower than the disk above and to the right of that disk. However with random initial conditions, or run stochasticity, there is no enforcement of this property, and there appears no theoretical reason why the dynamics of the disk-stacking model should tend to return the asymmetry. As a consequence, a generic run of the model from arbitrary initial conditions should be expected to yield parastichy lines with nearly equal angles, and correspondingly nearly equal parastichy counts. If, as in these simulations, the disk radius at the end of the simulation is calculated to allow a (13,21) pair, then this argument shows that two nearly equal parastichy counts adding to 34, in other words (17,18) are also likely patterns. Similarly, if the geometry is such that a (21,34) pair is possible then a generic pattern will be close to a (27,28) pair. These columnar solutions to the disk stacking model were identified by Gol\'e and colleagues~\cite{goleFibonacciQuasisymmetricPhyllotaxis2016} as QSS or `quasi-symmetric scenarios'  and an example is shown in Figure~\ref{fig:scpColumn}.
\pdffig{scpColumn}{Column-type patterns (and roughly equal parastichy pairs) emerge when noise is large enough to move away from Fibonacci structure. The resulting pattern contains patches in which the parastichy lines are each at $45^\circ$ and the disks are arranged in alternating vertical columns, which correspond to roughly equal parastichy numbers, here 16, 17 or 18. As Figure~\ref{fig:scpSch813} with $r'=0.1$ and $\sigma=0.1$, }{1}

\subsection{Double-Fibonacci and Lucas numbers emerge at large $r$ near the critical $r'$ and are promoted by noise}
The possibility of double-Fibonacci and Lucas numbers in disk-stacking models has already been demonstrated to occur near critical $r'$ values (e.g. Figure 15 of~\cite{goleFibonacciQuasisymmetricPhyllotaxis2016} or Figure 5 of ~\cite{yonekuraMathematicalModelStudies2019}.) However these results by themselves do not indicate at which point in pattern evolution these shifts from exact to more general Fibonacci structure occur. 

van Iterson theory shows a strong geometrical constraint on the parastichy numbers near lattice transitions. If the parastichy numbers before the bifurcation are $m$ and $n$ with $m<n$ then then the lattice geometry enforces that the higher parastichy number after bifurcation must be $m+n$; depending on the choice of branch in Figure~\ref{fig:scpVanIterson} the lower parastichy number is either $m$ or $n$, and only on the opposed branch is it $n$ so that the new pair is $n,n+m$. This branch choice is enough to enforce successive Fibonacci parastichy pairs from a starting pair of $(1,2)$, but it has long  been noticed that it also enforces successive Lucas pairs from a starting pair of $(1,3)$ and successive double-Fibonacci pairs from a starting pair of $(2,4)$, and that as we saw in the MOSI dataset such counts are frequent amongst non-Fibonacci observations. van Iterson lattice theory, though, cannot readily explain why and when these jumps which  ignore the normal branch choice occur.  If  it is possible to jump across the van Iterson tree once why not many times? In other words, suppose a Lucas pair $(11,18)$ is observed. Has that indeed evolved from an initial $(1,3)$ pattern? Or is it more likely for the jump to occur later so that, e.g., a $(8,13)$ pair directly transitions into an $(11,18)$. 
Disk stacking models support the hypothesis that only early transitions are likely: if either noise or too-fast transitions are strong enough effects to shift away from the left-right asymmetry of the  Fibonacci structure, the most likely resulting patterns are the columnar structures of the previous section, and these are close to Lucas or double-Fibonacci patterns only at the beginning of pattern formation. 


\subsection{Disk stacking models generate falling phyllotaxis and observed asymmetry on the capitulum}

The observational approach adopted in the 2016 analysis of the MOSI images was to visually identify spirals as far out on the capitulum as possible, and so for the primary analysis, Figure~\ref{fig:scpMOSIHistogram} it has been enough to consider only rising phyllotaxis, corresponding to decreasing $f$. 
However when analysing the MOSI images, we did call attention to a previously unreported departure from Fibonacci structure, which we there called a lack of rotational symmetry~\cite{swintonNovelFibonacciNonFibonacci2016}. Perhaps surprisingly, the stacked-disk model for falling phyllotaxis, with increasing $f$ is also capable of illustrating this type of pattern, as illustrated in Figure~\ref{fig:scpFalling}, which 
shows such a simulation, together with a mapping of the simulation back to the capitulum surface as described in the Methods section. 
For rising phyllotaxis, the transition zones where parastichy number change are very narrow and horizontal, even in the highly-disordered example of Figure~\ref{fig:scpColumn}. By contrast, the transition zone in this falling phyllotaxis example is not horizontal, but follows the gradient of the parastichy lines and so  there is a large region of the cylinder on which the parastichy counts are difficult and perhaps meaningless to assign, Looking at the lower right panels we can see how the uncountability manifests itself: there is a family of 21-ish red contact lines making one angle in the cylinder, and another family of 8 red-contact lines making another, but the two families coexist over an extended region of the cylinder. This is the pattern that was identified in sample 667 of the MOSI image, whose hand-assigned parastichy lines are reproduced in Figure~\ref{fig:667nophoto}.
 
\pdffig{scpFalling}{Stacked-disk models on the cylinder can generate radially asymmetric patterns by rising phyllotaxis to the capitulum rim followed by falling phyllotaxis on the capitulum. This simulation shows horizontal variability in the location on the cylinder of dislocations of the red right-ward leaning parastichy lines. When the cylinder is mapped to a disk, this corresponds to a visible rotational asymmetry on the simulated capitulum.    There is no satisfying way of assigning `spiral counts' on the outer rim for the red contact lines: compare Figure~\ref{fig:667nophoto}.}{1}

\jpgfig{667nophoto}{Observed contact line patterns in sample 667 of the MOSI dataset, with no satisfying way to assign a spiral count to the family of red/yellow/green spirals reflecting a rotational asymmetry as modelled in Figure~\ref{fig:scpFalling}.}{.5}


