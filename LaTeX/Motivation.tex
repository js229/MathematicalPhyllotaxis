

\chapter{Motivation and outline}
The Fibonacci series is
\[
0, 1, 1, 2, 3, 5,8, 13, 21, 34, 55, 89, 144, \dots
\] where each number after the second is the sum of the previous two. Phyllotaxis means the arrangement of the organs of plants, like leaves and seeds,  and Fibonacci phyllotaxis is a reference to the appearance of Fibonacci numbers in different kinds of counts of these organs. 
 \jpgfig{BletchleyDaisyCrop}{Thirteen green bracts on the underside of a Bletchley Park daisy \textit{Bellis perennis}, picked in 2019 from the lawn where Joan Clarke taught Alan Turing about phyllotaxis in the summer of 1941.}{0.5}

\section{Joan Clarke's daisy}

%
 For a few months in the summer of 1941, a young British couple were engaged to be married. Both were talented and overworked cryptographers working at Bletchley Park, the wartime codebreaking station, and  
 Joan Clarke and Alan Turing would spend their off-duty hours together, playing chess,  going for bicycle rides and visiting each other's parents. They broke the engagement off after a disastrous wet weekend in Wales, but when in the 1970s Turing's biographer Andrew Hodges interviewed Clarke, what she remembered distinctly about her long summer with her fianc\'e was lying on the lawn in front of Bletchley Park looking at the daisies. Turing and Clarke already knew about and shared an enthusiasm for the appearance of Fibonacci numbers in plants and though both were first-class mathematicians, it was Clarke who had to teach Turing about the botanical classifications of plant structure. So perhaps she  pulled up one of the summer daisies and explained that the distinct green `petals' on the underside of the flower should be called bracts, but either way, when they counted the bracts they will  most likely have found thirteen of them, as I too did in the summer of 2019 (Figure~\ref{fig:BletchleyDaisyCrop}).

\mmafig{Ch1Sunflower91}{
	Seeds packed onto the head of a sunflower. Three separate sets of spirals have been identified. For each set every tenth spiral is marked with a thick line.  To me the most visually obvious spiral set is the red one of which there are 55, curving anticlockwise as they move outwards, the next the set of 34 clockwise blue spirals, but the set of 89 green spirals is also prominent towards the outside and especially in the upper right hand region. This sunflower head was labelled as sample 91 in~\cite{swintonNovelFibonacciNonFibonacci2016}, and is typical of the more ordered arrangements described there.
}{.8}
Clarke went on to be a significant figure in post-war British cryptography, but it is Turing's legacy which is of course now very well known both inside and outside mathematics.
Undergraduates studying mathematical biology encounter his Turing instability as a mechanism for pattern formation, but what is less commonly taught is that one of the unpublished motivations for his published theory was to explain the appearance of Fibonacci numbers. Indeed, at his death he left behind a manuscript scrap, \textit{Outline of the Development of the Daisy}. Turing died tragically before he fully worked out his theory, although thirty years later other mathematicians, not aware of his work, would take more or less the same direction to reach the mathematics which we are now able to outline in this book. 


\section{The sunflower}

As part of the Turing birth centenary celebrations in 2012, the Museum of Science and Industry in Manchester invited members of the public across the globe to grow sunflowers, \textit{Helianthus annuus} and send them, or pictures of them, into the museum, as a citizen science experiment to determine the prevalence of Fibonacci numbers in the spirals of the sunflower head. Figure~\ref{fig:Ch1Sunflower91} shows one such submission, with different spirals visible to the eye, 
and coloured highlighting showing that these spirals can be organised in families with either 34, 55, or 89 members, at least around the rim of the sunflower, and that the subjective question of which family is most prominent has a different answer at different places in the sunflower.  This is a typical seedhead recorded in the experiment, in which a family of spirals with a Fibonacci count was seen in 74\% of the sample, so that deviations from Fibonacci structure were a common minority.

  The mature seedhead provides useful evidence about the relative position on the plant that the cell lineages leading to each seed occupied when they were making the decision to commit to develop into a seed at all. In particular the centre of the seedhead is newest in the sense that as we move into the centre the seeds are the ones which have arisen from the most recent decisions to commit and the pattern on the outside of the sunflower is laid down before that towards the centre.  In this sample there is a suggestion of a change from ${55,89}$ to ${34,55}$ in the pairs of most prominent parastichy counts over developmental time: this is called a falling phyllotaxis. 

Sunflower parastichy counts like 55 and 89, and occasionally 144 and very occasionally 233, are the largest Fibonacci numbers which can reliably be observed in any plant forms. A common Just So explanation for these observations is that they provide some form of selection advantage for the plant in allowing an `optimal' packing of the seedhead. But it is important to recognise that these large numbers are \textit{not} seen in population subject to natural selection, but in the giant forms of the sunflower, with seedheads 30cm or more across, which have been under intensive breeding pressure by humans over the ten thousand years or so that we have been relying on the sunflower as a food source. 
The evidence of the archaeological record is that both
 the sunflower seed, and the capitulum in which it sits, have grown larger over the development of agriculture, and that this has happened more than once in different human cultures~\cite{lentzSunflowerHelianthusAnnuus2008,burkeGeneticAnalysisSunflower2002}. It is likely that any selection pressure was for increased overall seed-mass, rather than any direct human desire for `optimal packing'. Nevertheless, more modest Fibonacci numbers do recur in other, less genetically modified species. 
\clearpage
\section{Fibonacci structure across the plant kingdom}
 \jpgfig{ChurchEuphorbia}{Four sucessive sections of the stem of a \textit{Euphorbia Wulfenii} in the Oxford Botanic Garden,
 	photographed by AH Church in 1904. From~\cite{churchRelationPhyllotaxisMechanical1904}.}{0.9}
 Daisies and sunflowers are close relatives within what modern taxonomy calls the \textit{Asteracaea} family,  and an older name for this family, the \textit{Compositae}, reflects the common structure of most of them, with many individual flowers packed together on a flat composite seedhead or capitulum. Other members of the family, such as the dahlia, can also exhibit pairs of Fibonacci counts.
However Fibonacci phyllotaxis is not at all a feature unique to this family.

Among flowering plants, Fibonacci patterns are also relatively common in the kinds of cacti sold as houseplants in the UK: they can sometimes be seen in, for example the prickly pear. Spurges (\textit{Euphorbiaceae}) are one genus of shrubs which are interesting in demonstrating rising phyllotaxis in a fairly clear way. One of the earliest systematic British studies of Fibonacci patterns was carried out by the Oxford botanist Arthur Harry Church, and self-published by him from 1904.  Figure~\ref{fig:ChurchEuphorbia} shows the stem of a spurge growing outside his office in the Oxford Botanic Garden.
%
Church has removed all of the former leaves, and in effect numbered each of their positions rising up the stem. Thus when he writes a 3 on the first specimen, he is implying that there are two other branch positions, which we can't see behind the stem. 
The lines drawn by Church in Figure~\ref{fig:ChurchEuphorbia}  are meant to suggest that the branch positions are organised in spirals making up families with, in one case, 13 and 21 members.  Church's book, though dated in places, remains the most valuable compendium of Fibonacci (and non-Fibonacci) structure in phyllotactic counts and is the basis for much of this chapter. For eample he also records that pairs of Fibonacci counts like 5 and 8 can also be seen in the leaf structure of arums (\textit{Arum}), teasels (\textit{Dipsacus}), and crassulas (\textit{Crassulacae}).
 \\\mbox{}
 \clearpage
 \section{The pineapple}
\mmafig{Txb0014LinfordPineapple}{Top: (a) whole pineapple and (b) flattened pineapple skin from~\cite{linfordFruitQualityStudies1933}; the black lines are natural markings on the scales. Bottom: as (b), but with the centre of each scale numbered in order of height in the photograph, and physically adjacent scales joined by lines.}{1.0}
A convenient example of Fibonacci patterning is often available to the supermarket consumer on the form of the pineapple.
\begin{exercise}
	\label{ex:doit}
	Show this.  It is \emph{not enough} to believe that you can find them if you look: you have to look. In the absence of a local grocer, consult Figure~\ref{fig:Txb0014LinfordPineapple}.
\end{exercise}
\begin{solution}
	The grocery proof is left as an exercise to the reader. If using the Linford pineapple, there are clearly 8 spirals going up and to the right; depending on how the scales as the edges were originally joined there are either 12 or 13 spirals up and to the left. 
\end{solution} 
Despite the easily visible presence of Fibonacci spirals in pineapples, Church's systematic 1904 survey of phyllotaxis includes no pineapples or any other representative of their family, the bromeliads. This may have been because at the time whole pineapples were very rarely grown in Europe except as a luxury item, and most pineapple consumed was transferred to Western consumers as a chopped and tinned item from colonial plantations. Indeed, it was not until 1933 that the observation of Fibonacci counts in pineapples made it into the scientific literature, when Linford, working on a commercial plantation, reported the occurrence of Fibonacci counts in a discussion of how to efficiently count the number of eyes~\cite{linfordFruitQualityStudies1933}. The advent of whole fruit transport in the 1950s, enabled  mathematical popularisation to mention the pineapple, notably in the work of Hal Coxeter, and with a suitable meaning for the word `natural',  the pineapple has come to rival the sunflower as an emblem of `mathematics in the natural world'.

 \clearpage
\section{The fir cone}

Fibonacci structure is not restricted to the flowering plants and can also be found in the conifers. 
Figure~\ref{fig:Fierz20151aCrop} shows a fir cone, one of six thousand collected from a black pine tree on the slopes above Lake Zurich in a remarkable experiment by Dr Veronika Fierz. As in the Museum of Science and Industry experiment, the main scientific motivation was to explore deviations from strict Fibonacci structure, but as with sunflowers the data also shows how prevalent Fibonacci structure was: all but a few hundred had eight spirals of the seed-protecting scales in one direction and thirteen in the other. 
%
\jpgfig{Fierz20151aCrop}{One of over 6000 pine cones of a single \textit{Pinus negra} studied by Fierz, of which 97\% had, like this specimen, eight spirals visible in in direction and thirteen in the other. The cone was attached to the tree through the central region, with the growing tip furthest away from the camera, so that the scales are numbered by youth, with scale 0 developing after the others. The thirteen spirals are outlined directly; that there \textit{are} eight in the other direction can be confirmed by picking out eg the sequence $2,10,18,26,34,\ldots$ and seeing we can do this eight times.  From~\cite{fierzAberrantPhyllotacticPatterns2015}. \copyright CC-BY 3.0 Fierz, 2015.
	}{1}
\clearpage
\begin{exercise}
	\label{ex:doitagain}
	Use Braun's 1831 drawings (Figure~\ref{fig:Braun1831}) of pine cones with the scales numbered to count the spirals they display.
\end{exercise}
\begin{solution}
Based on the consistent jumps between adjacent numbered scales (eg in Fig 1, from 22 to 26 to 40 to 34 is evidence of four separate spirals), I would say these cones display evidence of 4 and 7 spirals (Fig 1); 7 and 11 (Fig 2); 7 and 10 (Fig 4a). Based on the scale numbering alone, I can't assign an unambiguous spiral count  for Fig 3. 
\end{solution}

Fibonacci counts, typically but not always 3s, 5s and 8s, can be found in the fruit cones of other conifers, including spruces (\textit{Picea}), larches (\textit{Larix}),  cypresses (\textit{Cupressus}) \cite{fierzAberrantPhyllotacticPatterns2015} and the monkey-puzzle tree (\textit{Araucaria}) \cite{churchRelationPhyllotaxisMechanical1904}. 

\clearpage
\subsection{Monocots and dicots}
Much importance will attach in this book to considering the development of complexity from an initial simple pattern, so it is natural to pay attention to the initial pattern formed when the first embryonic leaf structures emerge from the seed. Most flowering plants are dicotyledons, meaning that these proto-leaves emerge as a pair, but a few are monocotyledons, in which 
only a single proto-leaf emerges. The pineapple family of bromeliads, for example are monocots. Although the distinction between monocots and dicots was once a major taxonomic classification, it is now believed that the ancestor of all flowering plants was a dicotyledon, and this symmetry was broken several different times in multiple cases of convergent evolution~\cite{ingrouillePlantsDiversityEvolution2006}. 
\jpgfig{Braun1831}{Spruce  \textit{Picea} cones from from Plate XXVII of~\cite{braunVergleichendeUntersuchungUber1831}; Fig 4(b) is the back view of Fig 4(a).}{1.0}


\section{Stem primordia}
The preceding examples of large Fibonacci numbers, detectable on a a macroscopic scale in the mature plant form, are more or less the only ones documented in the scientific literature, and so come from a very small fraction indeed of the `endless forms most beautiful' that Darwin spied in biological organisms, but it would be wrong to dismiss these cases as therefore uninformative on the processes generating plant form in general. Similar structures, albeit associated with much lower Fibonacci numbers, are very common indeed in the early embryonic forms of very many plant stems. 
\newpage

 \section{Summary}
 There are many accessible example of Fibonacci phyllotaxis: perhaps the most robustly repeatable observation for city-dwellers is to buy a pineapple. However for most species, including the pineapple,  large-scale scientific data is surprisingly sparse on the exact prevalence of different kinds of patterns, and especially on the ways in which patterns fail to be Fibonacci. An argument of this book is that it is when patterns fail to be Fibonacci that they are most informative for model testing, and we will return to this in the second half of the book. 
% \mmafig{Ch1Phylogeny}{A simplified phylogenetic tree of the vascular plants showing six of the species listed by Church~\cite{churchRelationPhyllotaxisMechanical1904} as displaying some kind of Fibonacci-related patterning. The flowering plants are thought to have diverged from the conifers about 300 million years ago (MYa)}{1}
\jEndChapter