

\documentclass[a4paper,10pt]{memoir}
\usepackage{ifthen}

\newboolean{printVersion}\setboolean{printVersion}{false}
\newboolean{answersAtEnd}\setboolean{answersAtEnd}{true}
\newboolean{notesAtEnd}\setboolean{notesAtEnd}{false}
\newboolean{compressBibliography}\setboolean{compressBibliography}{false}





\usepackage[british]{babel}\usepackage{csquotes}

% for main body text, might reset for bibliography
\settypeblocksize{*}{10cm}{1.6}
\setulmarginsandblock{2cm}{2cm}{*}
\setlrmarginsandblock{2cm}{15cm}{*}
\checkandfixthelayout
% \usepackage[showframe, pass]{geometry} 

\chapterstyle{dash}


\usepackage{amsmath} % \cfrac
\usepackage{amsfonts}
\usepackage{amsthm}  % \proof
\usepackage{amssymb} %R
\usepackage{overpic} % \overpic for cover page
\usepackage{textpos} % \textblock for cover page
\usepackage{gensymb}% \degree

\usepackage{mathtools} % intertext
\renewcommand{\implies}{\Rightarrow}

\DeclareMathOperator\sign{sign} % from amsmath

\usepackage{todonotes} % consider fixme instead - would probably be faster but may not work out of the box

\usepackage{fontspec}
\usepackage{CormorantGaramond} % by default using lining not oldstyle numnbers; seems to get those from Latin Modern...
\usepackage[scaled=0.9]{gillius2} % sets only sffamily; use \usepackage[sfdefault]{gillous} to set rm text too 

\usepackage{xcolor}
\definecolor{CylinderColour}{rgb}{0.86, 0.86, 0.64}
\definecolor{parastichy1}{rgb}{0.71, 0.3214, 0.27}
\definecolor{parastichy2}{rgb}{0.46, 0.2947, 0.591}
\definecolor{parastichy3}{rgb}{0.45, 0.54, 0.24}
\definecolor{parastichy4}{rgb}{0.95, 0.94,0.94}
\definecolor{parastichy5}{rgb}{0.21, 0.1875, 0.15}
\newcommand{\jHeadingColour}{\color{parastichy1}}
\newcommand{\jEmphasisColour}{\color{parastichy1}}
\newcommand{\jLinkColour}{\color{parastichy2}}
\newcommand{\jBodyTextColour}{\color{black}}

%\addtodef\chaptitlefont{}{\jHeadingColour}\renewcommand\colorchapnum{\color{ared}}

% chapter number and name
\addtodef\chapnumfont{}{\jHeadingColour}
\renewcommand*{\chaptitlefont}{\huge\bfseries\centering\jHeadingColour}


\setsecheadstyle{\jHeadingColour}% Set \section style
\setsubsecheadstyle{\jHeadingColour}% Set \section style
\setsecnumdepth{subsection}
\addtodef{\tocheadstart}{\color{black}}{} 
\addtoiargdef{\printtoctitle}{\color{black}}{} 


\setbeforesecskip{-\onelineskip}
\setaftersecskip{\onelineskip}

\newcommand{\mydraft}{NOT FOR PUBLICATION \jdraftnumber{}}
\renewcommand{\mydraft}{}

\newcommand{\jPage}{\jHeadingColour[\thepage]}
\makeoddfoot{plain}{\mydraft}{\jPage}{}
%\makeoddfoot{empty}{\null\vspace{2cm}\null\mydraft\jPage}{}
\makeoddfoot{empty}{}{}{}
\makeoddfoot{simple}{\mydraft}{\jPage}{simple}
\makeoddfoot{headings}{\mydraft}{\jPage}{}
\makeoddhead{headings}{}{\jHeadingColour\leftmark}{}
\makeevenhead{headings}{}{\large\jHeadingColour MATHEMATICAL PHYLLOTAXIS}{}
\makeevenfoot{headings}{}{\jPage}{}
%%%%%%%%epigraph
\setlength{\epigraphwidth}{6cm}

%%%%%%%%%%%%%%% bib 
\usepackage[backend=biber,style=ext-numeric,giveninits=true,
bibencoding=utf8,articlein=false]{biblatex}
\addbibresource{MPCite.bib} % generated by Zotero straight into this directory#
%\renewcommand*{\bibannotationprefix}{bibannotation-}

\renewbibmacro{finentry}{%
	\usebibmacro{annotation}%
		\setunit{%
		\finentrypunct
		\renewcommand*{\finentry}{}%
		\par}%
	\finentry
}

\renewbibmacro*{doi+eprint+url}{%
	\iftoggle{bbx:doi}{\printfield{doi}}{}\newunit\newblock
	\iftoggle{bbx:url}{\iffieldundef{doi}{\usebibmacro{url}}{}}{}
}

\DeclareOuterCiteDelims{cite}{\jLinkColour{\bibopenbracket}}{\jLinkColour{\bibclosebracket}}
\DeclareFieldFormat{doi}{%
	\mkbibacro{DOI}\addcolon\space
	\ifhyperref
	{\href{https://doi.org/#1}{	\jLinkColour\nolinkurl{#1}}}
	{\color{blue}\nolinkurl{#1}}}
\DeclareFieldFormat{url}{%
	\ifhyperref
	{\href{#1}{	\jLinkColour	\mkbibacro{URL}}}
	{\color{blue}\nolinkurl{#1}}}
\DeclareFieldFormat{isbn}{%
	\mkbibacro{ISBN}\addcolon\space
	\ifhyperref
		{\href{https://worldcat.org/isbn/#1}{\jLinkColour\nolinkurl{#1}}}
	{\color{blue}\nolinkurl{#1}}}

\renewbibmacro*{annotation}{\iffieldundef{annotation}{}{\printfield{annotation}}}

\ifthenelse{\boolean{printVersion}}{
\AtEveryBibitem{% 
	\clearfield{url}
	\clearfield{lang}
	\clearfield{urldate}
	\clearfield{issn}
	\clearfield{isbn}
	\clearfield{pmid}
	\clearfield{doi}
	\clearfield{visited}
}
}{% not printversion
\AtEveryBibitem{% 
	\clearfield{lang}
	\clearfield{urldate}
	\clearfield{issn}
	\clearfield{visited}
}
}
\DeclareFieldFormat[article, inbook, incollection, inproceedings, misc, thesis, unpublished]{title}{#1} % ie no quotes

\DeclareFieldFormat
[article,inbook,incollection,inproceedings,patent,thesis,unpublished]
{titlecase:title}{\MakeSentenceCase*{#1}}

\newcommand{\jBibliography}{
	\raggedright
	\ifthenelse{\boolean{compressBibliography}}{
		\makeoddhead{headings}{}{}{}
		\clearpage
		% set the margins wide;
		\setlrmarginsandblock{2cm}{2cm}{*}
		\setulmarginsandblock{2cm}{2cm}{*}
		\checkandfixthelayout
		% stack exchange says do this to make it happen
		% Magic happens here. Comment this to get non-symmetric margins
		\makeatletter
		\ch@ngetext
		\makeatletter
		\clearpage}{} % compressBibliography
	
	\printbibliography 
}


\usepackage{newunicodechar} % 
\newunicodechar{γ}{$\gamma$}
\newunicodechar{λ}{$\lambda$}
%%%%%%%%%%%%%%%

%%%%%%%%%%%%%%%%%%%%%%%%%%%%%%%%%%%%%%%


\newsubfloat{figure}% Create subfloat in figure environment
\newtheorem{theorem}{Theorem}
\newtheorem{definition}{Definition}

% biblatex manual recommends loading  hyperref after biblatex
% but it also it needs to come before exsheets
\usepackage{hyperref} 
\hypersetup{linkcolor=black,colorlinks= true,citecolor=parastichy2,urlcolor=parastichy2} % wont take a \color{} argument

%%%%%%%%%%%%%% examples 
\usepackage{exsheets}
\newenvironment{jExercise}{\question}{\endquestion}
\newenvironment{exercise}{\question}{\endquestion}
\newenvironment{jAnswer}{\solution}{\endsolution}
\SetupExSheets{counter-format=ch.qu}
\SetupExSheets{counter-within=ch}
\ifthenelse{\boolean{answersAtEnd}}{
	\newcommand{\jBookEndSolutions}{\chapter{Answers to Exercises}\printsolutions}
}
{
	\SetupExSheets{solution/print=true}
	\newcommand{\jBookEndSolutions}{}
}

%
%%%%%%%%%%%%%

\newcommand{\pvec}[1]{{\mathbf p}_{#1}}
\newcommand{\phatvec}[1]{\hat{\mathbf p}_{#1}}
\newcommand{\rpvec}[1]{\mathbf{r}_{#1}}
\newcommand{\Pvec}[1]{{\mathbf P}_{#1}}
\newcommand{\mvec}{\pvec{m}}
\newcommand{\nvec}{\pvec{n}}
\newcommand{\fvec}{\mathbf{f}}
\newcommand{\svec}{\mathbf{s}}
\newcommand{\tvec}{\mathbf{t}}

\newcommand{\jvec}[1]{\mathbf{#1}}

\newcommand{\jhalf}{\textstyle{\frac{1}{2}}}
\newcommand{\jfrac}[2]{\textstyle{\frac{#1}{#2}}}
\newcommand{\jpoint}[1]{\ell_{{#1}}}
\newcommand{\tends}{\rightarrow}


\newcommand{\gp}[1]{\textsf{(#1)}}
\newcommand{\gphat}[2]{\textsf{%
(\ensuremath{\textsf{#1}},\ensuremath{\hat{\textsf{#2}}})%
}}%
\newcommand{\gphatnothat}[3]{\textsf{%
(\ensuremath{\textsf{#1}},\ensuremath{\hat{\textsf{#2}}},\ensuremath{{\textsf{#3}}})%
}}%
\newcommand{\gpbug}[1]{(#1)}
 

\newcommand{\pp}[1]{\ppstyle{(#1)}}
\newcommand{\pphat}[2]{\textsf{
		(\ensuremath{\textsf{#1}},\ensuremath{\hat{\textsf{#2}}})
}}%
\newcommand{\ppstyle}[1]{\textsf{#1}}

\newcommand{\branch}[1]{\textsf{#1}}


\newcommand{\Mod}[1]{\ (\mathrm{mod}\ #1)}
\newcommand{\Sign}{\mathrm{sign}}

\newcommand{\jC}{\mathbb{C}}
\newcommand{\jR}{\mathbb{R}}
\newcommand{\jS}{\mathbb{S}}
\newcommand{\jN}{\mathbb{N}}
\newcommand{\jZ}{\mathbb{Z}}

\newcommand{\jq}{{\jHeadingColour q}}
\newcommand{\jqi}{{\jHeadingColour q_i}}
\newcommand{\jqn}[1]{{\jHeadingColour #1}}
\newcommand{\jqix}[1]{{\jHeadingColour q_{#1}}}

\newlength{\jfigwidth}
\setlength{\jfigwidth}{\textwidth}

\date{\today}

\usepackage{pgothic}
%\newcommand{\copynote}{{\pgothfamily C}}
%\newcommand{\placenote}{{\pgothfamily P}}
\newcommand{\copynote}{{ C}}
\newcommand{\placenote}{{ P}}




\newcommand{\nmafig}[2]{\vnmafig{#1}{#2}{1.0}}

\newcommand{\vnmafig}[3]{\jdofig{#1}{#2}{#3}{./Figures}{.pdf}}
\newcommand{\jpgfig}[3]{\jdofig{#1}{#2}{#3}{./Figures}{.jpg}}
\newcommand{\pdffig}[3]{\jdofig{#1}{#2}{#3}{./Figures}{.pdf}}

\newcommand{\jdofig}[5]{
	\begin{marginfigure}\centering\includegraphics[width=#3\jfigwidth]{#4/#1#5} \caption{#2}\label{fig:#1}\end{marginfigure}
}
%\newcommand{\jdofig}[5]{
%	\begin{figure}\centering\includegraphics[width=#3\jfigwidth]{#4/#1#5} \caption{#2}\label{fig:#1}\end{figure}
%}

\usepackage{wrapfig}
\newcommand{\jwrapfig}[4]{
\begin{wrapfigure}{r}{#3}
	\includegraphics[width=#3]{./Figures/#4.jpg}
	\caption{#2}
	\label{fig:#1}
\end{wrapfigure}
}
\newcommand{\mmawrapfig}[3]{
	\begin{wrapfigure}{r}{#3}
		\includegraphics[width=#3]{./Figures/#1.pdf}
		\caption{#2}
		\label{fig:#1}
	\end{wrapfigure}
}

\usepackage[most]{tcolorbox}
\newtcolorbox{tcbdoublebox}[1][]{%
	enhanced jigsaw,
	sharp corners,
	colback=CylinderColour,
	#1
}

\newcommand{\copypdffig}[4]{
	\begin{figure}
		\centering
\begin{tcbdoublebox}

This figure is not included in this draft because it is copyright. It can be found at #4. 
			 \caption{#2}\label{fig:#1}
			 	\end{tcbdoublebox}
	\end{figure}
}
\newcommand{\copyjpgfig}[4]{\copypdffig{#1}{#2}{#3}{#4}}
\newcommand{\clearedpdffig}[4]{\pdffig{#1}{#2}{#3}}
\newcommand{\clearedjpgfig}[4]{\jpgfig{#1}{#2}{#3}}

\ifthenelse{\boolean{showUnclearedCopyright}}{
\renewcommand{\copypdffig}[4]{\pdffig{#1}{#2}{#3}}
\renewcommand{\copyjpgfig}[4]{\jpgfig{#1}{#2}{#3}}
}
	

\newcommand\Tstrut{\rule{0pt}{2.6ex}}         % = `top' strut for spacing in tables

\usepackage{datetime}\shortdate

\ifthenelse{\boolean{notesAtEnd}}{
\renewcommand{\notedivision}{\chapter{Notes}}
\makepagenote
\continuousnotenums
\newcommand{\jNote}[1]{\pagenote{#1}}
% or at the end of the chapter..
%\newcommand{\jEndChapter}{\printpagenotes*} 
\newcommand{\jEndChapter}{}
}{ % not notesAtEnd
\newcommand{\jNote}[1]{\footnote{#1}}
\newcommand{\jEndChapter}{}
}


\newcommand{\jFarey}[4]{\left[\displaystyle\frac{#1}{#2},\frac{#3}{#4}\right]}


\usepackage{array}
\newcolumntype{L}[1]{>{\raggedright\let\newline\\\arraybackslash\hspace{0pt}}m{#1}}
\newcolumntype{C}[1]{>{\centering\let\newline\\\arraybackslash\hspace{0pt}}m{#1}}
\newcolumntype{R}[1]{>{\raggedleft\let\newline\\\arraybackslash\hspace{0pt}}m{#1}}



\begin{document}

\newcommand{\jPublicationYear}{2024}



\newlength{\titlepagespacing}
\ifthenelse{\boolean{a5Target}}{
\setlength{\titlepagespacing}{5ex}
}{
\setlength{\titlepagespacing}{10ex}
}

\ifthenelse{\boolean{a5Target}}{
\thispagestyle{titlingpage}
{
	\mbox{}\vfill
	\centering	
	\includegraphics[width=.6\jfigwidth]{./Figures/infang.jpg}
	\\[2 \titlepagespacing]
	Infang Publishing is a registered trademark of Deodands Ltd, 
	a company registered in England with company number 07589092 and
	a registered office at
	20-22 Wenlock Road, London N1 7GU.
	\\[10ex] %		These ISBNs will refer to the final version of the text, not this draft\\	
	{\textsc{ISBN}} 978-0-9931789-6-2 (softback).		\\		{\textsc{ISBN}} 978-0-9931789-7-9 (ebook)
	\\[\titlepagespacing]
	\copyright Deodands Ltd \jPublicationYear.
	\\[2\titlepagespacing]
	\textbf{Book website:}
	\\
	\texttt{www.phyllotaxis.uk}
	\\[\titlepagespacing]
	%		\textbf{This is pre-publication draft \jdraftnumber{} created on \today.}
	{Compiled  on \today\ from SHA}
	\\
	\texttt{\jGithubRepoSHA}
	\\[\titlepagespacing]
	Front cover image from Flickr user \texttt{aiko vanhulsen}  \copyright CC-BY 2.0.\\
}
\newpage

\thispagestyle{titlingpage}
{
	\centering
	\mbox{}
%	{\LARGE A textbook of }	\\
\\[	2 cm]
	{\HUGE\jEmphasisColour
		\textsc{Mathematical \\[5 mm] phyllotaxis}}
	\\
	\vfill
	{\Huge	\scshape Jonathan Swinton}
	\vfill
	{\LARGE	Infang Publishing}	\\
		{\LARGE \textsc{\jPublicationYear}\\}
}
\newpage


}
{ % not a5 target, a4 



\thispagestyle{titlingpage}
{
\mbox{}\vspace{\fill}
		\centering	
		\includegraphics[width=.6\jfigwidth]{./Figures/infang.jpg}
		\\[10ex]
		Infang Publishing is a registered trademark of Deodands Ltd.
		\\
		Email \texttt{info@deodands.co.uk}; website \texttt{deodands.co.uk}.
	\\[5ex]
		Deodands Ltd is a company registered in England with company number 07589092. 
%		\\
		Deodands Ltd has a registered office at
		20-22 Wenlock Road, London N1 7GU.
		\\[10ex] %		These ISBNs will refer to the final version of the text, not this draft\\	
		{\textsc{ISBN}} 978-0-9931789-6-2 (softback)
%		\\		{\textsc{ISBN}} 978-0-9931789-7-9 (ebook)
		\\[10ex]
		\copyright Deodands Ltd \jPublicationYear
		\\[20ex]
%	\textit{The publishers believe that Deodands Ltd believe every such image has been identified.   Deodands Ltd have  received specific license permission for each image via the Copyright Clearance Center to reproduce these images  on the basis that this text is intended to be published by Cambridge University Press. } 	\\[10ex]
		\textbf{Book website:}
		\\
		\texttt{www.phyllotaxis.uk}
		\\[10ex]
%		\textbf{This is pre-publication draft \jdraftnumber{} created on \today.}
		{Based on an author manuscript  \jdraftnumber{} released on \today.}
		\\[10ex]
The front cover image is taken from the Flickr stream of user \texttt{aiko vanhulsen} under a Creative Commons CC BY 2.0 license.\\
}
\newpage

\thispagestyle{titlingpage}
{
	\centering
	{\Huge A textbook of }
	\\[3ex]
	{\HUGE\jEmphasisColour	\textsc{Mathematical phyllotaxis}}\\
	\vfill
	{\Huge	\scshape Jonathan Swinton}
	%\\\vfill{\jEmphasisColour This is a DRAFT version \jdraftnumber{}}
	\vfill
	{\Large	Infang Publishing}
	\\[3ex]
	{\Large \textsc{\jPublicationYear}
		\\}
}
\newpage

10}



\preface


Over the past hundred years or more there has been a confusing variety of attempts to explain in mathematical language why Fibonacci numbers appear in plant spirals, but most modern accounts have now converged on a framework which might be called a Standard Picture. This book explains that mathematical framework and explores the extent to which it can explain Fibonacci phyllotaxis in the light of modern molecular biology.
 This Standard Picture relies on a biological model of how plants decide where to place individual new  organs over the course of development,
a model which when iterated tend to produce patterns we can model mathematically as lattices. Straight lines in these lattices are \textit{parastichies}\index{parastichy}, with an associated parastichy number counting how many of the parastichies make up the lattice, and there is a natural way to model what we do when we visually identify spirals as obvious that selects two of these parastichy numbers as the \textit{principal parastichy pair}.  This enables us to classify many model patterns by pairs of integers. Then we change a parameter of the model slowly, corresponding to some slow change -- almost always geometry -- in the underlying developmental process. This gives us a bifurcation tree for principal parastichy pairs as that parameter varies. The culmination of the Standard Picture is that, under quite general assumptions, Fibonacci-like structures form  stable branches of solutions.

It has been my goal in this book to equip readers to critique and participate in the applications of mathematical phyllotaxis.  I aim to give you an understanding of enough of the underlying biology to motivate this Standard Picture, to analyse it mathematically, and to evaluate it scientifically. Actually the main goal was to learn about the subject myself, but in writing the text I have imagined the needs of a recently graduated student of mathematics embarking on research in an interdisciplinary laboratory. (Since I don't fit any of these criteria, I would be particularly grateful for critique from those who do.) 

This is a textbook in which I have re-shaped and sometimes  re-derived a lot of other people's work into a story I find coherent and worth sharing. Douady et al.'s recent book~\cite{douadyPlantsKnowMath2024} provides a lively and accessible history of mathematical phyllotaxis showing how the subject has attracted a range of remarkable thinkers.  I have not felt obliged to provide a source for every statement or idea, but notes to each Chapter situate it in the literature for further reading. One exception is that I have highlighted the 
contributions made by Alan Turing (and indeed his fianc\'e), reflecting the route by which I myself became fascinated by the subject.

The first half of the book is an account of the mathematical theory of two-dimensional lattices. While lattice theory in higher dimensions has been an important area of mathematical research, the much simpler results we need in two dimensions are well-established, and I think intuitive. However my account is regrettably dense with Theorems and proofs. This was not my original aspiration, but evolved as I tried to make sense of multiple overlapping concepts, and occasional mathematical untruths, in the phyllotaxis literature.  So there is in places a density of mathematical argument of the pedantic kind I learned to loathe when a graduate student many decades ago. To my surprise, in the end I rather enjoyed writing these parts, but for those whose taste does not run this way they are largely skippable. I have put less technical summary sections at the ends of these chapters.  

Those whose taste \textit{does} run pure-mathematically might find the exposition painful to read in a different way. Pure mathematicians will recognise entirely conventional ideas about co-primality and winding-numbers, cylindrical lattices as the quotient of planar lattices by a periodicity group, and hyperbolic geometry. If you are one of these pre-privileged readers I urge you to persist with my approach even though it does not always follow the classic presentations. Phyllotaxis, I believe, can offer an under-appreciated motivating example for teaching,  because it traces a route from this pure mathematics to contemporary biological research issues. 


The second half of the book connects the mathematical principles of the first half to biological observation by reviewing a range of mathematical models. This part reviews some basic plant embryology and molecular biology and introduces the apical meristem as the site of developmental and positional commitment. I will discuss older ideas, like Hofmeister's hypothesis that primordia are generated as soon as they are beyond a fixed distance from all other primordia, together with the modern molecular understanding of  auxin-based mechanisms of primordia initiation. Then we can begin the applied mathematical work of evaluating how well our lattice models perform, and what they do and don't explain. In particular, we will see the mathematical and biological motivations for generalising lattice models to stacked-coin models. Finally we'll  review some of the outstanding mathematical questions, both pure and applied, and how they can inform the wider and continuing scientific debate on the generation of form in biology.

%Finally, to rephrase the advice buried above: skip all the mathematics you want, especially on a first reading. 

\vspace{\baselineskip}
\begin{flushright}\noindent
	Altrincham \& Cambridge, March 2025\hfill {\it Jonathan Swinton}\\
\end{flushright}






\begin{KeepFromToc}	\tableofcontents\end{KeepFromToc} 
\part{Introduction}


\chapter{Test chapter}


\label{CH:0}

\section{Fibonacci sequences}
\label{sec:fib}
\subsection{This is a subsection in chapter~\ref{CH:0}}
This is Chapter~\ref{CH:0} and {\jHeadingColour Chapter~\ref{CH:0}} and \arabic{chapter} and section \arabic{section} and page \arabic{page}
This is \Sref{sec:fib} in \Cref{CH:0}.

The Fibonacci sequence $F_n$ has as its first two members  $F_0=0$, $F_1=1$ and every subsequent member is the sum of the previous two:  $F_{n+2}=F_{n+1}+F_{n}$. 
Although there is a substantial literature on Fibonacci and related sequences~\cite{vajdaFibonacciLucasNumbers2008} we really only need this simple sum property for the Standard Picture. 
%
\begin{table}[h]
	\begin{center}
		\begin{tabular}{ll}
			\hline
			Fibonacci  &  $1,1,2,3,5,8,13,21,34,55,89,144,\ldots$ \Tstrut
			\\
			Double Fibonacci & $2,2,4,6,10,16,26,42,68,110,\ldots$
			\\
			\hline
		\end{tabular}
		\caption{Various sequences with Fibonacci structure}
		\label{tab:sequences}
	\end{center}
\end{table}
%

We might note that from Table~\ref{tab:sequences} that the Fibonacci, double Fibonacci and Lucas sequences together include all  of the first eleven integers except 9, so there is little remarkable about the observation that a particular system exhibits a structure including a low member of one of the sequences~\cite{cookeFibonacciNumbersReveal2006}. 



\subsection{The golden ratio}
Starting the sequence with $F_0=0$, $F_1=1$, the general Fibonacci term is\jNote{The golden angle was so-named around 1900, and supposedly given the greek letter $\Phi$ in reference to the Greek architect Phidias~\cite{cookCurvesLifeBeing1914}.}
\begin{eqnarray}
F_n &=& \frac{\tau^n - (1-\tau)^n}{\sqrt{5}}
\end{eqnarray}
where $\tau$ is the golden ratio%
\jNote{It is not uncommon to define the golden ratio as $\tau-1=0.618\ldots$ instead,  or to notate it as $\phi$ or sometimes $\omega$.  If a line  is cut at a fraction $\tau$, the fraction of the cut to the whole line is the same as that of the remainder to the cut.
	The first robust evidence of this concept in Greek mathematics, known as  `division in extreme and mean ratio' emerges somewhere between the Pythagoreans of  around 500 BCE and Euclid around  300 BCE~\cite{herz-fischlerMathematicalHistoryGolden1987}. The Fibonacci numbers 89 and 144 are first known to appear in connection with the golden ratio in a manuscript of the early 1500s, while the epithet 'golden' is first known in the 18th century~\cite{herz-fischlerEarlyUsageExpression2019,beckerEvenEarlier17172019}.%
}%
satisfying
\begin{eqnarray}
\tau^2 &=& \tau+1
\\
\tau &=& \frac{1+\sqrt{5}}{2} \approx 1.618
\\
&=& \lim_{n\rightarrow\infty} \frac{F_{n+1}}{F_n} 
\end{eqnarray}
Any sequence obeying the Fibonacci rule has $\tau$  as the limit of the ratio of its terms.




\section{Co-prime integers and the B\'ezout relation}

\label{sec:coprime}
Two integers $(m,n)$ are co-prime iff their greatest common divisor is equal to 1. We need to be explicit about some edge cases, by noticing that every integer is a divisor of $0$, but the only divisors of $1$ are $1$ and $-1$. Specifically we recognise both of the pairs $(0,1)$ and $(1,n)$ as co-prime, and note that $1$ is co-prime to itself, but $0$ is not, and $(0,n)$ is not co-prime for $n>1$.%
%\footnote{We don't require the integers to be non-negative for this co-primality to make sense, but we avoid situations where this matters.}

\subsection{B\'ezout relations}
The integer pair $(u,v)$ satisfies the B\'ezout relation for the non-negative integer pair $(m,n)$ iff
\begin{align}
	|n  u-mv| = 1.
\end{align}
If $u$ and $v$ exist then they are a proof of co-primality: we will see below that $(m,n)$ can satisfy  $ n  u - m v= k$ iff $|k|$ is the greatest common factor of $m$ and $n$. They are not unique because if $(u,v)$ satisfies the B\'ezout relation then so does $(u+km,v+kn)$ for any integer $k$. We can introduce a range condition by picking a particular $k$ which can be used to enforce $0\leq v< n$, but there is still a further ambiguity because if
\jNote{In other texts, sometimes $nu-mv= 1$ is required by the  B\'ezout eponym, and sometimes not. Only this book names as the  winding-number pair the unique pair defined in this range-restricted way. This is for reasons which will appear later in Chapter XX. }



\begin{jExercise}
	Compute the winding number pair for $(m,n)$ equal to  $(1,n)$, $(n,1)$, and $(F_j,F_{j+1})$.
\end{jExercise}
\begin{jAnswer} 
	\label{ex:wnp}
	Some winding number pairs are given in Table~\ref{tab:wnp2}.
	\begin{table}
		\begin{equation*}
			\begin{array}{lllll}
				\hline
				\text{} & 0\times 0-1\times (-1) & \text{} & \text{} & \text{} \\
				1\times 1-0\times 0 & 1\times 1-1\times 0 & 1\times 1-2\times 0 & 1\times 1-3\times 0 & 1\times 1-4\times 0 \\
				\text{} & 2\times 1-1\times 1 & \text{} & 2\times 2-3\times 1 & \text{} \\
				\text{} & 3\times 1-1\times 2 & 3\times 1-2\times 1 & \text{} & 3\times 3-4\times 2 \\
				\hline
			\end{array}
		\end{equation*}
		\caption{Winding number pairs $m\times v-n\times u=1$ for small co-prime integers $(m,n)$}
		\label{tab:wnp2}
	\end{table}
	
\end{jAnswer}

For the integers in this book, with $n\lesssim 500$ it is perfectly feasible to compute highest common factors and winding-number pairs by exhaustive search. But nevertheless the rest of this Chapter shows how an algorithmic approach sheds a light on the structure of co-prime pairs and their winding-numbers in a way that has been helpful in the past to mathematicians puzzling over Fibonacci structure. 

\begin{theorem}
	Euclid's algorithm terminates with the greatest common factor $GCF(m,n)$:
	\begin{eqnarray}
		r_N =  
		GCF(m,n) 
	\end{eqnarray}
\end{theorem}
\begin{proof}
	The $r_i$ are a strictly decreasing sequence of positive integers and so the algorithm always terminates. Suppose the $GCF$ is $k$. Now $k$ divides $r_0$, $k|r_0$, and the $i$-th step the iteration preserves the fact that  $k|r_i$,  and so in particular $k|r_N$. Now $r_N|r_{N-1}$ and the iteration step also shows that if $r_N|r_i$ and $r_N|r_{i-1}$ then $r_N|r_{i-2}$, and so following the $r_i$s in reverse order we see $r_N$ divides all of them and divides both $m$ and $n$.  So $r_N$ is a common divisor and so $r_N\leq k$ but since $k|r_N$, $r_N=k$. 
\end{proof}
As an example, consider calculating the highest common factor of $4$ and $11$:
\begin{align}
	11 - {\jHeadingColour 2}\times 4 & = 3 
	\\
	4 - {\jHeadingColour 1}\times 3 & = 1\label{eq:411}
	\\
	3 - {\jHeadingColour 3}\times 1 & = 0 
\end{align}
Thus the particular sequence of $\jqi={\jHeadingColour 2, 1, 3}$ shows the co-primality of 4 and 11, and it is possible to use this decomposition to compute the winding-number pair. To see this we rewrite the algorithm in matrix form. 

\subsection{Matrix form of the Euclidean algorithm}
\vnmafig{Ch2EuclideanTree}{Computing the highest common factor of 4 and 11 by succesive matrix reductions. The first column of each matrix contains the successive co-prime pairs through the reduction. The branch choice is defined as we traverse the tree upwards, giving the $\jqi$s as the number of $E$s between each $S$; then the B\'ezout relation $11\times1-3\times 4$ is computed by following the branch downwards from the identity matrix}{1}
%
See Figure~\ref{fig:Ch2EuclideanTree}. 
We can solve the B\'ezout relation by putting  Euclid's algorithm in matrix form. For example
the first reduction of~\eqref{eq:411} is
\begin{align*}
\begin{pmatrix} 11 \\ 4 \end{pmatrix}  &= \begin{pmatrix} 1 & \jqn{2} \\ 0 & 1\end{pmatrix} \begin{pmatrix} 3 \\ 4\end{pmatrix} 
\\
&
= \begin{pmatrix} 1 &1 \\ 0 & 1\end{pmatrix}^{\jqn{2}}  \begin{pmatrix} 0 & 1  \\ 1 & 0  \end{pmatrix} \begin{pmatrix} 4 \\ 3 \end{pmatrix}
\\
&=
E^{\jqn{2}} S \begin{pmatrix} 4 \\ 3 \end{pmatrix}
\end{align*}
where we have defined matrices corresponding to the successive Euclidean reductions $E$ and the swap of the integer pair $S$, making use of the fact that $E^\jq$ has a $q$ in the upper-right corner:

It's possible, but unhelpful to think of a M\"obius map $f:\jC\rightarrow\jC$ as a  map $\jR^2\rightarrow\jR^2$.  The reason this is unhelpful is that $f$ is not a linear map on these vectors, and $f(z)$ is not the vector computed as the matrix product of $P_f$ and the vector $(\Re{z},\Im{z})$. A 2x2 matrix more normally represents such a linear transformation which allows rotation, scaling and shear; here the same number of parameters represent a transformation allowing rotation, scaling and translation. 

It's possible, but unhelpful to think of a M\"obius map $f:\jC\rightarrow\jC$ as a  map $\jR^2\rightarrow\jR^2$.  The reason this is unhelpful is that $f$ is not a linear map on these vectors, and $f(z)$ is not the vector computed as the matrix product of $P_f$ and the vector $(\Re{z},\Im{z})$. A 2x2 matrix more normally represents such a linear transformation which allows rotation, scaling and shear; here the same number of parameters represent a transformation allowing rotation, scaling and translation. 
It's possible, but unhelpful to think of a M\"obius map $f:\jC\rightarrow\jC$ as a  map $\jR^2\rightarrow\jR^2$.  The reason this is unhelpful is that $f$ is not a linear map on these vectors, and $f(z)$ is not the vector computed as the matrix product of $P_f$ and the vector $(\Re{z},\Im{z})$. A 2x2 matrix more normally represents such a linear transformation which allows rotation, scaling and shear; here the same number of parameters represent a transformation allowing rotation, scaling and translation. 
It's possible, but unhelpful to think of a M\"obius map $f:\jC\rightarrow\jC$ as a  map $\jR^2\rightarrow\jR^2$.  The reason this is unhelpful is that $f$ is not a linear map on these vectors, and $f(z)$ is not the vector computed as the matrix product of $P_f$ and the vector $(\Re{z},\Im{z})$. A 2x2 matrix more normally represents such a linear transformation which allows rotation, scaling and shear; here the same number of parameters represent a transformation allowing rotation, scaling and translation. 
It's possible, but unhelpful to think of a M\"obius map $f:\jC\rightarrow\jC$ as a  map $\jR^2\rightarrow\jR^2$.  The reason this is unhelpful is that $f$ is not a linear map on these vectors, and $f(z)$ is not the vector computed as the matrix product of $P_f$ and the vector $(\Re{z},\Im{z})$. A 2x2 matrix more normally represents such a linear transformation which allows rotation, scaling and shear; here the same number of parameters represent a transformation allowing rotation, scaling and translation. 
It's possible, but unhelpful to think of a M\"obius map $f:\jC\rightarrow\jC$ as a  map $\jR^2\rightarrow\jR^2$.  The reason this is unhelpful is that $f$ is not a linear map on these vectors, and $f(z)$ is not the vector computed as the matrix product of $P_f$ and the vector $(\Re{z},\Im{z})$. A 2x2 matrix more normally represents such a linear transformation which allows rotation, scaling and shear; here the same number of parameters represent a transformation allowing rotation, scaling and translation. 

It's possible, but unhelpful to think of a M\"obius map $f:\jC\rightarrow\jC$ as a  map $\jR^2\rightarrow\jR^2$.  The reason this is unhelpful is that $f$ is not a linear map on these vectors, and $f(z)$ is not the vector computed as the matrix product of $P_f$ and the vector $(\Re{z},\Im{z})$. A 2x2 matrix more normally represents such a linear transformation which allows rotation, scaling and shear; here the same number of parameters represent a transformation allowing rotation, scaling and translation. 
It's possible, but unhelpful to think of a M\"obius map $f:\jC\rightarrow\jC$ as a  map $\jR^2\rightarrow\jR^2$.  The reason this is unhelpful is that $f$ is not a linear map on these vectors, and $f(z)$ is not the vector computed as the matrix product of $P_f$ and the vector $(\Re{z},\Im{z})$. A 2x2 matrix more normally represents such a linear transformation which allows rotation, scaling and shear; here the same number of parameters represent a transformation allowing rotation, scaling and translation. 
It's possible, but unhelpful to think of a M\"obius map $f:\jC\rightarrow\jC$ as a  map $\jR^2\rightarrow\jR^2$.  The reason this is unhelpful is that $f$ is not a linear map on these vectors, and $f(z)$ is not the vector computed as the matrix product of $P_f$ and the vector $(\Re{z},\Im{z})$. A 2x2 matrix more normally represents such a linear transformation which allows rotation, scaling and shear; here the same number of parameters represent a transformation allowing rotation, scaling and translation. 
It's possible, but unhelpful to think of a M\"obius map $f:\jC\rightarrow\jC$ as a  map $\jR^2\rightarrow\jR^2$.  The reason this is unhelpful is that $f$ is not a linear map on these vectors, and $f(z)$ is not the vector computed as the matrix product of $P_f$ and the vector $(\Re{z},\Im{z})$. A 2x2 matrix more normally represents such a linear transformation which allows rotation, scaling and shear; here the same number of parameters represent a transformation allowing rotation, scaling and translation. 
It's possible, but unhelpful to think of a M\"obius map $f:\jC\rightarrow\jC$ as a  map $\jR^2\rightarrow\jR^2$.  The reason this is unhelpful is that $f$ is not a linear map on these vectors, and $f(z)$ is not the vector computed as the matrix product of $P_f$ and the vector $(\Re{z},\Im{z})$. A 2x2 matrix more normally represents such a linear transformation which allows rotation, scaling and shear; here the same number of parameters represent a transformation allowing rotation, scaling and translation. 
It's possible, but unhelpful to think of a M\"obius map $f:\jC\rightarrow\jC$ as a  map $\jR^2\rightarrow\jR^2$.  The reason this is unhelpful is that $f$ is not a linear map on these vectors, and $f(z)$ is not the vector computed as the matrix product of $P_f$ and the vector $(\Re{z},\Im{z})$. A 2x2 matrix more normally represents such a linear transformation which allows rotation, scaling and shear; here the same number of parameters represent a transformation allowing rotation, scaling and translation. 
It's possible, but unhelpful to think of a M\"obius map $f:\jC\rightarrow\jC$ as a  map $\jR^2\rightarrow\jR^2$.  The reason this is unhelpful is that $f$ is not a linear map on these vectors, and $f(z)$ is not the vector computed as the matrix product of $P_f$ and the vector $(\Re{z},\Im{z})$. A 2x2 matrix more normally represents such a linear transformation which allows rotation, scaling and shear; here the same number of parameters represent a transformation allowing rotation, scaling and translation. 


\cite{swintonNovelFibonacciNonFibonacci2016}
\cite{theophrastusEnquiryPlantsBook1916}
\cite{thompsonGrowthForm1917}
\cite{traasPhyllotaxis2013}
\cite{cookCurvesLife1914}
\chapter*{Acknowledgements}
Mark and Ang Davis extended an invitation to give a talk in the pub for Bollington Science Festival in 2012 which started me wondering why it wasn't possible to explain to the citizens of Bollington why plants had Fibonacci numbers, even though Ian Stewart had written twenty years earlier that this was a mathematically solved problem. Around the same time, Erinma Ochu, then of the MOSI Turing's Sunflowers project was amazed that no-one knew how to breed Lucas sunflowers. 

\jBookEndSolutions % will be empty if in-text solutions
\ifthenelse{\boolean{notesAtEnd}}{\printpagenotes}{}

\jBibliography

\abstract*{Acknowledgements}

\extrachap{Acknowledgements}

In 2012, Mark and Ang Davis asked me to give a talk in the pub for Bollington Science Festival which started me wondering why it wasn't possible to explain why plants had Fibonacci numbers, even though I'd believed twenty years earlier that this was a mathematically solved problem. Around the same time, Erinma Ochu of the MOSI Turing's Sunflowers project was amazed not only that no-one knew how to breed Lucas sunflowers, but that I was surprised by the question.  
Julia Gog  made some usefully mathematical remarks on a very early draft of the lattice theory chapters, and  Paul Glendinning  long ago gave some advice on how to phrase my challenge to the Fundamental Theorem of Phyllotaxis.
Christopher Golé told me about his work with Stéphane Douady on the stability of rhombic lattices with three disks. Stéphane, together with  Annemiek Cornelissen, made me welcome on a visit to the CNRS Laboratoire Matière et Systèmes Complexes. Tamsin Spelman and several of her colleagues at the Sainsbury Laboratory in Cambridge made helpful comments, and Hugo Tavares and Katie Abley's neighbourly support was particularly nourishing.   Philip Maini graciously hosted me in the Mathematical Biology seminar at the University of Oxford. and Andrew Krause, Mark Muldoon and Ronjoy Adhikari listened patiently and constructively to my pitches for this material. Phil Ramsden currently holds the record for the fastest finding of a mistake.  Adam Swinton has been a patient and critical reader of several drafts. 
Diana Gillooly and David Tranah, as editors at two major university Presses,  offered supportive, engaged and productive publication advice over a rather extended period. But in the end it was Alain Goriely who on his own initiative invited me to join the Springer Texts series, and I thank him and the Editorial Board for support for this book. Christopher Golé, Stéphane Douady and Richard Schwartz gave constructive advice as reviewers for Springer. 
%I expect to be able to write that Donna Chernyk shepherded this through the press at Springer Nature with friendly efficiency.


Thinking about this subject allowed me to rediscover a delight in mathematics. But love of mathematics alone would never have got this book finished over its many years in the writing. For rediscovering the love of almost everything else, including getting things done, I thank in particular Emma Anderson. But it was the questions  of \textit{all} these people and more, by email, in the pub, the museum and the seminar room, that shaped this attempt at answers.  Thanks to you all. 




\end{document}



