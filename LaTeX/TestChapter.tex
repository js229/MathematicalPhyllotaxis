

\chapter{Test chapter}


\label{CH:0}

\section{Fibonacci sequences}
\label{sec:fib}
\subsection{This is a subsection in chapter~\ref{CH:0}}
This is Chapter~\ref{CH:0} and {\jHeadingColour Chapter~\ref{CH:0}} and \arabic{chapter} and section \arabic{section} and page \arabic{page}
This is \Sref{sec:fib} in \Cref{CH:0}.

The Fibonacci sequence $F_n$ has as its first two members  $F_0=0$, $F_1=1$ and every subsequent member is the sum of the previous two:  $F_{n+2}=F_{n+1}+F_{n}$. 
Although there is a substantial literature on Fibonacci and related sequences~\cite{vajdaFibonacciLucasNumbers2008} we really only need this simple sum property for the Standard Picture. 
%
\begin{table}[h]
	\begin{center}
		\begin{tabular}{ll}
			\hline
			Fibonacci  &  $1,1,2,3,5,8,13,21,34,55,89,144,\ldots$ \Tstrut
			\\
			Double Fibonacci & $2,2,4,6,10,16,26,42,68,110,\ldots$
			\\
			\hline
		\end{tabular}
		\caption{Various sequences with Fibonacci structure}
		\label{tab:sequences}
	\end{center}
\end{table}
%

We might note that from Table~\ref{tab:sequences} that the Fibonacci, double Fibonacci and Lucas sequences together include all  of the first eleven integers except 9, so there is little remarkable about the observation that a particular system exhibits a structure including a low member of one of the sequences~\cite{cookeFibonacciNumbersReveal2006}. 



\subsection{The golden ratio}
Starting the sequence with $F_0=0$, $F_1=1$, the general Fibonacci term is\jNote{The golden angle was so-named around 1900, and supposedly given the greek letter $\Phi$ in reference to the Greek architect Phidias~\cite{cookCurvesLifeBeing1914}.}
\begin{eqnarray}
F_n &=& \frac{\tau^n - (1-\tau)^n}{\sqrt{5}}
\end{eqnarray}
where $\tau$ is the golden ratio%
\jNote{It is not uncommon to define the golden ratio as $\tau-1=0.618\ldots$ instead,  or to notate it as $\phi$ or sometimes $\omega$.  If a line  is cut at a fraction $\tau$, the fraction of the cut to the whole line is the same as that of the remainder to the cut.
	The first robust evidence of this concept in Greek mathematics, known as  `division in extreme and mean ratio' emerges somewhere between the Pythagoreans of  around 500 BCE and Euclid around  300 BCE~\cite{herz-fischlerMathematicalHistoryGolden1987}. The Fibonacci numbers 89 and 144 are first known to appear in connection with the golden ratio in a manuscript of the early 1500s, while the epithet 'golden' is first known in the 18th century~\cite{herz-fischlerEarlyUsageExpression2019,beckerEvenEarlier17172019}.%
}%
satisfying
\begin{eqnarray}
\tau^2 &=& \tau+1
\\
\tau &=& \frac{1+\sqrt{5}}{2} \approx 1.618
\\
&=& \lim_{n\rightarrow\infty} \frac{F_{n+1}}{F_n} 
\end{eqnarray}
Any sequence obeying the Fibonacci rule has $\tau$  as the limit of the ratio of its terms.




\section{Co-prime integers and the B\'ezout relation}

\label{sec:coprime}
Two integers $(m,n)$ are co-prime iff their greatest common divisor is equal to 1. We need to be explicit about some edge cases, by noticing that every integer is a divisor of $0$, but the only divisors of $1$ are $1$ and $-1$. Specifically we recognise both of the pairs $(0,1)$ and $(1,n)$ as co-prime, and note that $1$ is co-prime to itself, but $0$ is not, and $(0,n)$ is not co-prime for $n>1$.%
%\footnote{We don't require the integers to be non-negative for this co-primality to make sense, but we avoid situations where this matters.}

\subsection{B\'ezout relations}
The integer pair $(u,v)$ satisfies the B\'ezout relation for the non-negative integer pair $(m,n)$ iff
\begin{align}
	|n  u-mv| = 1.
\end{align}
If $u$ and $v$ exist then they are a proof of co-primality: we will see below that $(m,n)$ can satisfy  $ n  u - m v= k$ iff $|k|$ is the greatest common factor of $m$ and $n$. They are not unique because if $(u,v)$ satisfies the B\'ezout relation then so does $(u+km,v+kn)$ for any integer $k$. We can introduce a range condition by picking a particular $k$ which can be used to enforce $0\leq v< n$, but there is still a further ambiguity because if
\jNote{In other texts, sometimes $nu-mv= 1$ is required by the  B\'ezout eponym, and sometimes not. Only this book names as the  winding-number pair the unique pair defined in this range-restricted way. This is for reasons which will appear later in Chapter XX. }



\begin{jExercise}
	Compute the winding number pair for $(m,n)$ equal to  $(1,n)$, $(n,1)$, and $(F_j,F_{j+1})$.
\end{jExercise}
\begin{jAnswer} 
	\label{ex:wnp}
	Some winding number pairs are given in Table~\ref{tab:wnp2}.
	\begin{table}
		\begin{equation*}
			\begin{array}{lllll}
				\hline
				\text{} & 0\times 0-1\times (-1) & \text{} & \text{} & \text{} \\
				1\times 1-0\times 0 & 1\times 1-1\times 0 & 1\times 1-2\times 0 & 1\times 1-3\times 0 & 1\times 1-4\times 0 \\
				\text{} & 2\times 1-1\times 1 & \text{} & 2\times 2-3\times 1 & \text{} \\
				\text{} & 3\times 1-1\times 2 & 3\times 1-2\times 1 & \text{} & 3\times 3-4\times 2 \\
				\hline
			\end{array}
		\end{equation*}
		\caption{Winding number pairs $m\times v-n\times u=1$ for small co-prime integers $(m,n)$}
		\label{tab:wnp2}
	\end{table}
	
\end{jAnswer}

For the integers in this book, with $n\lesssim 500$ it is perfectly feasible to compute highest common factors and winding-number pairs by exhaustive search. But nevertheless the rest of this Chapter shows how an algorithmic approach sheds a light on the structure of co-prime pairs and their winding-numbers in a way that has been helpful in the past to mathematicians puzzling over Fibonacci structure. 

\begin{theorem}
	Euclid's algorithm terminates with the greatest common factor $GCF(m,n)$:
	\begin{eqnarray}
		r_N =  
		GCF(m,n) 
	\end{eqnarray}
\end{theorem}
\begin{proof}
	The $r_i$ are a strictly decreasing sequence of positive integers and so the algorithm always terminates. Suppose the $GCF$ is $k$. Now $k$ divides $r_0$, $k|r_0$, and the $i$-th step the iteration preserves the fact that  $k|r_i$,  and so in particular $k|r_N$. Now $r_N|r_{N-1}$ and the iteration step also shows that if $r_N|r_i$ and $r_N|r_{i-1}$ then $r_N|r_{i-2}$, and so following the $r_i$s in reverse order we see $r_N$ divides all of them and divides both $m$ and $n$.  So $r_N$ is a common divisor and so $r_N\leq k$ but since $k|r_N$, $r_N=k$. 
\end{proof}
As an example, consider calculating the highest common factor of $4$ and $11$:
\begin{align}
	11 - {\jHeadingColour 2}\times 4 & = 3 
	\\
	4 - {\jHeadingColour 1}\times 3 & = 1\label{eq:411}
	\\
	3 - {\jHeadingColour 3}\times 1 & = 0 
\end{align}
Thus the particular sequence of $\jqi={\jHeadingColour 2, 1, 3}$ shows the co-primality of 4 and 11, and it is possible to use this decomposition to compute the winding-number pair. To see this we rewrite the algorithm in matrix form. 

\subsection{Matrix form of the Euclidean algorithm}
\vnmafig{Ch2EuclideanTree}{Computing the highest common factor of 4 and 11 by succesive matrix reductions. The first column of each matrix contains the successive co-prime pairs through the reduction. The branch choice is defined as we traverse the tree upwards, giving the $\jqi$s as the number of $E$s between each $S$; then the B\'ezout relation $11\times1-3\times 4$ is computed by following the branch downwards from the identity matrix}{1}
%
See Figure~\ref{fig:Ch2EuclideanTree}. 
We can solve the B\'ezout relation by putting  Euclid's algorithm in matrix form. For example
the first reduction of~\eqref{eq:411} is
\begin{align*}
\begin{pmatrix} 11 \\ 4 \end{pmatrix}  &= \begin{pmatrix} 1 & \jqn{2} \\ 0 & 1\end{pmatrix} \begin{pmatrix} 3 \\ 4\end{pmatrix} 
\\
&
= \begin{pmatrix} 1 &1 \\ 0 & 1\end{pmatrix}^{\jqn{2}}  \begin{pmatrix} 0 & 1  \\ 1 & 0  \end{pmatrix} \begin{pmatrix} 4 \\ 3 \end{pmatrix}
\\
&=
E^{\jqn{2}} S \begin{pmatrix} 4 \\ 3 \end{pmatrix}
\end{align*}
where we have defined matrices corresponding to the successive Euclidean reductions $E$ and the swap of the integer pair $S$, making use of the fact that $E^\jq$ has a $q$ in the upper-right corner:

It's possible, but unhelpful to think of a M\"obius map $f:\jC\rightarrow\jC$ as a  map $\jR^2\rightarrow\jR^2$.  The reason this is unhelpful is that $f$ is not a linear map on these vectors, and $f(z)$ is not the vector computed as the matrix product of $P_f$ and the vector $(\Re{z},\Im{z})$. A 2x2 matrix more normally represents such a linear transformation which allows rotation, scaling and shear; here the same number of parameters represent a transformation allowing rotation, scaling and translation. 

It's possible, but unhelpful to think of a M\"obius map $f:\jC\rightarrow\jC$ as a  map $\jR^2\rightarrow\jR^2$.  The reason this is unhelpful is that $f$ is not a linear map on these vectors, and $f(z)$ is not the vector computed as the matrix product of $P_f$ and the vector $(\Re{z},\Im{z})$. A 2x2 matrix more normally represents such a linear transformation which allows rotation, scaling and shear; here the same number of parameters represent a transformation allowing rotation, scaling and translation. 
It's possible, but unhelpful to think of a M\"obius map $f:\jC\rightarrow\jC$ as a  map $\jR^2\rightarrow\jR^2$.  The reason this is unhelpful is that $f$ is not a linear map on these vectors, and $f(z)$ is not the vector computed as the matrix product of $P_f$ and the vector $(\Re{z},\Im{z})$. A 2x2 matrix more normally represents such a linear transformation which allows rotation, scaling and shear; here the same number of parameters represent a transformation allowing rotation, scaling and translation. 
It's possible, but unhelpful to think of a M\"obius map $f:\jC\rightarrow\jC$ as a  map $\jR^2\rightarrow\jR^2$.  The reason this is unhelpful is that $f$ is not a linear map on these vectors, and $f(z)$ is not the vector computed as the matrix product of $P_f$ and the vector $(\Re{z},\Im{z})$. A 2x2 matrix more normally represents such a linear transformation which allows rotation, scaling and shear; here the same number of parameters represent a transformation allowing rotation, scaling and translation. 
It's possible, but unhelpful to think of a M\"obius map $f:\jC\rightarrow\jC$ as a  map $\jR^2\rightarrow\jR^2$.  The reason this is unhelpful is that $f$ is not a linear map on these vectors, and $f(z)$ is not the vector computed as the matrix product of $P_f$ and the vector $(\Re{z},\Im{z})$. A 2x2 matrix more normally represents such a linear transformation which allows rotation, scaling and shear; here the same number of parameters represent a transformation allowing rotation, scaling and translation. 
It's possible, but unhelpful to think of a M\"obius map $f:\jC\rightarrow\jC$ as a  map $\jR^2\rightarrow\jR^2$.  The reason this is unhelpful is that $f$ is not a linear map on these vectors, and $f(z)$ is not the vector computed as the matrix product of $P_f$ and the vector $(\Re{z},\Im{z})$. A 2x2 matrix more normally represents such a linear transformation which allows rotation, scaling and shear; here the same number of parameters represent a transformation allowing rotation, scaling and translation. 

It's possible, but unhelpful to think of a M\"obius map $f:\jC\rightarrow\jC$ as a  map $\jR^2\rightarrow\jR^2$.  The reason this is unhelpful is that $f$ is not a linear map on these vectors, and $f(z)$ is not the vector computed as the matrix product of $P_f$ and the vector $(\Re{z},\Im{z})$. A 2x2 matrix more normally represents such a linear transformation which allows rotation, scaling and shear; here the same number of parameters represent a transformation allowing rotation, scaling and translation. 
It's possible, but unhelpful to think of a M\"obius map $f:\jC\rightarrow\jC$ as a  map $\jR^2\rightarrow\jR^2$.  The reason this is unhelpful is that $f$ is not a linear map on these vectors, and $f(z)$ is not the vector computed as the matrix product of $P_f$ and the vector $(\Re{z},\Im{z})$. A 2x2 matrix more normally represents such a linear transformation which allows rotation, scaling and shear; here the same number of parameters represent a transformation allowing rotation, scaling and translation. 
It's possible, but unhelpful to think of a M\"obius map $f:\jC\rightarrow\jC$ as a  map $\jR^2\rightarrow\jR^2$.  The reason this is unhelpful is that $f$ is not a linear map on these vectors, and $f(z)$ is not the vector computed as the matrix product of $P_f$ and the vector $(\Re{z},\Im{z})$. A 2x2 matrix more normally represents such a linear transformation which allows rotation, scaling and shear; here the same number of parameters represent a transformation allowing rotation, scaling and translation. 
It's possible, but unhelpful to think of a M\"obius map $f:\jC\rightarrow\jC$ as a  map $\jR^2\rightarrow\jR^2$.  The reason this is unhelpful is that $f$ is not a linear map on these vectors, and $f(z)$ is not the vector computed as the matrix product of $P_f$ and the vector $(\Re{z},\Im{z})$. A 2x2 matrix more normally represents such a linear transformation which allows rotation, scaling and shear; here the same number of parameters represent a transformation allowing rotation, scaling and translation. 
It's possible, but unhelpful to think of a M\"obius map $f:\jC\rightarrow\jC$ as a  map $\jR^2\rightarrow\jR^2$.  The reason this is unhelpful is that $f$ is not a linear map on these vectors, and $f(z)$ is not the vector computed as the matrix product of $P_f$ and the vector $(\Re{z},\Im{z})$. A 2x2 matrix more normally represents such a linear transformation which allows rotation, scaling and shear; here the same number of parameters represent a transformation allowing rotation, scaling and translation. 
It's possible, but unhelpful to think of a M\"obius map $f:\jC\rightarrow\jC$ as a  map $\jR^2\rightarrow\jR^2$.  The reason this is unhelpful is that $f$ is not a linear map on these vectors, and $f(z)$ is not the vector computed as the matrix product of $P_f$ and the vector $(\Re{z},\Im{z})$. A 2x2 matrix more normally represents such a linear transformation which allows rotation, scaling and shear; here the same number of parameters represent a transformation allowing rotation, scaling and translation. 
It's possible, but unhelpful to think of a M\"obius map $f:\jC\rightarrow\jC$ as a  map $\jR^2\rightarrow\jR^2$.  The reason this is unhelpful is that $f$ is not a linear map on these vectors, and $f(z)$ is not the vector computed as the matrix product of $P_f$ and the vector $(\Re{z},\Im{z})$. A 2x2 matrix more normally represents such a linear transformation which allows rotation, scaling and shear; here the same number of parameters represent a transformation allowing rotation, scaling and translation. 

