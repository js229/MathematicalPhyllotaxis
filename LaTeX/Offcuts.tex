% !TeX program = lualatex
\documentclass{memoir}
\usepackage[backend=biber,style=ext-numeric,giveninits=true,bibencoding=utf8,articlein=false]{biblatex}
\addbibresource{MPCite.bib} % generated by Zotero straight into this directory#

\begin{document}
\chapter{Offcuts}

\section{Fibonacci numbers in historic mathematical cultures}
One of the reasons why the appearance of Fibonacci numbers in plant form is so compelling to some is because of their long history in mathematics. 
In the third and second centuries BCE, the Sanskrit scholar Acharya Piṅgala recorded an algorithm for generating poetic metre in which the Fibonacci sequence could be said to be implicit, and 
one modern scholar suggests this step was first explicitly taken by Virahāṅka in the seventh century CE~\autocite{singhSocalledFibonacciNumbers1985,velankarVrttajatisamuccayaKaviVirahanka1962}.  Writers in this algorithmic tradition over more than a millenium continued to note properties of the resulting syllabic groupings or \textit{mātrā-vṛtta}, most explicitly around 1150 when the scholar Hemacandra observed `the sum of the last and the one before the last is the number ... of the next \textit{mātrā-vṛtta}'\autocite{singhSocalledFibonacciNumbers1985}. The simplicity of the Fibonacci rule makes it quite possible that other sophisticated mathematical cultures knew of the sequence:  it has been speculated that the \textit{yupana}, the abacus-like calculating device of  Inca culture, may have encoded addition as operations on Fibonacci sums~\cite{leonardIncanAbacusCurious2010}.

Beginning at around same time, the golden ratio $\tau$ can be found in the ancient Greek concept of `extreme and mean ratio' recorded in Euclid's \textit{Elements}~\autocite{herz-fischlerMathematicalHistoryDivision1987}. 
A connection between the golden ratio and Fibonacci numbers was made in passing by Simon Jacob's 1564 \textit{Ein New und Wolgegründt Rechenbuch}~\cite{schreiberSupplementShallitsPaper1995} but not rediscovered until the nineteenth century when the sequence was well known under the name of the Lamé sequence~\autocite{lucasTheorieNombresPremiers1876}. It was Édouard Lucas who renamed the sequence as the Fibonacci sequence in the 1870s, and who in turn gave own his name to the $F^3$ sequence~\autocite{lucasTheorieNombresPremiers1876}. For this reason the `Lucas' in `Lucas numbers' should be pronounced in the same way as the French surname.  The golden angle acquired the symbol $\Phi$, from the ancient Greek sculptor Phidias, at the beginning of the twentieth century~\autocite{barrParametersBeauty1929}.

Lucas's reason for choosing the name Fibonacci for the sequence was based on its appearance in a famous early modern arithmetic textbook, the \textit{Liber abbaci} of 1202, compiled by the scholar known in his lifetime as Leonardo of Pisa, and in the nineteenth century as Fibonacci~\cite{siglerFibonaccisLiberAbaci2002}. This manuscript book was one of the first to introduce long-standing Islamic traditions of computation into Christian European accountancy.  The Fibonacci numbers appear as a somewhat strained example of rabbit population dynamics intended to motivate a computational exercise~\cite{hoyrupFibonacciProtagonistWitness2014}: Fibonacci did not intend the example to be taken seriously as mathematical biology.  While it is plausible that the exercise itself also derived from the Arabic algorithmic tradition, no surviving Arabic text containing such an example has been reported.

Lucas carefully researched  traces of the Fibonacci-golden ratio pattern in the European mathematical literature. These started with Albert Girard, a Flemish mathematician who noted in 1625 that 
\begin{quotation}
	Since I have entered into the subject of rational numbers, I will add two or three more observations on how  to approximate radicals extremely closely, by certain well suited numbers without greatly increasing the number of digirs. For example, in order to approximate by rationals the ratios of the segments of the line cut in the mean and extreme rato, this would make the progression 0, 1, 1, 2, 3, 5, 8, 13, 21, &c. of which each number is equal to the two preceding ones, then two numbers taken immediately will have the same ratio,  like 5 to 8 or 8 to 13 & c. & so much bigger, so much closer, like these two 59475986 & 96234155, so that 13, 13, 21 constitute quite precisely an Isosceles triangle having the angle of the pentagon%
%\footnote{From~\cite{stevinLarithmetiqueSimonStevin1625}; as Lucas notes the two large numbers, which were later repeated uncritically through several editions, are wrong.}
	\end{quotation}
\printbibliography
\end{document}


