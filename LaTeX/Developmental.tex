\chapter[Developmental biology of the plant]{Developmental biology of the plant stem}
\label{ch:developmental}
\abstract{This chapter provides a mathematician's guide to the developmental biology of the plant stem.
	Plants have flexible but tightly controlled developmental mechanisms for generating new organs in the larger spaces left by previous ones and this generates pattern. Many of the molecular and genetic control processes are quite well understood, particularly in the model organism \textit{Arabidopsis}. This will allow us, in Chapter~\ref{ch:placement}, to recognise the basis of a number of different primordium-placement models. 
 }
\label{sec:bio}
\section{Stem extension and thickening}
The first plants arrived on land around 470 million years ago \cite{ingrouillePlantsDiversityEvolution2006,harrisonOriginEarlyEvolution2018}. They were leafless, without stems, and reproduced through spore distribution. The plant stem evolved relatively quickly around 40 million years later, probably driven by the selective advantage of distributing spores over a wider area.
The molecular mechanism that allowed stem production seems to have evolved only once, but was so useful that it soon after was associated with a explosive radiation of species. There is some suggestion that very early plants did possess some form of regularity in their architecture, and by the time of the first angiosperms there was a central and strongly conserved role for auxin in the patterning mechanism~\cite{reinhardtLawOrderPlants2022}. 

The stem growth mechanism in vascular plants is now well understood at a genetic level. Broadly speaking, control over cellular division and commitment is exerted by the shoot apical meristem (SAM), a relatively small region of cells in a bulge over the top of the stem itself (Figure~\ref{fig:Txb0701SAMb1}). During stem growth, cells at the lower rim of the SAM and in the outer two or three layers of the stem are prompted to differentiate (Figure~\ref{fig:Txb0702SAMb2}), lifting the `roof' of the SAM, and causing the more internal cells, labelled as rib meristem in Figure~\ref{fig:Txb0701SAMb1} to proliferate so as to resolve the resulting mechanical tension. It is near the border of this continuously regenerating rib meristem region and the central organising zone that pluripotent cells are maintained and from time to time commit to primordia formation. While a variety of models of this process with detailed representations of cellular geometry and signalling are now emerging,  it is not the aim of this book to compare such models.\todo{cite models} Instead we want to show how their common geometric features influence pattern development. 

\mmafig{Txb0701SAMb1}{Cross-section through an idealised shoot apex, with a central control region, a more peripheral apical meristem which contains regions of cells that are capable of committing to primordium formation, and an incipient primordium in the process of developing into specific organ such as a leaf. Redrawn and modified from~\cite{fuchsAimingTopNoncell2020}.}{1.0}
%
\mmafig{Txb0702SAMb2}{The same shoot apex as Figure~\ref{fig:Txb0701SAMb1}. The outermost two cell layers, usually known as L1 and L2, preserve a distinct 
	structure and function over development. In particular cells within L1 and L2 derive from anticlinal (ie perpendicular to the surface)  cell division within the layer, associated with extending the stem. By contrast cells within L3 arise from both anticlinal division within L3 and periclinal division from L2 or L3 into L3 which contribute to girth increase. In \textit{Arabidopsis}, primordium initiation occurs from cells in L1.}
{1.0}





\section{Developmental commitment of primordia}


As the stem extends, the plant has to decide when and where to commit some of the newly created cells, just at the lower rim of the SAM, to become new organs: we call these newly committed cells \textit{primordia}, and the position of the corresponding organ in the mature plant \textit{nodes}. 

It is representing this initial commitment decision as a primordia placement function which is the central task of model-building for mathematical phyllotaxy. We will assume radial symmetry, so that the process takes place in a region at the top of a cylinder which is extending vertically and perhaps thickening radially, and the position of the $p$-th primordium can be written as $\ell_p=(x,z)$ where $x$ represents a circumferential distance around the cylinder and $z$ the height: the notation of Chapter~\ref{ch:cylinder} is consistent with this for the special case when the $\ell_p$ make up a lattice with a regular rise and divergence. 

It is difficult, even today, to make repeated non-destructive observations of the developing SAM. A crucial assumption made in this book is that angular position of nodes in the mature plant reflects angular position at the point of commitment. Though stems can acquire systematic helical twist during growth, there appears to be little or no evidence that angular rotation about the growth axis occurs locally: once placed,  nodes do not rotate relative to their neighbours. Similarly, while considerable  extension occurs during the vegetative growth of the stem, this is uniform around the stem cylinder, so that a node created earlier in development than its neighbours typically remains relatively lower on the stem.  In the case of a composite seedhead seen from above, outer seeds were placed by the plant before inner ones were. 
These two assumptions together allow data on the temporal pattern of primordia in the plant embryonic region to be inferred from the macroscopic spatial form of the nodes in the mature plant.
Figure~\ref{fig:F0703vanItersonFig42} shows how a transverse section across the growing bud at around the level of the SAM can reveal the relative divergence angles of each of the successive primordia. In this figure, the oldest primordia are those with the highest numbers. The shoots from each primordium are growing in the same direction as the main stem so that they encircle and enclose the SAM. 
\jpgfig{F0703vanItersonFig42}{ Transverse and longitudinal section of a growing bud of the bridewort \emph{Spirae salicifolia} from van Iterson~\cite{vanitersonjrMathematischeUndMikroscopischAnatomische1907}.}{0.4}
 
With the advent of higher quality microscopy, enabling inspection of the SAM and of primordium formation,  Hofmeister developed around 1860 the rule  that primordia are formed regularly in time and placed at the least crowded spot on the meristem. Much later, in the 1940s, came \emph{Snows' rule}, that primordia form as soon as they can but only when and where there is enough space for them.%
\jNote{The apostrophe here marks this as a concept attributed jointly to both Mary Snow and George Snow.}
 These rules can be tested by ablation experiments, removing previously created primordia and observing how this modifies the primordium placement decision. These experiments, over the course of a century, provide support to the principle that it is only the set of most recently formed primordia that determine the positions of the next~\autocite{snowExperimentsPhyllotaxisII1933,reinhardtRegulationPhyllotaxisPolar2003} and that this is because they are the most adjacent. 
 
 One parameter of geometric interest is the girth of the SAM.  It is common for the diameter of the SAM to broaden during the plastochrone, that is when new primordia are not being formed, and then to narrow again at the times of primordium formation, to create an oscillatory diameter as a function of time~\autocite{cronkMolecularOrganographyPlants2009}, but I have not been able to find many quantitative studies on how the diameter of the SAM varies over longer course of plant growth. There is some evidence that, in eg, sunflowers, the SAM does significantly increase in size as the plant reaches the stage when it is committing cells to form the outer edge of the composite seedhead, and then contracts again  corresponding to development of the central regions of the head~\cite{palmerPhysiologicalBasisPattern1998}. Alternatively, the SAM might remain fairly constant in size over the life of the plant, and any stem girth thickening happens later, in older,  lower, portions of the stem. It is likely that both processes combine to yield  macroscopic changes in stem girth. 
 

\section{Cellular architecture of the shoot-apical meristem}
Figure~\ref{fig:F0704Shi2019Fig1} shows the cellular structure of the SAM in \textit{Arabidopsis}.  In simpler plants, there may be only a single cell in the control zone, but in more complex plants like \textit{Arabidopsis} the SAM is multicellular; the dimension of the overall SAM can vary widely from 50$\mu$m to 3mm in the case of palms and large cacti. The relatively small numbers of cells poses a modelling choice: whether to treat  plant tissue as a continuous substrate for biochemical or mechanical dynamics or whether to explicitly recognise and model cellular architecture. As technology for the latter has advanced, generations of finite-element type models have become increasingly powerful. Identifying accurate cellular architectures is a significant problem for these models, and they have so far been limited to the highly characterised development of \textit{Arabidopsis}. 
 
\jpgfig{F0704Shi2019Fig1}{Cellular structure at the SAM of \textit{Arabidopsis}. $P_1$ to $P_{10}$ label recently generated new primordia, with $P_{10}$ the oldest, illustrating a $(3,5)$ parastichy where eg $P_1$ is in contact with $P_4$ and $P_6$. Note that the circumference of the SAM, which is roughly indicated in green, is of the order of 40 cells and each cell is very roughly 5~$\mu m$ at its widest. The transect view below shows two single cell layers $L_1$ and $L_2$, which in higher plants will form the epidermis and subepidermis of the mature stem and which divide only `anticlinally', that is down the axis of the stem in a lengthening manner, while the region $L_3$ will develop into vascular tissue and can divide in all directions. Recent evidence is that the mechanics of auxin flux differs between these layers. \copyright Elsevier, from~\autocite{shiPatterningShootApical2019}.}{.6}

\section{Molecular phyllotaxis}

A few gene mutations which cause phenotypic changes in phyllotaxis have been identified, but they have proved hard to interpret to understand primordium placement~\cite{kuhlemeierPhyllotaxis2017}. Instead,  understanding of phyllotaxis has instead largely come from spatial imaging studies of tagged proteins and  reporter genes. The processes that control the size of the SAM region and the primordium-placement function are increasingly well characterised at this molecular level, at least in the model organism \textit{Arabidopsis thaliana}~\cite{galvan-ampudiaTemporalIntegrationAuxin2020,cronkMolecularOrganographyPlants2009}.

 A central role of auxin in self-organised patterns has long been suspected.  Auxin plays many other roles in the plant, notably in shoot tropism, but its significance for primordium development was confirmed at a molecular level by Reinhardt and his colleagues~\autocite{reinhardtRegulationPhyllotaxisPolar2003} and this, together with a modelling component was by 2006 forming a `plausible model of phyllotaxis'~\cite{smithPlausibleModelPhyllotaxis2006}. Recently a consensus has emerged about many though not all of the details of how auxin drives stem node formation~\autocite{shiPatterningShootApical2019}.

	 Broadly, pattern formation results from local suppression of the auxin response in the SAM, a suppression controlled by the mutual inhibition between the \textit{KNOX1} and \textit{ARP} gene families. \textit{KNOX1} is widely expressed in the SAM, but primordium commitment requires \textit{ARP} expression. Auxin transport is controlled at the cell wall by proteins from the \textit{PIN1} and \textit{AUX1} gene families.  Low levels of auxin within the interior of the SAM promote SAM expansion, and that these auxin levels depends more on auxin transport below the epidermal layer~\autocite{shiFeedbackLateralOrgans2018} and may lead to more complicated auxin flux patterns, which can be different in the different tissue layers. Low levels of auxin within the interior of the SAM promote SAM expansion, and these auxin levels depend more on auxin transport in L3 than in L1 and L2~\autocite{shiFeedbackLateralOrgans2018,shiPatterningShootApical2019}.


Some recent work~\autocite{galvan-ampudiaTemporalIntegrationAuxin2020} has shown that PIN polarity directs auxin flows to both the centre of the SAM and also the most recently created primordia. These authors also suggested that the third to fifth most recent primordia acted as auxin production centres, and suggested a time-dependence in the  development of auxin levels after primordium commitment through mechanisms which are as yet unclear. 

Early flow models assumed that cells respond to auxin flux by increasing their transport capacity in the flux direction. Although these `with-the-flux' based models are still believed to work well to explain  sub-epidermal protein patterning, they have been challenged as explanations of pattern at the epidermis.  If instead the cell concentrates its PIN near the cell membrane of the one of its neighbours that already has the highest auxin level, then it is possible to 
have an `up-the-gradient' transport. The relative importance of these two mechanisms is  not  clear: as van Berkel et al~\autocite{berkelPolarAuxinTransport2013} concluded, 'all current models explain \textit{in planta} auxin and PIN patterning to this same limited extent'. A related point of recent controversy and relevance to modelling is what contributes to observed decreases in auxin concentration within new primordia. Conceptual models from the 2010s suggested that auxin was actively degraded but more recent experimental data has been used to argue that it is transport that removes high auxin concentrations from these areas.

Empirical data at this cell-membrane level poses a particular challenge  to current imaging technologies, and while there is active biological and computational research in this area much remains unknown. Another active researcher, Cris Kuhlemeier,  argued in 2017 that the coexistence of these `competing theories \ldots  for the underlying molecular mechanism emphasize the central fact that we don't really know how the patterning is driven'~\autocite{kuhlemeierPhyllotaxis2017}. 


Another important hormone is cytokinin, long known to have a role in the maintenance of the SAM itself. It has been interpreted as providing noise-reduction to phyllotaxis, through a proposed mechanism in which a mutual inhibition of cytokinin and \textit{AHP6} oscillates between successive primordia in such a way as to stop the simultaneous formation of primordia at different divergence angles at the same time~\cite{besnardCytokininSignallingInhibitory2014}. Both the effect of auxin on developing nodes and the way in which they in turn affect auxin flux through, for example, vein development are increasingly  be understood in three dimensions~\autocite{debPhyllotaxisInvolvesAuxin2015}.
 
The existence of reliable molecular probes for auxin concentration and PIN transport will create considerable new opportunities for pattern modellers. Although the tools used need further validation,  Figure~\ref{fig:F0705G-A2018FigS5crop} shows how it may now be possible in principle to quantify auxin transcription around the outer edge of the SAM. Quantification of concentration levels is a harder problem but one likely to resolve in future. %
\pdffig{F0705G-A2018FigS5crop}{Levels of auxin transcription, measured with DII-VENUS, at radius 35$\mu$m in a developing \textit{Arabidopsis} phyllotaxy.  Green arrows show the estimated polarity of PIN1-controlled auxin efflux from cells. Copyright CC-BY-NC-ND the authors, from~\autocite{galvan-ampudiaTemporalIntegrationAuxin2020}.}{1}

One of the difficulties with molecular techniques is their relatively high species specificity, so that it is hard to reproduce the very wide range of species observations of the earlier botanists, but there seems no reason to doubt that we have the fundamentals of an understanding of the origin of Snows' and Hofmeister's rules. By contrast there is there no known molecular basis  for any mechanism that enforces a particular divergence angle between successive primordia irrespective of the existing primordium pattern. While plants have sophisticated molecular timing systems used to synchronise with light variation over the course of a day, and transcriptional  delays between primordia commitment and formation probably play a significant role~\autocite{galvan-ampudiaTemporalIntegrationAuxin2020}
 there is also no evidence that the SAM possesses the molecular technology for a pacemaker system which could `drop' new primordia at fixed times. 

\section{Mechanical stress}

 Plant cells are attached to their neighbours by relatively rigid cell walls, and are subject to both hydrostatic pressure and mechanical strain. Cells have long been known to respond biochemically to these triggers, since primordium commitment causes the meristem to bulge this could be a way in which the primordium-placement function is coupled to the pre-existing pattern. 
 One early study was by Wardlaw~\autocite{wardlawPhyllotaxisOrganogenesisFerns1949}, based on a long series of experiments in ferns.  The visual analogy between the patterns of phyllotaxis and the buckling modes of a stressed rigid plate was noted in the 1990s by Green~\cite{greenPhyllotacticPatternsCharacterization1987}.  
 
  Patterning through stress dynamics is a potent mechanism for form generation,  but the coupling between this and auxin dynamics is unclear.  It might be that stress provides the local feedback between auxin and PIN localisation \autocite{heislerAlignmentPIN1Polarity2010},
   or that mechanical stresses do not originate but can subsequently stabilise patterning and  subsequent organogenesis~\autocite{kuhlemeierPhyllotaxis2017}.  Although models have been developed combining both stress and molecular dynamics~\autocite{pennybackerPhyllotaxisProgressStory2015} that exhibit Fibonacci patterning, they have as yet been inconclusive as to whether modelling can contribute to  these questions.
 
\section{Vascular bundles and leaf traces}
Simple primordium formation models assume a cylindrical symmetry to the stem. However all seed plants develop vascular bundles, each a discrete strand combining an outer phloem tube and an inner xylem tube, and with a number of bundles arranged fairly regularly in a ring to form what is known as a eustele. Typically bundles are interrupted at each primordium; these anatomical issues which directly affect auxin flux are rarely taken account of in current models. While it's unlikely these developing vascularisations affect the original primordia placement choice, they do certainly depend on it. Because each new organ must be connected to one of these vascular bundles, their branching structure provides an alternative record of the pattern formation process. 
% deleted because Wiley want £130 for it
%Figure~\ref{fig:Jensen1968F14} shows one example.
%\copyjpgfig{Jensen1968F14}{Portion of a stem of the succulent \textit{Kalanchoe tubiflora}, with vascular bundles from which leaf traces split off. The node pattern transitions from a bijugate $\gp{2,2}$ to a $\gp{2,3}$ pattern.   \copyright Jensen 1968 from~\cite{jensenPrimaryStemVascular1968}.
%	}{.5}{\cite{jensenPrimaryStemVascular1968}}
Modellers have not as yet made much use of these further potential sources of data, and  there has been little modelling of vascular morphogenesis itself from this perspective.
	


\section{The relationship to the Standard Picture}
\label{sec:jugacy}
The bifurcation-theoretic approach of the first half of this book naturally focuses our attention on the very earliest stages of primordia formation as the embryonic stem emerges from the germinated seed.  The first leaf-like structures on the emergent stem are called cotyledons and are distinguished in the botanical literature from the subsequent `true' leaves; they have related but at least partly independent functions and developmental pathways. In particular, the cotyledon structure develops \textbf{before} the shoot meristem itself and through distinct mechanisms which bear little relationship to the developmentally later meristem pattern described above~\autocite{yoshidaGeneticControlPlant2014}. This cotyledon structure could though potentially provide a $180^\circ$ polarity cue to the later SAM. At present we know  little about SAM during the crucial initiating period after the formation of the cotyledons but before the establishment of patterns with the relative complexity of say Figure~\ref{fig:F0704Shi2019Fig1}. This is one of the key empirical areas which would need to be resolved before we could convincingly say we had a tested model of Fibonacci phyllotaxis. 

The mathematically natural way in which Lucas numbers emerge from the same patterning dynamics as Fibonacci ones, save for a change of initial condition, poses a direct biological question: when does this happen change embryologically?  Is this jump in the van Iterson tree more likely to be accomplished during very early development or later, during a substantial non-linear expansion of SAM geometry? Similar issues arise when thinking about transitions to multijugate patterns. And transitions to these whorled patterns pose further mathematical and empirical challenges. We saw in Section~\ref{sec:Jlattice} that it is possible to represent whorled patterns as multijugate lattices in which the symmetry is almost accidentally emergent when the  rise is zero and the divergence passes through $1/2J$, but it is very likely that in practice there are other symmetry-enforcing mechanisms at work. 



\section{Development of the capitulum of the sunflower}
Although almost all of our molecular understanding of phyllotaxis comes from \textit{Arabidopsis} and a handful of other model plants, there is a great deal of morphological data that is worth interpreting in the light of the Standard Picture. 
A naive interpretation of this Standard Picture is that in the sunflower, say, we should observe simple spiral patterns in the leaf placement on the stem, with typically lower order Fibonacci numbers as parastichy counts, and these patterns increase in complexity by transitioning through ever higher Fibonacci numbers, traceable in parastichies through leaf, bract and floret primordia. How does this fit with the observed form of the mature plant? This varies considerably by cultivar, and by environmental influence.
But it is common in fact to see paired leaves at the lowest levels, and then a symmetry breaking higher up the stem. Although in principle consistent with the Standard Picture, there is no substantial dataset on parastichy counts on the stem, or even whether spiral parastichies can routinely be detected, but the divergence is typically between $2/5$ and $5/13$~\autocite{churchRelationPhyllotaxisMechanical1904}.  

When the stem develops into the seed head, or capitulum, there is a substantial broadening of its diameter. On the underside of this disk,  there are typically no mature organs, and then bracts are closely placed around the rim of the disk. This combination of an absence of recent mature organs and an increase in SAM diameter is entirely consistent with recent findings in \textit{Arabidopsis}~\cite{shiFeedbackLateralOrgans2018}. It is a clear -- but untested -- prediction of the Standard Picture that, although no mature primordia survive in this intermediate range, at least during development there could be  auxin patterns of intermediate Fibonacci structure observed,  and moreover it is possible these would be reflected in the surviving vasculature. Moving to the bract region,  there are few observations on parastichy counts or countability of these bract placements, but there are a few observations of the total number of bracts occurring which do cluster around Fibonacci numbers (Majumder~\cite{majumderVariationNumberRay1976}). Finally the stem surface folds back on itself so that the floret placement becomes on a near horizontal surface.\jNote{A surprisingly common misconception amongst mathematicians is that the pattern has been laid down from the centre of the capitulum to the outside. Of course the outer ring of the disk is developmentally the \textit{oldest}, and floret placement proceeds from the outside of the disk to the inside.}
	
When the sunflower is two to three weeks old there is an increase in the cell division rate in the SAM, which causes the apex to swell and within 3 more days to produce a dome. This domed SAM generates a series of primordia which will become side leaves or branches.%
Within a further ten days or so the dome flattens and broadens, and primordia which will develop into the green bracts become visible around the outside.  Figure~\ref{fig:F0706Palmer1998} shows these two stages:  note how the diameter of the dome at about 20 days is already several millimetres, and that this planar disk expands to a diameter of about 5-6mm, at the point when floret primordia start to appear.

\pdffig{F0706Palmer1998}{Sunflower capitulum at about 20 days (B), when the SAM is generating lateral leaves, about ten days later when the SAM has flattened to a disk (C), and then later floret formation on the disk.  \copyright World Scientific from Palmer~\cite{palmerPhysiologicalBasisPattern1998}.}%
{1}

 As the receptacle continues to grow, the vacant generative area stays approximately constant.%
 The seed-heads comprise  ray florets and disk florets: ray florets are male but sterile florets which often function to attract pollinators while the disc florets are the hermaphroditic fertile flowers which will develop into seeds. Some ray florets, which will develop into the large yellow organs which the layperson might call `petals', appear inside the bracts but on the outside of the ring; disc florets paired with ray florets form the body of the ring and then ray florets form again on the inside. Within the ring, the paired ray and disk florets form at primordia, where the disk floret appears first and the ray floret is greatly reduced in size~\autocite{hottonPossibleActualPhyllotaxis2006}.  Seed placement thus follows the placement of 
the disk floret primordia, and  in the capitulum typically displays falling phyllotaxis in which the parastichy counts decline as we move towards the most central, and most-recent primordium locations.

When the disc florets actually flower, the flowers tend to obscure the patterning of their primordium placement and it can appear that it is only the as-yet-unflowered disc floret that are regularly arranged, but as the floret heads decay the visibility of the regularity of the entire seed-head returns.    In some species (such as that on the cover of this book), the inner ray florets are not visible, but more often they can be seen at the centre of the mature seed-head.

It is hard to quantify the exact scale of the developmental arena on which primordium commitment is occurring, because of the delay between primordium commitment at the gene transcription level and visible microscopic change. But even in the absence of reliable genetic markers, commitment seems likely to be occurring in an arena of the order of half a centimetre wide. The  argument that there is no known global commitment mechanism to co-ordinate divergence angles becomes even stronger on an arena of this scale.    The fact of pattern generation on this scale has also been used as an argument that reaction-diffusion of the kind envisaged by Turing cannot be acting, as the diffusion times across the capitulum are much too long to allow dynamical pattern generation. Although it is unlikely for other reasons that simple diffusion is occurring, this argument is not strong.   Even though the particular mechanism is not similar to that hypothesised by Turing in 1952, the core idea that a local primordium placement function can create global pattern does not require instantly synchronised global information: all that is necessary is that there is a local pattern pre-established over developmental timescales, which we might think of as a morphogen field visualised by the bract placements in Figure~\ref{fig:F0707Hotton2006Fig6}. 
 

\pdffig{F0707Hotton2006Fig6}{Two successive images (A and B) of the same sunflower capitulum collected two days apart by casting with epoxy resin. C and D are respective close-ups showing how the large ray floret (marked with a black or white dot) and small disk floret are paired close together, and how the primordium patterns, and parastichies, take shape based on the pre-existing pattern at the edge of the capitulum. \copyright Springer, from~\autocite{hottonPossibleActualPhyllotaxis2006}.}{0.8}{\autocite{hottonPossibleActualPhyllotaxis2006}}


Hernandez and Palmer created a  revealing insight by artificially causing an annular wound to the generative area, reproduced in Figure~\ref{fig:F0708hernandez1988-79A}\autocite{hernandezRegenerationSunflowerCapitulum1988}. On the inside of this annulus, primordia production continues but the Fibonacci structure is lost. (Figure~\ref{fig:F0709hernandez1988B}). Moreover the first formed primordia on the inside of the wound develop into bracts rather than florets~\autocite{hernandezRegenerationSunflowerCapitulum1988}.%

\jpgfig%
{F0708hernandez1988-79A}%
{
	SEM microscopy of a sunflower head seven and nine days after wounding. Nine days after wounding a pattern of floret primordia has redeveloped. 
	The horizontal bar represents $300\mu${}m. 
	From~\autocite{hernandezRegenerationSunflowerCapitulum1988}
}%
{0.5}%
\jpgfig%
{F0709hernandez1988B}%
{Floret positions of an unwounded (top) and wounded (bottom) sunflower. From~\autocite{hernandezRegenerationSunflowerCapitulum1988}.}%
{0.5}%


This supports the idea that prior to wounding the tissue environment was already promoting primordium formation, but that the inhibitory effect of earlier primordia was lost by wounding; the relative uniformity of the post wounding primordium distribution also supports a local inhibitory effect of primordium commitment. The creation of bracts at the rim of the wound suggests that a commitment to bract rather than floret formation might be a consequence of a lack of local inhibition as well. 

  
\section{Notes}
\todo{Notee}