
%%%%%%%%%%%%%%%%%%%%%%%%%%%%%%%%%%555
\chapter{Some future directions}
\abstract{
Although it is the goal of this book to equip readers to decide future research priorities for themselves, here are some brief observations on the kinds of mathematical investigations that could contribute to a mathematical biology of phyllotaxis. These cover both what are conventionally called pure and applied mathematical topics. 
}

\section{Mathematical issues}
\subsection{The van Iterson diagram}
The van Iterson diagram is now of venerable age, and is normally analysed for monojugate lattices with $J=1$, although many writers have recognised that multijugate lattices with any fixed $J>1$ can also be described by a copy of the van Iterson lattice. These slice of lattice space can be glued together at the $h=0$ boundary, and this provides a path in lattice space along which parameters can be smoothly changed, but the examples of the stacked-coin model show that transitions between monojugate and multijugate lattices, which are commonly seem, are not transitions of this form. It would be valuable to re-visualise lattice space in ways that make these transitions seem more natural. 
The renormalisation transform that structures lattice space can itself be used to map between monojugate and multijugate lattices, and this may provide a tool for this.

\subsection{Cylindrical tilings}
The stacked-coin model is particularly useful for drawing attention to cylindrical tilings rather than lattices as a basic unit of pattern. Relative to planar polygonal tilings, which have been widely studied (and yet only yielded the surprise of the Penrose tiling in the 1980s), cylindrical tilings have been less intensively analysed, and again might be worth considering in the light of a tiling renormalisation that exploits the cylindrical periodicity. The stacked-coin model itself possesses mathematical structure, in the form of conserved quantities and the periodicity constraint, that are unlikely to have been fully worked out and may well yield a cornucopia of dynamical systems behaviour beyond the interesting cases already known.
 
\subsection{Regular patterns as attractors}
The central prize from more mathematical work on stacked-coin or other models would be to get a good general understanding of when and why lattice-like structures might be expected to emerge from the dynamical systems of phyllotaxis at all. For example, is it really the case that it is the cytokinin feedback that stabilises the dynamics of node formation order, as~\cite{besnardCytokininSignallingInhibitory2014} suggest, or is this an inevitable consequence of any evolvable regular pattern formation process?  We do not have an analytical expression for how slow geometrical change needs to be to preserve lattice structures nor do we have good understandings of whether different lattices (eg Fibonacci) are more stable than others under such changes. 

It is likely that other models, perhaps with softer-edged coins, or other longer range interactions, will smooth over the difficulty the hard stacked-coin model has in converging to lattice patterns, while still providing a simple connection, via the Hypothesis of Geometrical Phyllotaxis, with the appearance of Fibonacci structure. 

\section{Modelling challenges}
\subsection{The phyllotactic jump}

One missing link in the Standard Picture is the observed leap, in the \textit{Compositae} at least, between parastichy counts in the stem and in the disk floret. The common daisy has a handful of leaves, all branching at ground level, whose parastichy structure has never been systematically reported, then a stem which is completely smooth to the naked eye, before the bract, ray floret and disk floret structure appears. That the daisy usually has 13 bracts suggests a transition through 3 and 5 and 8 in the principal parastichies of the node pattern during development: at a minimum that requires 8 nodes, and probably rather more. But where are they? The situation is even more dramatic in the exaggerated seed disk of the sunflower. Sunflower stems have been the object of phyllotactic study, but typically report $(1,2)$ or $(2,3)$ parastichy counts if any at all can be evaluated. As we have seen, bract counts in the sunflower are not tightly clustered on Fibonacci numbers, though the Standard Picture can explain this away by saying that the bract generation region is not necessarily a tight annulus representing a single front. But even so the Standard Picture does demand that node placement in the bract region relies on a pattern predisposition with at least 13 nodes and probably more. Faced with the smooth underside of the disk capitulum, there is no visual evidence of the past trace of such a pattern. It is perhaps no coincidence that the major modern  proponent of the divergence angle as the driver of pattern, Okabe, has paid particular attention to the low order prepatterns that are in practice seen~\autocite{okabeBiophysicalOptimalityGolden2015}. Of course it is common in development for early commitments to be erased later, but whether this has left more subtle anatomical traces than are visible, or whether the right molecular probe at the right stage of development can yield the missing intermediate parastichy counts, remains to be seen. 
  Relatedly, the Standard Picture naturally explains Lucas, or double Fibonacci patterns as evolving developmentally from an initial $\gp{1,3}$ or $\gp{2,2}$ pattern say,
  and  Couder's insightful study linking stem- and seedhead-phyllotaxis provides some limited support for this~\cite{couderInitialTransitionsOrder1998}. 
  
   At a molecular level there is as yet only tantalisingly sparse experimental evidence about where this switching between branches actually occurs in development. Such evidence will be very experimentally demanding, but it is already the case that molecular probes visualising genetic or protein activity at cellular scales, stacked using confocal microscopy across the whole geometry of the shoot apical meristem, are providing whole new arenas in which models need to be developed and tested. 
  
 
\subsection{Macroscopic tests of the Standard Picture}
As we have seen, older theoretical views of stem form often envisaged the divergence as the central controlling parameter. In that viewpoint, the sequence of organ formation is essentially implicit: the 'next' organ is the one placed at the next rotation by the divergence. Correspondingly, changes in global phyllotactic counts have to be interpreted as changes in the order of node production, as in the otherwise  sophisticated statisical analyses of~\cite{guedonPatternIdentificationCharacterization2013}. However the Standard Picture, as presented in this book, rejects the idea of the divergence as a morphological parameter under direct genetic control, and correspondingly allows observed macroscopic pattern changes to be interpreted as the consequence of local node-node interaction dynamics. 


Even macroscopically, it is feasible to test the Standard Picture. Is the distribution of ordered but not strict Fibonacci patterns of the types enumerated in the MOSI experiment consistent in other studies? Is it consistent with deviation from Fibonacci seen in the fir cone studies? And are these distribution modellable as the outcomes of, say, stacked-coin models which differ only in the speed of geometric change? 
Columnar lattices, though a-priori rather obvious as good packings of relatively small disks on cylinders, were never looked for empirically until after modellers like Douady and Golé pointed out their existence as a class of model solutions. It is rather instructive that before there was a concept of such lattices it was very hard to see them, and now they are easily detectable. 


Beyond this, can the additional fitting power of cylindrical tilings that Doady and Golé found~\cite{douadyFibonacciQuasisymmetricPhyllotaxis2016} be shown to be statistically significant beyond the additional degrees of freedom, and is it feasible to use these as the basic unit of empirical observation? While recent work using discrete Fourier transforms to quantify phyllotactic patterns is valuable~\cite{negishiDeterminingParastichyPairs2022}, these tools need to be widened to consider cylindrical tiling patterns. Anyone who has analysed these patterns will yearn for a good image-analytic toolkit to take the drudgery out of finding the coordinates of nodes and there has been some promising recent work on this~\cite{aliyevMathematicalModelPlacement2023,aliyevStudyDistributionPhenotypic2024}.  As three-dimensional scanned representations of objects such as fir-cones become increasingly feasible at scale a new class of macroscopic analysis in three dimensions will become possible. 
 

When untangling the interplay of growth and development in the mature plant one useful principle is the near-universal lack of observations of organs significantly changing their angular position on the stem once committed, and the frequent observation of golden angle divergences in mature plants is a core constraint on any model of what is going on in the developmental stage. Some writers have gone further than this and argued that the developmental processes themselves must be controlled by divergence-based mechanisms, but it is the view of this book that this is quite untenable.  Instead it is my assumption that the high degree of order implied by Fibonacci parastichy counts can best be explained by a local mechanism repeatedly acting under the constraints of a slowly changing combination of genetic control and pre-existing pattern, and that there can in principle be a decent model representation of this process which locally produces lattices. 

\subsection{Universality across phyla}
Our most robust understanding of the molecular mechanism of Fibonacci phyllotaxis comes of course from \textit{Arabidopsis}, while the most useful pattern observations come from the \textit{Compositae}. Of course there are very many other phyllotactic patterns across the huge variety of land plants and beyond: there is evidence of Fibonacci structure in the bryophytes for example~\cite{gomez-campoPhyllotacticPatternsBryophyllum1974}. As Cronk wrote `it would be extremely instructive to examine cases in which phyllotactic patterns change within a plant\ldots and to correlate this with hormonally driven phase changes in plant development. It would be instructive too to consider how general the primordium auxin sink model is. The early microsurgical experiments (Wardlaw, 1950) were done with ferns, particularly Dryopteris, and the results are consistent with this model. However the microphylls of lycophytes and the phyllidia of bryophytes have precisely determined phyllotaxy and it would be interesting to know how similar the determining system is'~\autocite[p120]{cronkMolecularOrganographyPlants2009}. 
While most plant molecular biologists are only recently leaving the constraints of  \textit{Arabidopsis}, modellers are under no such constraint.
There is an opportunity to evaluate how adequate node-placement models are to explain observed pattern  across a wide range of plant phyla.

\section{Conclusion}
I wrote this book because I wanted to understand what the simple reason was for Fibonacci structure in lattice-like patterns of nodes in plants. I have been surprised to discover that one of the biggest gaps in my understanding is not  what the most appropriate biological model for node placement is, nor why lattices consistently transform through Fibonacci pairs,  but a lack of understanding how those models might deliver lattices at all. Rarely in mathematical biology, that is simultaneously a mathematical and a biological problem. 

Kuhlemeier's recent review of phyllotaxis concluded  that `regular patterning is simply an emergent property of the molecular mechanism of lateral inhibition', and that this was  `unsatisfactory'~\autocite{kuhlemeierPhyllotaxis2017}. 
The notion of an emergent property is a very slippery one, which theoretical biology has long struggled with: too useful to do without but far too often impossible to rigorously characterise. As a mathematical biologist I differ from Kuhlemeier about the degree of satisfaction mathematical biology can bring here. Fibonacci phyllotaxis is for once an emergent property in a mathematically rigorous, biologically consistent framework which generates novel hypotheses to advance scientific understanding. 

The grand biological project of the twenty-first century is to move from knowledge of the genome, a list of named DNA sequences, to understanding of the organism, an interaction of functional systems.  But genes are not mathematical objects: not only is there no gene that codes the number 55, still less can there be one that encodes the irrational golden ratio.
It will not be possible to fully describe these common, robust, developmental pathways of morphogenesis without the use of at least some mathematical structure to  understand `emergent' pattern. %\jNote
Fibonacci phyllotaxis is an existence proof for the necessity of a mathematical systems biology, with suitable modesty and grounding in the culture of biology, to play a role in this grand project. 

