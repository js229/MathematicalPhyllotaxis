

\preface


Over the past hundred years or more there has been a confusing variety of attempts to explain in mathematical language why Fibonacci numbers appear in plant spirals, but most modern accounts have now converged on a framework which might be called a Standard Picture. This book explains that mathematical framework and explores the extent to which it can explain Fibonacci phyllotaxis in the light of modern molecular biology.
 This Standard Picture relies on a biological model of how plants decide where to place individual new  organs over the course of development,
a model which when iterated tend to produce patterns we can model mathematically as lattices. Straight lines in these lattices are \textit{parastichies}, with an associated parastichy number counting how many of the parastichies make up the lattice, and there is a natural way to model what we do when we visually identify spirals as obvious that selects two of these parastihy numbers as the \textit{principal parastichy pair}.  This enables us to classify many model patterns by pairs of integers. Then we change a parameter of the model slowly, corresponding to some slow change -- almost always geometry -- in the underlying developmental process. This gives us a bifurcation tree for principal parastichy pairs as that parameter varies. The culmination of the Standard Picture is that, under quite general assumptions, Fibonacci-like structures form  stable branches of solutions.

It has been my goal in this book to equip readers to critique and participate in the applications of mathematical phyllotaxis.  I aim to give you an understanding of enough of the underlying biology to motivate this Standard Picture, to analyse it mathematically, and to evaluate it scientifically. Actually the main goal was to learn about the subject myself, but in writing the text I have imagined the needs of a recently graduated student of mathematics embarking on research in an interdisciplinary laboratory. (Since I don't fit any of these criteria, I would be particularly grateful for critique from those who do.) This is a textbook in which I have re-shaped and sometimes  re-derived a lot of other people's work into a story I find coherent and worth sharing. I have not felt obliged to provide a source for every statement or idea. Notes to each Chapter situate it in the literature for further reading. 



The first half of the book is an account of the mathematical theory of two-dimensional lattices. While lattice theory in higher dimensions has been an important area of mathematical research, the much simpler results we need in two dimensions are well-established, and I think intuitive. However my account is regrettably dense with Theorems and proofs. This was not my original aspiration, but evolved as I tried to make sense of multiple overlapping concepts, and occasional mathematical untruths, in the phyllotaxis literature.  So there is in places a density of mathematical argument of the pedantic kind I learned to loathe when a graduate student many decades ago. To my surprise, in the end I rather enjoyed writing these parts, but for those whose taste does not run this way they are largely skippable. I have put less technical summary sections at the ends of these chapters.  


The second half of the book connects the mathematical principles of the first half to biological observation by reviewing a range of mathematical models. This part reviews some basic plant embryology and molecular biology and introduces the apical meristem as the site of developmental and positional commitment. I will discuss older ideas, like Hofmeister's hypothesis that primordia are generated as soon as they are beyond a fixed distance from all other primordia, together with the modern molecular understanding of  auxin-based mechanisms of primordia initiation. Then we can begin the applied mathematical work of evaluating how well our lattice models perform, and what they do and don't explain. In particular, we will see the mathematical and biological motivations for generalising lattice models to stacked-coin models. Finally we'll  review some of the outstanding mathematical questions, both pure and applied, and how they can inform the wider and continuing scientific debate on the generation of form in biology.

%Finally, to rephrase the advice buried above: skip all the mathematics you want, especially on a first reading. 





